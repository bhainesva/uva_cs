\documentclass[paper=a4, fontsize=11pt]{jhwhw} % A4 paper and 11pt font size
\usepackage{amsmath,amsfonts,amsthm, amssymb} % Math packages
\setlength\parindent{0pt} % Removes all indentation from paragraphs - comment this line for an assignment with lots of text
\usepackage{graphicx}
\usepackage{verbatim}
\usepackage{enumerate}
\usepackage{mathtools}
\usepackage{color}
\newcommand\SetSymbol[1][]{\:#1\vert\:}
\providecommand\given{} % to make it exist
\DeclarePairedDelimiterX\Set[1]\{\}{\renewcommand\given{\SetSymbol[\delimsize]}#1}

\begin{document}
\title{Real Analysis - Assignment \# 3}
\author{Ben Haines}

\problem{1}
Let $A$ and $B$ be non-empty subsets of $\mathbb R$ such that $a\le b$ for all $a\in A$ and $b\in B$. Prove that $A$ is bounded above, $B$ is bounded below and $\sup(A)\le \inf(B)$. \textbf{Hint: }This can be proved directly from the definitions of supremum and infimum without any computations or using any theorems, but you need to proceed in two steps.
\solution
Select any element $b\in B$. Then for all $x\in A$, $x\le b$ so $b$ is an upper bound for $A$. Now select any element $a\in A$. Then for all $x\in B$, $a \le x$ so $a$ is a lower bound for $B$. Assume that $\inf(B) < \sup(A)$. Then there must exist some $x\in A$ such that $x> \inf(B)$ otherwise there would be an upper bound for $A$ less than $\sup(A)$. However, this means that $x$ is a lower bound for $B$ and $x > \inf(B)$. This is not possible. Thus $\sup(A) \le \inf(B)$. 

\problem{2}
Use induction to prove the formlua for the sum of a (finite) geometric progression: $a + ar + ar^2 + \ldots + ar^{n-1} = a\frac{1-r^n}{1-r}$ where $a,r \in \mathbb R$ and $r\not= 1$.
\solution
Base Case P(1):
$$a = a\frac{1-r}{1-r} = a$$
Assume P(k):
$$a + ar + \ldots + ar^{k-1} = a\frac{1-r^k}{1-r}$$
Prove P(k+1):
\begin{align*}
    a + ar + \ldots + ar^{k-1} + ar^k &= a\frac{1-r^k}{1-r} + ar^k\\
                                      &= a\left(\frac{1-r^k}{1-r} + r^k\right)\\
                                      &= a\left(\frac{1-r^k}{1-r} + \frac{(1-r)r^k}{1-r}\right)\\
                                      &= a\left(\frac{1-r^{k+1}}{1-r}\right)
\end{align*}


\problem{3}
Prove the following inequalities by induction:
\begin{enumerate}[(i)]
    \item $n < 2^n$ for all $n\in \mathbb N$
    \item $n^2 < 2^n$ for all integers $n\ge 4$ 
\end{enumerate}
\solution
\begin{enumerate}[i]
    \item 
        Base Case:
        $$1 < 2$$
        Assume P(k):
        $$k < 2^k$$
        Prove P(k+1):
        $$k < 2^k \implies 2k < 2^{k+1}$$
        $k\ge 1$ so $k+1 \le 2k$. Then by the transitive property, $k+1 < 2^{k+1}$.
    \item 
        First I'll prove that $2n + 1 < n^2$ for all $n \ge 5$.
        Base Case:
        $$11 < 25$$
        Assume $P(k)$:
        $$2k + 1 < k^2$$
        Prove $P(k+1)$:
        \begin{align*}
            2k + 1 &< k^2\\
            2k &< k^2\\
            2 &< k\\
            2 &< k^2\\
            3 &< k^2 + 1\\
            2k + 3 &< k^2 + 2k + 1\\
            2(k+1) + 1 &< (k+1)^2
        \end{align*}

        Now I'll use that result to prove that $n^2 < 2^n$ for all $n \ge 5$.
        Base Case:
        $$25 < 30$$
        Assume $P(k)$:
        $$k^2 < 2^k$$
        Prove $P(k+1)$:
        We know that $k^2 < 2^k$ so $2k^2 < 2^{k+1}$. By the result from above, $k^2 + 2k + 1 < k^2 + k^2 \implies (k+1)^2 < 2k^2$. Then by transitivity $(k+1)^2 < 2^{k+1}$.

\end{enumerate}


\problem{4}
Use induction to prove Bernoulli's inequality:
$$(1 + x)^n \ge 1 + nx\text{ for all }n\in \mathbb N\text{ and } x\ge -1$$
\solution
Base Case:
$$1 = (1+x)^0 \ge 1 = 1 + 0x$$
Assume P(k):
$$(1+x)^k \ge 1 + kx$$
Prove P(k+1):
\begin{align}
    (1 + x)^{k+1} &= (1+x)^k(1+x)\\
                  &\ge (1 + kx)(1+x)\\
    (1 + kx)(1+x) &= 1 + kx + kx^2\\
                  &= 1 + (k+1)x + kx^2\\
                  &\ge 1 + (k+1)x
\end{align}
So $(1+x)^{k+1} \ge 1 + (k + 1)x$

\problem{5}
Prove that the sequence converges to $L$, and explicitly find a function $M(\epsilon)$ satisfying $(1')$ above.
\begin{enumerate}[(i)]
    \item $a_n = \frac{2n^2 + 3}{n^2 - n - \cos(n)}$, $L = 2$
    \item $a_n = \frac{n}{4^n}$, $L = 0$
\end{enumerate}
\solution
\begin{enumerate}[(i)]
    \item Fix $\epsilon > 0$. $|a_n - 2| = \left|\frac{2n^2 + 3}{n^2 - n - \cos(n)} - 2\right| = \left|\frac{2n + 2\cos(n) + 3}{n^2-n-\cos(n)}\right|$. We want to find a simpler fraction to bound this from above. In order to do this we need the numerator of the new fraction to bound $2n + 2\cos(n) + 3$ from above and the denominator of the new fraction to bound $n^2 - n - \cos(n)$ from below. For all $n$ we know that $2\cos(n) \le 2n$ and $3 \le 3n$. This means the numerator can be bound from above by $2n + 2n + 3n = 7n$. Similarly with the denominator we know that for all $n$, $\cos(n) \le n$ so it can be bounded below by $n^2 - n$. As a whole the fraction can be bounded above  by $\frac{7n}{n^2-n} = \frac{7}{n-2}$. 

        Let $M(\epsilon) = \frac{7}{\epsilon} + 2$. Then $\forall n > M(\epsilon)$ we have $n > \frac{7}{\epsilon} + 2 \implies \epsilon > \frac{7}{n-2}$. Then by the calculations done above we have
        $$|a_n - 2| = \left|\frac{2n + 2\cos(n) + 3}{n^2 - n - \cos(n)}\right| \le \left|\frac{7}{n-2}\right| < \epsilon$$
        Thus the sequence converges to 2.

    \item Fix $\epsilon > 0$. We want to find a simpler fraction to bound $\left|\frac{n}{4^n}\right|$ from above. In order to do this we need the numerator of the new fraction to bound $n$ from above and the denominator of the new fraction to bound $4^n$ from below. For all $n$ we know that $n^2 < 4^n$ and $n \ge n$. Thus the fraction can be bounded above by $\left|\frac{n}{n^2}\right|$. We can disregard the absolute value signs because these fractions will always be positive.

        Let $M(\epsilon) = \frac{1}{\epsilon}$. Then $\forall n > M(\epsilon)$ we have $n > \frac{1}{\epsilon} \implies \epsilon > \frac{1}{n}$. Then by the calculations done above we have
        $$|a_n - 0| = \left|\frac{n}{4^n}\right| \le \frac{n}{n^2} < \epsilon$$
        Thus the sequence converges to 0.
\end{enumerate}

\problem{6}
Let ${a_n}$ and ${b_n}$ be sequences. Suppose that for every $\epsilon > 0$ the following is true: $|a_n - 3| < \epsilon$ for all $n > \frac{10}{e^2}$ and $|b_n - 4| < \epsilon$ for all $n> \frac{1}{e^3}$. Find an explicit function $M(\epsilon)$ such that $|a_n + b_n - 7| < \epsilon$ for all $n > M(\epsilon)$. 
\solution
Let $\epsilon' = \epsilon/2$. This is guaranteed to be a number larger than 0. Then $|a_n - 3| < \epsilon'$ for all $n > (10/(e/2)^2)$ and $|b_n - 4| < \epsilon'$ for all $n > (1/(\epsilon/2)^3)$. Define $M(\epsilon) = \max((10/(e/2)^2), (1/(\epsilon/2)^3))$. Then by the proof of part (i) of Theorem 2.12 for all $n > M(\epsilon)$, $|a_b + b_n - 7| < \epsilon' + \epsilon' = \epsilon$. 

\problem{7}
Let ${a_n}$ be a sequence, and define $b_k$ and $c_k$ (with $k\in \mathbb N$) by $b_k = a_{2k-1}$ and $c_k = a_{2k}$, that is ${b_k}$ and ${c_k}$ are subsequences of ${a_n}$ consisting of its elements located in odd (respectively even) position. Suppose that ${b_k}$ and ${c_k}$ both converge and $\lim_{k\to \infty}b_k = \lim_{k\to \infty}c_k = L$ for some $L\in \mathbb R$. Prove that ${a_n}$ converges to $L$ as well. 
\solution
For all $\epsilon > 0$ there exist $R_1, R_2$ such that $|b_k - L| < \epsilon$ for all $n > R_1$ and $|c_k - L| < \epsilon$ for all $n > R_2$. Let $M = \max(R_1, R_2)$. Then for any $a_n$ where $n > M$ if $a_n\in \{b_k\}$ then $|a_n - L| < \epsilon$. Similarly if $a_n\in \{c_k\}$ then $|a_n - L| < \epsilon$. Therefore the limit of $\{a_n\}$ is $L$.

\problem{8}
Let $f: X\to Y$ be a function. Prove that the following conditions are equivalent:
\begin{enumerate}[(a)]
    \item $f$ is injective
    \item $f(A\cap C) = f(A)\cap f(C)$ for any two subsets $A, C$ of $X$.
\end{enumerate}
\solution
First let $f$ be injective and consider any element $x\in f(A\cap C)$. $x = f(y)$ for some $y\in A\cap C$. Then $y\in A\implies x\in f(A)$ and $y\in B\implies x\in f(C)$ so $x\in f(A)\cap f(C)$. Now consider any $x\in f(A)\cap f(C)$. Then $x = f(a)$ for some $a\in A$ and $x = f(c)$ for some $c\in C$. $f$ is injective so $a = c\in A$ and $a=c\in C$. Then $x\in f(A\cap C)$. 

The second part is proved by contraposition. Assume that $f$ is not injective. Then we must show there are two subsets $A$ and $C$ of $X$ such that it is not the case that $f(A\cap C) = f(A) \cap f(C)$. We know by the fact that $f$ is not injective that there exist two elements $x_1, x_2\in X$ such that $f(x_1) = f(x_2)$ but $x_1 \not= x_2$. Let $A=\Set{x_1}$ and $C = \Set{x_2}$. Then $f(A\cap C) = \emptyset \not= f(A) \cap f(C) = \Set{f(x_1)}$. 
\end{document}
