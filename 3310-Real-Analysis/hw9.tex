\documentclass[paper=a4, fontsize=11pt]{jhwhw} % A4 paper and 11pt font size
\usepackage{amsmath,amsfonts,amsthm, amssymb} % Math packages
\setlength\parindent{0pt} % Removes all indentation from paragraphs - comment this line for an assignment with lots of text
\usepackage{enumerate}
\usepackage{mathtools}
\newcommand\SetSymbol[1][]{\:#1\vert\:}
\providecommand\given{} % to make it exist
\DeclarePairedDelimiterX\Set[1]\{\}{\renewcommand\given{\SetSymbol[\delimsize]}#1}

\begin{document}
\title{Real Analysis - Assignment \# 9}
\author{Ben Haines}

\problem{1}
\begin{align*}
    f'(x) &= \lim\limits_{h\to 0}\frac{(x+h)^n - x^n}{h}\\
          &= \lim\limits_{h\to 0}\frac{(x^n + {n\choose 1}x^{n-1}h + \cdots + h^n) - x^n}{h}\\
          &= \lim\limits_{h\to 0}\frac{{n\choose 1}x^{n-1}h + \cdots + h^n}{h}\\
          &= \lim\limits_{h\to 0}{n\choose 1}x^{n-1} + \cdots + h^{n-1}\\
          &= nx^{n-1}
\end{align*}
\problem{2}
\part
\begin{align*}
    f'(a) &= \lim\limits_{h\to 0} \frac{f(a+h) - f(a)}{h}\\
          &= \lim\limits_{h\to 0} \frac{f(a) + f(h) - f(a)}{h}\\
          &= \lim\limits_{h\to 0} \frac{f(h)}{h}
\end{align*}
This value is constant.
\part
\begin{align*}
    f'(a) &= \lim\limits_{h\to 0} \frac{f(a+h) - f(a)}{h}\\
          &= \lim\limits_{h\to 0} \frac{f(a)f(h) - f(a)}{h}\\
          &= \lim\limits_{h\to 0} \frac{f(a)(f(h) - 1)}{h}\\
          &= f(a) \lim\limits_{h\to 0} \frac{f(h) - 1}{h}\\
\end{align*}
The value of $\lim\limits_{h\to 0} \frac{f(h) - 1}{h}$ is constant so $f'(a) = c\cdot f(a)$ for some constant $C$.

\problem{3}
Fix some $\epsilon > 0$. 
Assume that $|x-y| < \delta$ for some $\delta$ to be determined. Then by MVT $|f(x) - f(y)| = |f'(c)(x-y)|$ for some $c$. However, the value of $f'(c)$ is bounded by $C$ so 
$$|f'(c)(x-y)| \le C|x-y| = C\delta < \epsilon$$
if $\delta < \frac{\epsilon}{C}$. 

\problem{4}
\part
By definition of a root there are at least $n$ values of $x$ for which $f(x) = 0$. Order these roots in increasing order. Then we can apply Rolle's Theorem to each consecutive pair of roots to obtain a minimum of $n-1$ values of $x$ where $f'(x) = 0$. 

\part
It is trivial that a polynomial of degree one has at most one root. Now assume that polynomials with degree $n$ have $n$ roots for all $n\le k$ for some $k$. Now consider a polynomial $f$ of degreen $k$. Assume that $f$ has more than $k$ roots. Then by problem 1 and part a) $f'$ has degree $<k$ but at least $k$ roots. This contradicts our inductive hypothesis and therefore polynomials with degree $n+1$ have at most $n+1$ roots. By the inductive hypothesis polynomials with degree $n\in \mathbb N$ have $n$ roots.

\problem{5}
We assume that $\lim\limits_{x\to +\infty} f(x)$ exists. Thus we can construct a sequence $x_n$ of real numbers such that $x_n$ goes to infinity. To select $x_n$ set $\epsilon = 1/n$. We know because the limit exists at infinity that there exists some $M_n$ such that $|f(x) - f(y)| < \epsilon$ for all $x, y \ge M_n$. Note that we can always increase $M_n$ such that it is larger than $n$ in order to ensure that the sequence goes to infinity. For each $n$ apply mean value theorem on the interval $[M, M+1]$. Then $f(M+1) - f(M) = f'(x_n)(M + 1 - M) \implies f'(x_n) = (f(M+l) - f(M))$. This is less than $1/n$. Therefore as $n\to \infty$, $f'(x_n) \to 0$. Then by the sequential characterization of limits at infinity, $\lim\limits_{x\to+\infty}f'(x) = 0$. 

\problem{6}
Let $f(x) = x^{1/2} + \frac{1}{2}\log(x) - x$. Then $f'(x) = \frac{1}{2\sqrt{x}} + \frac{1}{2x} - 1$ is less than 0 for all $x > 1$. This means that $f$ is a strictly decreasing function by Theorem 4.17. Thus $\forall x > 1$:
$$f(x) = x^{1/2} + \frac{1}{2}\log(x) - x > 0 \implies f(x) = x^{1/2} + \frac{1}{2}\log(x) >  x$$


\end{document}
