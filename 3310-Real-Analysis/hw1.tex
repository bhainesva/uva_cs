\documentclass[paper=a4, fontsize=11pt]{jhwhw} % A4 paper and 11pt font size
\usepackage{amsmath,amsfonts,amsthm, amssymb} % Math packages
\setlength\parindent{0pt} % Removes all indentation from paragraphs - comment this line for an assignment with lots of text
\usepackage{graphicx}
\usepackage{verbatim}
\usepackage{enumerate}
\usepackage{mathtools}
\usepackage{color}
\newcommand\SetSymbol[1][]{\:#1\vert\:}
\providecommand\given{} % to make it exist
\DeclarePairedDelimiterX\Set[1]\{\}{\renewcommand\given{\SetSymbol[\delimsize]}#1}

\begin{document}
\title{Real Analysis - Assignment \# 1}
\author{Ben Haines}

\problem{1}
\begin{enumerate}
    \item Prove $-(-x) = x$.
    \item Prove $-(xy) = (-x)y$.
\end{enumerate}
\solution
\part
\begin{flalign*}
    && 0 + -(-x) &= -(-x) && \text{(A3)}\\
    && [x + -x] + -(-x) &= -(-x) && \text{(A4)}\\
    && x + [-x + -(-x)] &= -(-x) && \text{(A2)}\\
    && x + 0 &= -(-x) && \text{(A4)}\\
    && x &= -(-x) &&\text{(A3)}
\end{flalign*}

\part
\begin{flalign*}
    && (-x)y + xy &= (-x + x)y && \text{(D)}\\
    && (-x)y + xy &= (0)y && \text{(A4)}
\end{flalign*}
In class it was proved that $0\cdot x = 0$ for all $x$. By this result I get
\begin{flalign*}
    && (-x)y + xy &= 0 &&\\
    && (-x)y + xy + -(xy) &= 0 + -(xy)&& \text{add} -(xy)\\
    && (-x)y + 0 &= 0 + -(xy) &&\text{(A4)}\\
    && (-x)y &= -(xy) &&\text{(A3)}
\end{flalign*}

\problem{2}
\begin{enumerate}
    \item Prove if $x>y, z<0$ then $xz < yz$.
    \item Prove if $x>y>0, z>w>0$ then $xz > yw$.
    \item Prove if $x>0$ then $x^{-1} > 0$.
\end{enumerate}
\solution
\part
In order to prove this I will first prove consequent 7 introduced in class, that $(-x)\cdot(y) = (-xy) = x\cdot(-y)$. 
\begin{flalign*}
    && x + [(-1)\cdot x] &= [1\cdot x] + [(-1)\cdot x]&&\text{(M3)}\\
    && x + [(-1)\cdot x] &= (1 + (-1))\cdot x&&\text{(D)}\\
    && x + [(-1)\cdot x] &= 0\cdot x&&\text{(A4)}\\
    && x + [(-1)\cdot x] &= 0&&\text{(consequent 3)}\\
    && -x + x + [(-1)\cdot x] &= -x + 0&&\\
    && 0 + [(-1)\cdot x] &= -x + 0&&\text{(A4)}\\
    && [(-1)\cdot x] &= -x &&\text{(A3)}
\end{flalign*}
Using the equivalence established above;
\begin{flalign*}
    &&(-x)(y) &= (-1\cdot x)\cdot(y)&&\\
    &&(-x)(y) &= -1\cdot (x\cdot y)&&\text{(D)}\\
    &&(-x)(y) &= -(xy)&&
\end{flalign*}
This shows that $(-x)(y) = -(xy)$. The argument that $-(xy) = (x)(-y)$ has an identical structure.

$z < 0$ and $-1 < 0$ so
\begin{flalign*}
    && 0\cdot -1 &< z \cdot (-1)&&\text{(O7)}\\
    && 0 &< z \cdot (-1)&&\text{consequent 3 proved in class}\\
    && 0 &< -(z\cdot 1)&& \text{consequent 7}\\
    && 0 &< -z&& \text{M3}\\
    && -z &> 0&& \text{definition of $>$}
\end{flalign*}

Having $-z > 0$ and $x > y$ I use (O6) to get $x(-z) > y(-z)$
\begin{flalign*}
    && -(xz) &> -(yz)&&\text{consequent 7}\\
    && -(xz) + ((xz) + (yz)) &> -(yz) + ((xz) + (yz))&&\text{(O4)}\\
    && -(xz) + ((xz) + (yz)) &> -(yz) + ((yz) + (xz))&&\text{(A1)}\\
    && (-(xz) + (xz)) + (yz) &> (-(yz) + (yz)) + (xz)&&\text{(M2)}\\
    && 0 + (yz) &> 0 + (xz)&&\text{(A4)}\\
    && (yz) &> (xz)&&\text{(A3)}
\end{flalign*}
By the definition of $>$ this is the same as saying $xz < yz$.

\part
It is given that $z>w$ and $w>0$. By (O2) $z > 0$. It is also given that $x>y$. Then by (O6) $xz>yz$. It is also given that $y>0$. By (O6) again $zy > wy$. Then by (M1) $yz > yw$ and finally by (O2) $xz > yw$.

\part
First assume that $x^{-1} < 0$. Then by (O7) proved in class:
$$x\cdot x^{-1} < 0\cdot x^{-1}$$
It was also proved in class that $0\cdot x = 0$ for all $x$. Thus,
\begin{flalign*}
    &&x\cdot x^{-1} &< 0 &&\\
    &&1 &< 0 &&\text{by (M4)}
\end{flalign*}
This is a contradiction with (O1) because I know that $1>0$, so $x^{-1} > 0$.

\problem{3}
Prove that there does not exist an $x\in \mathbb Z$ such that $0 < x < 1$. \\$\mathbb Z = \Set{x\in \mathbb R\given x\in \mathbb N \lor x = 0 \lor -x\in \mathbb N}$.
\solution
Consider any arbitrary $x\in \mathbb R$. There are three possible cases.
\begin{enumerate}
    \item Case 1: $x\in \mathbb N$\\
        It was proven in class that for all $x$ in $\mathbb N$, $x \ge 1$. Thus it is impossible that $x < 1$.
    \item Case 2: $x = 0$\\
        If $x=0$ then it is impossible that $x>0$.
    \item Case 3: $-x\in \mathbb N$\\
        By the same fact used in case 1, 
        \begin{flalign*}
            &&-x &\ge 1&&\\
            &&-x  + (-1)&\ge 1 + (-1)&&\\
            &&-x  + (-1)&\ge 0&&\text{(A4)}\\
            &&x + (-x)  + (-1)&\ge 0 + x&&\\
            &&0  + (-1)&\ge 0 + x&&\text{(A4)}\\
            &&-1&\ge x&&\text{(A3)}
        \end{flalign*}
        If $x = -1$ it is shown in problem 4 that $-1 < 0$. If $x < -1$ then by (O2) $x < 0$. So it is impossible that $x > 0$.
\end{enumerate}

There is no case in which it is possible that $0 < x < 1$. 

\problem{4}
Prove that it is impossible to define inequalities in $\mathbb C$ such that (O1)-(O4) hold.
\solution
The proof given in the book that for any nonzero $a\in \mathbb R$, $a^2 > 0$ depends only on axioms (O1)-(O4). Thus if these axioms held in $\mathbb C$ then it would have to be the case that the square of any nonzero element of $\mathbb C$ was greather than 0. However, $i$ is defined such that $i^2 = -1$. Using the fact introduced in class that $1>0$ I can say 
\begin{flalign*}
    &&1 + (-1) &> 0 + (-1)&&\text{(O4)}\\
    &&0 &> -1&&\text{(A4)}
\end{flalign*}
By axiom (O1) it is impossible for it also to be the case that $0 < -1$. Thus this is a contradiction. Therefore it is impossible to define inequalities in $\mathbb C$ in such a way that axioms (O1)-(O4) hold.

\problem{5}
\begin{enumerate}
    \item Let $x, y\in \mathbb R$. Prove $x\le y$ if and only if $x-\epsilon < y + \epsilon$  $\forall \epsilon > 0$.
    \item Let $x, y\in \mathbb R$ with $x<y$. Prove there exists $z\in \mathbb R$ with $x < z < y$.
    \item Let $a, x, b\in \mathbb R$ with $a < x < b$. Prove there exists $\epsilon > 0$ such that $a < x-\epsilon < x+\epsilon < b$. Deduce that $(x-\epsilon, x+\epsilon) \subset (a, b)$.
\end{enumerate}
\solution
\part
First let $x\le y$. By part (i) of theorem 1.9 proved in the textbook I know that $x < y + \epsilon$ for all $\epsilon > 0$. As done with $z$ in problem 2.a I can show that for any $\epsilon$, $-\epsilon < 0$. Thus by (O5) $x - \epsilon > y + \epsilon$.

Now let $x -\epsilon < y + \epsilon$ for all $\epsilon > 0$. Assume that $x > y$. Then $x-y > 0$ so I can set $\epsilon_0 = \frac{x-y}{3}$. Then plugging in I get $x - \frac{x-y}{3} < y + \frac{x-y}{3}$. $\epsilon_0 > 0$ so by (O5)
\begin{flalign*}
    &&x &< y + \epsilon_0 + \epsilon_0&&\\
    &&x + \epsilon_0 &< y + \epsilon_0 + \epsilon_0 + \epsilon_0&&\text{(O5)}\\
    &&x + \epsilon_0 &< y + (x + (-y))&&\\
    &&x + \epsilon_0 &< y + ((-y) + x)&&\text{(A1)}\\
    &&x + \epsilon_0 &< (y + (-y)) + x&&\text{(A2)}\\
    &&x + \epsilon_0 &< 0 + x&&\text{(A4)}\\
    &&x + \epsilon_0 &< x&&\text{(A3)}\\
    &&(-x) + x + \epsilon_0 &< (-x) + x&&\text{(O4)}\\
    &&0 + \epsilon_0 &< 0&&\text{(A4)}\\
    &&\epsilon_0 &< 0&&\text{(A3)}
\end{flalign*}
This is a contradiction with (O1) because I know that $\epsilon_0 > 0$. Therefore $x \le y$.

\part
Let $n$ be the largest natural number such that $\frac{1}{n} < y-x$. Let $k$ be the largest natural number such that $\frac{k}{n}\le x$. Then by our selection of $k$, $\frac{k+1}{n} > x$. 

Now assume that $y \le \frac{k+1}{n}$. Then I have that $\frac{k +1}{n} \ge y$ and $-\frac{k}{n} \ge -x$ so by (O5)'':
$$\frac{1}{n} = \frac{k+1}{n} - \frac{k}{n} \ge y-x$$.
This is a contradiction so it must be the case that $y > \frac{k+1}{n}$. Thus $z = \frac{k+1}{n}$ is a number satisfying $x<z<y$.

\part
First I prove consequent 5 introduced in class that $-(x-y) = y-x$.
\begin{flalign*}
    &&-(x-y) &= -(x + (-y))&&\text{def. of -}\\
    &&-(x-y) &= -1\cdot(x + (-y))&&\text{see prob. 2}\\
    &&-(x-y) &= -1\cdot x + -1\cdot(-y))&&\text{(D)}\\
    &&-(x-y) &= -x + -(-y))&&\text{see prob. 2}\\
    &&-(x-y) &= -x + y&&\text{proved in class}\\
    &&-(x-y) &= y + (-x)&&\text{(A1)}\\
    &&-(x-y) &= y - x&&\text{def. of -}
\end{flalign*}

Let $y$ be the smaller value of $b-x$ and $x-a$, both of which are positive. If $y = x-a$ then 
\begin{flalign*}
    &&x-y &= x + (-(x - a))&&\\
    &&x-y &= x + (a + (-x))&&\text{consequent 5}\\
    &&x-y &= x + ((-x) + a)&&\text{(A1)}\\
    &&x-y &= (x + (-x)) + a&&\text{(A2)}\\
    &&x-y &= 0 + a&&\text{(A4)}\\
    &&x-y &= a&&\text{(A3)}
\end{flalign*}

$a\le a$ by definition so in this case $a\le x-y$.

The other case is when $y = b-x$. By our selection of $y$ I know $x-a > y$ so $(x-a)-y > 0$ and I also know from the first case that $x-(x-a)\ge a$. So by (O5)'
\begin{flalign*}
    &&x + (-(x-a)) + ((x-a) + (-y))&\ge a&&\\
    &&x + ((-(x-a)) + (x-a)) + (-y)&\ge a&&\text{(A2)}\\
    &&x + 0 + (-y)&\ge a&&\text{(A4)}\\
    &&x + (-y)&\ge a&&\text{(A3)}
\end{flalign*}
so in both cases $a \le x - y$.
It is given that both $b-x$ and $x-a$ are positive so in either case $y > 0$. By (O4) $x + y > x$. Application of (O5) obtains $x > x-y$.

Now I want to show that $x + y \le b$. In the case when $y=b-x$
\begin{flalign*}
    &&x + y &= x + (b + (-x))&&\\
    &&x + y &= x + ((-x) + b)&&\text{(A1)}\\
    &&x + y &= (x + (-x)) + b&&\text{(A2)}\\
    &&x + y &= 0 + b&&\text{(A4)}\\
    &&x + y &= b&&\text{(A3)}
\end{flalign*}
so $x + y \le b$ by definition.

In the case when $y = x-a$ I know by our selection of $y$ that $y < (b-x)$ and thus $0 > y + (-(b-x))$ by (O4). I also know from the first case that $b \ge x + (b-x)$. Then by (O5)'
\begin{flalign*}
    &&b + 0 &> x + (b-x) + (y + (-(b-x)))&&\\
    &&b + 0 &> x + (b-x) + ((-(b-x)) + y)&&\text{(A1)}\\
    &&b + 0 &> x + ((b-x) + (-(b-x))) + y&&\text{(A2)}\\
    &&b + 0 &> x + 0 + y&&\text{(A4)}\\
    &&b &> x + y&&\text{(A3)}\\
\end{flalign*}
So therefore $x + y \le b$ by the definition of $\le$.

Having shown that $x < x + y$ I can use the result of part b) to produce some number $z$ such that $x < z < x + y$. Let $\epsilon = z-x$. To show that $\epsilon$ satisfies the desired characteristics it must be shown that $\epsilon > 0$, and that $a < x-\epsilon < x + \epsilon < b$. It is known that $x < z$ so by(O4) $\epsilon = z-x > 0$. By the same process as before, $0 > -\epsilon$. Then by (O2) $-\epsilon < \epsilon$ and by (O4) $x-\epsilon < x + \epsilon$.\\

It is known that $z < x + y$. Then by (O4) $z-x < y$ so $\epsilon < y$. It is also known $a \le x-y$. Then by (O5)' $a + \epsilon < x$ and by (O4) $a < x-\epsilon_0$.\\

From above it is known that $x + y \le b$ and that $\epsilon < y$. Then by (O5)' $x + y + \epsilon< b + y$ and by (O4) $x + \epsilon < b$.\\

Thus $\epsilon$ satisfies the desired properties and by definition $(x-\epsilon, x + \epsilon)\subset (a, b)$.

\problem{6}
Prove that each of the following are metric spaces.
\begin{enumerate}
    \item $X= \mathbb R, d(x, y) = |y-x|$
    \item $X = \text{any set}, d(x, y) = 1$ if $x\not= y$ and $d(x, y) = 0$ if $x = y$.
    \item Give another example of a metric space.
\end{enumerate}
\solution
\part
This proof will use the fact that $-1\cdot x = -x$. This was proven as an intermediate step in problem 2. 

First I will prove that $-(x-y) = y-x$.
\begin{flalign*}
    &&-(x-y) &= -1 \cdot (x + (-y))&&\text{see problem 2}\\
    &&-(x-y) &= -1 \cdot x + -1\cdot -y&&\text{(D)}\\
    &&-(x-y) &= -x + -(-y)&&\text{see problem 2}\\
    &&-(x-y) &= -x + y&&\text{proved in class}\\
    &&-(x-y) &= y + (-x)&&\text{(A1)}\\
    &&-(x-y) &= y - x&&\text{def. of -}
\end{flalign*}

\begin{enumerate}[i]
    \item $d(x, y) = 0 \iff x = y$\\
        First assume $x = y$. Then $|y-x| = |0| = 0$.
        Now assume that $|y-x| = 0$. Then either $y-x = 0$ or $x-y=0$. In the first case $y - x + x = x$ so by (A4) $y = x$. In the second case $x - y + y = y$ so by (A4) $x = y$.
    \item $d(x, y) = d(y, x)$\\
        This would directly follow from a proof of property 2 of absolute valuesthat states $|y - x| = |x - y|$. There are two cases.
        \begin{enumerate}
            \item Case: $y-x > 0$.\\
                By the definition of absolute value $|y-x| = y-x$. Then
                \begin{flalign*}
                    &&y-x &> 0&&\\
                    &&y-x + x &> 0 + x&&\text{O4}\\
                    &&y + 0 &> 0 + x&&\text{A4}\\
                    &&y &> x&&\text{A3}\\
                    &&y + (-y) &> x + (-y)&&\text{O4}\\
                    &&0 &> x + (-y)&&\text{A4}\\
                    &&0 &> x - y&&\text{def. of -}
                \end{flalign*}
                Thus by the definition of absolute value $|x-y| = -(x-y)$ which, as proved at the beginning of this problem, is equal to $y-x$. 
                \begin{enumerate}
                    \item Case: $y-x < 0$.\\
                        By the definition of absolute value $|y-x| = -(y-x)$. Using the same fact as above, this equals $x-y$.
                    \begin{flalign*}
                        &&y-x &< 0&&\\
                        &&y-x + x &< 0 + x&&\text{O4}\\
                        &&y + 0 &< 0 + x&&\text{A4}\\
                        &&y &< x&&\text{A3}\\
                        &&y + (-y) &< x + (-y)&&\text{O4}\\
                        &&0 &< x + (-y)&&\text{A4}\\
                        &&0 &< x - y&&\text{def. of -}
                    \end{flalign*}
                    Thus $|x-y| = x-y$ by definition.
                    \item Case: $y-x = 0$\\
                        In this case $|y-x| = y-x = 0$ by definition.
                        \begin{flalign*}
                            &&y-x &= 0&&\\
                            &&y-x+x &= 0 + x&&\\
                            &&y + 0 &= 0 + x&&\text{(A4)}\\
                            &&y &= x&&\text{(A3)}\\
                            &&y + (-y) &= x + (-y)&&\\
                            &&0 &= x + (-y)&&\text{(A4)}\\
                            &&0 &= x - y&&\text{def. of -}\\
                        \end{flalign*}
                        So $|x-y| = y-x = 0$ by definition.
                \end{enumerate}
            \end{enumerate}

    \item $d(x, z) \le d(x, y) + d(y, z)$\\
        $|z - x| \le |y - x| + |z - y|$ by the triangle inequality proved in class.
\end{enumerate}

\part
\begin{enumerate}[i]
    \item $d(x, y) = 0 \iff x = y$\\
        This is true by the definition of the function $d$.
    \item $d(x, y) = d(y, x)$\\
        In the case when $x = y$, $d(x, y) = 0 = d(y, x)$.
        In the case when $x\not= y$, $d(x, y) = 1 = d(y, x)$.
    \item $d(x, z) \le d(x, y) + d(y, z)$
        \begin{enumerate}
            \item Case: $x = y = z$\\
                $0 \le 0$
            \item Case: $x \not= y \not= z$\\
                $1 \le 2$
            \item Case: $x = y\not= z$\\
                $1\le 1$
            \item Case: $x \not= y = z$\\
                $1\le 1$
            \item Case: $x = z \not= y$\\
                $0\le 1$
        \end{enumerate}
\end{enumerate}

\part
$X = \mathbb C, d(x, y) = \sqrt{x^2 + y^2}$.
\end{document}
