\documentclass[paper=a4, fontsize=11pt]{jhwhw} % A4 paper and 11pt font size
\usepackage{amsmath,amsfonts,amsthm, amssymb} % Math packages
\setlength\parindent{0pt} % Removes all indentation from paragraphs - comment this line for an assignment with lots of text
\usepackage{graphicx}
\newcommand\SetSymbol[1][]{\:#1\vert\:}
\providecommand\given{} % to make it exist
\DeclarePairedDelimiterX\Set[1]\{\}{\renewcommand\given{\SetSymbol[\delimsize]}#1}
\DeclareMathOperator{\lcm}{lcm}

\begin{document}
\title{MATH 3310 - Assessment}
\author{Ben Haines (bmh5wx)}
%SECTION 4.5
\problem{\# 0}
MATH 3354 - Fall 2014

\problem{\# 1}
Assume that $\sqrt{12}$ is rational. Then it can be written in the form $\frac{P}{Q}$ for some integers $P$ and $Q$. We can assume that this fraction is in lowest terms without loss of generality. Then
$$\sqrt{12} = \frac{P}{Q} = \frac{12}{\sqrt{12}} = \frac{12Q}{P}$$
It is clear that 12 is not a perfect square so each of the above representations has a nonzero fractional component. For $\frac{P}{Q}$ it can be represented as $\frac{q}{Q}$ where $q < Q, q\not=0$. For $\frac{12Q}{P}$ it can be represented as $\frac{p}{P}$ where $p < P, p\not=0$. These fractional components must be equal to each other, therefore
$$\frac{p}{P} = \frac{q}{Q} \implies \frac{p}{q} = \frac{P}{Q}$$
This is a contradiction with the assumption that $\frac{P}{Q}$ was in lowest terms, therefore $\sqrt{12}$ is irrational.

\problem{\# 2}
First we show that the statement is true for $n=2$. 
$$a_2 = 2a_{n-1} - 3 = 2(4) - 3 = 5 = 2^1 + 3$$
Now assume that the statement is true for some $k > 1$.
$$a_k = 2^{k-1} + 3$$ 
and show that this implies truth for $k+1$.
$$a_{k+1} = 2a_{k} - 3 = 2(2^{k-1} + 3) - 3 = 2^k -3$$
Therefore the statement is true for all integers $n > 1$. 

\problem{\# 3}
\begin{enumerate}
    \item
        Injective, surjective.
    \item
        Injective, not surjective.
    \item
        Not injective, not surjective.
    \item
        Not injective, surjective.
\end{enumerate}

\problem{\# 4}
\begin{enumerate}
    \item True.\\
        By definition of $f^{-1}(B)$, the mappings of each of the elements in that set must be in $B$ so $f(f^{-1}(A)) \subseteq A$. 
    \item False.\\
        Let $X = Y = \mathbb Z$ and $B = \Set{1, 2, 3}$. Let $f$ be defined by $f(x) = 1$. Then $f(f^{-1}(B)) = \Set{1}$. $B$ is not a subset of this set. 
    \item False.\\
         Let $X = \Set{0, 1}$, $A = \Set{0}$, and $C = \Set{1}$. Let $f$ be defined by $f(x) = 1$. Then $f(A\cap C) = \emptyset$ while $f(A)\cap f(C) = \Set{1}$.
    \item True.\\
         For any element $x\in f^{-1}(B\cap D)$, $f(x)\in B$ and $f(x)\in D$ implies that $x\in f^{-1}(B)\cap f^{-1}(D)$. Thus $f^{-1}(B\cap D) \subseteq f^{-1}(B)\cap f^{-1}(D)$. \\

        For any $x$ in $f^{-1}(B)\cap f^{-1}(D)$, $f(x)\in D$ and $f(x)\in B$ implies that $f(x)\in B\cap D$ and s $f^{-1}(B)\cap f^{-1}(D) \subseteq f^{-1}(B\cap D)$. \\

        Therefore $f^{-1}(B\cap D) = f^{-1}(B) \cap f^{-1}(D)$
\end{enumerate}
\end{document}
