\documentclass[paper=a4, fontsize=11pt]{jhwhw} % A4 paper and 11pt font size
\usepackage{pgfplots}
\pgfplotsset{compat=1.9}
\usepackage{amsmath, amssymb}
\newcommand\SetSymbol[1][]{\:#1\vert\:}
\providecommand\given{} % to make it exist
\DeclarePairedDelimiterX\Set[1]\{\}{\renewcommand\given{\SetSymbol[\delimsize]}#1}
\usepackage{listings}
\usepackage{color}
\DeclareSymbolFont{extraup}{U}{zavm}{m}{n}
\DeclareMathSymbol{\varheart}{\mathalpha}{extraup}{86}
\DeclareMathSymbol{\vardiamond}{\mathalpha}{extraup}{87}
\definecolor{dkgreen}{rgb}{0,0.6,0}
\definecolor{gray}{rgb}{0.5,0.5,0.5}
\definecolor{mauve}{rgb}{0.58,0,0.82}

\lstset{frame=tb,
  language=Python,
  aboveskip=3mm,
  belowskip=3mm,
  showstringspaces=false,
  columns=flexible,
  basicstyle={\small\ttfamily},
  numbers=none,
  numberstyle=\tiny\color{gray},
  keywordstyle=\color{blue},
  commentstyle=\color{dkgreen},
  stringstyle=\color{mauve},
  breaklines=true,
  breakatwhitespace=true,
  tabsize=3
}
\begin{document}
\title{Discrete Math - Homework Set \#9}
\author{Ben Haines (bmh5wx)}
\problem{\#1}
Give an example of a function from $\mathbb Z$ to $\mathbb Z$ with each of the following characteristics.
\begin{enumerate}
    \item onto but not one-to-one
    \item one-to-one but not onto
    \item both onto and one-to-one (don't use the identity functioni as your example)
    \item neither one-to-one nor onto
\end{enumerate}
\solution

\problem{\#2}
Determine which of the following properties are true and which are false. If the propert is true give a proof and if it is false give a counter example. Let $f: X\to Y$ and $A, B\subseteq X$. 
\begin{enumerate}
    \item If $A\subseteq B$, then $f(A)\subseteq f(B)$.
    \item For all subsets $A$ and $B$ of $X$, $f(A\cup B) = f(A)\cup f(B)$.
    \item For all subsets $A$ and $B$ of $X$, $f(A\cap B) = f(A)\cap f(B)$.
    \item For a subset $C$ of $Y$, $f^{-1}(C) = \overline{f^{-1}(C)$.
    \end{enumerate}
\solution

\problem{\#3}
Prove or disprove the following statements.
\begin{enumerate}
    \item If $f:\mathbb R\to \mathbb R$ and $g:\mathbb R\to \mathbb R$ are both 1-1 then $f+g$ is also 1-1.
    \item If $f:\mathbb R\to \mathbb R$ and $g:\mathbb R\to \mathbb R$ are both onto then $f+g$ is also onto.
    \item Let $f:\mathbb R\to \mathbb R$ and $c$ is any nonzero real number. If $f$ is 1-1, then $c\cdot f$ is also 1-1.
    \item Let $f:\mathbb R\to \mathbb R$ and $c$ is any nonzero real number. If $f$ is onto, then $c\cdot f$ is also onto.
\solution

\problem{\#4}
Prove the following. Given any set $X$ and given any functions $f:X\to X$, $g:X\to X$, and $h:X\to X$, if $h$ is 1-1 and $h\circ f = h\circ g$, then $f=g$.
\solution

\problem{\#5}
Let $f:X\to Y$ and $g:Y\to Z$ must be functions.
\begin{enumerate}
    \item If $g\circ f$ is 1-1, must $f$ and $g$ be1 1-1 as well? Prove or give a counter example.
    \item If $g\circ f$ is onto, must $f$ and $g$ be onto as well? Prove or give a counter example.
\end{enumerate}
\solution

\problem{\#6}
Suppose that $f$ is a function from $A$ to $B$, where $A$ and $B$ are finite sets with $\mid A\mid = \mid B\mid$. Show that $f$ is one-to-one if and only if it is onto.
\solution

\problem{\#7}
Show that a set $S$ is infinite if and only if there is a proper subset $A$ of $S$ such that there is a one-to-one correspondence between $A$ and $S$.
\solution

\problem{\#8}
Let $S$ be a subset of a universal set $U$. The \textbf{characteristic function} $f_s$ of $S$ is the function from $U$ to the set $\Set{0, 1}$ such that $f_s(x) = 1$ if $x$ belongs to $S$ and $f_s(x) = 0$ if $x$ does not belong to $S$. Let $A$ and $B$ be sets. Show that for all $x\in U$:
\begin{enumerate}
    \item $f_{A\cap B}(x) = f_A(x) \cdot f_B(x)$
    \item $f_{A\cup B}(x) = f_A(x) + f_B(x) - f_A(x) \cdot f_B(x)$
    \item $f_A(x) = 1 - f_A(x)$
    \item $f_{A\oplus B}(x) = f_A(x) + f_B(x) - 2f_A(x) \cdot f_B(x)$
\solution

\problem{\#1}
\solution


\end{document}
