\documentclass[paper=a4, fontsize=11pt]{jhwhw} % A4 paper and 11pt font size
\usepackage{amsmath,amsfonts,amsthm, amssymb} % Math packages
\setlength\parindent{0pt} % Removes all indentation from paragraphs - comment this line for an assignment with lots of text
\usepackage{graphicx}

\begin{document}
\title{Discrete Math - Homework Set \#2}
\author{Ben Haines (bmh5wx)}
\problem{}
Prove or disprove that the logical operators exclusive or, NAND and NOR are associative.
\solution
\part
To determine if XOR is associative we need to see if $(A\otimes B)\otimes C = A\otimes(B\otimes C)$. 
\begin{displaymath}
\begin{array}{|c|c|c|c|c|c|c}
   A
 & B
 & C
 & A\otimes B
 & B\otimes C
 & (A\otimes B)\otimes C
 & A\otimes(B\otimes C) \\
\hline
F & F & F & F & F & F & F \\
F & F & T & F & T & T & T \\
F & T & F & T & T & T & T \\
F & T & T & T & F & F & F \\
\hline
T & F & F & T & F & T & T \\
T & F & T & T & T & F & F \\
T & T & F & F & T & F & F \\
T & T & T & F & F & T & T \\
\hline
\end{array}
\end{displaymath}


They are equivalent for all values of $A,B,$ and $C$ so XOR is associative.
\part
To determine if NAND is associative we need to see if $(A\uparrow B)\uparrow C = A\uparrow(B\uparrow C)$. 
\begin{displaymath}
\begin{array}{|c|c|c|c|c|c|c}
   A
 & B
 & C
 & A\uparrow B
 & B\uparrow C
 & (A\uparrow B)\uparrow C
 & A\uparrow (B\uparrow C) \\
\hline
F & F & F & T & T & T & T \\
F & F & T & T & T & F & T \\
F & T & F & T & T & T & T \\
F & T & T & T & F & F & T \\
\hline
T & F & F & T & T & T & F \\
T & F & T & T & T & F & F \\
T & T & F & F & T & T & F \\
T & T & T & F & F & T & T \\
\hline
\end{array}
\end{displaymath}
They are not equivalent for all values of $A,B,$ and $C$ so NAND is not associative.

\part
To determine if NOR is associative we need to see if $(A\downarrow B)\downarrow C = A\downarrow(B\downarrow C)$. 
\begin{displaymath}
\begin{array}{|c|c|c|c|c|c|c}
   A
 & B
 & C
 & A\downarrow  B
 & B\downarrow  C
 & (A\downarrow  B)\downarrow  C
 & A\downarrow  (B\downarrow  C) \\
\hline
F & F & F & T & T & F & F \\
F & F & T & T & F & F & T \\
F & T & F & F & F & T & T \\
F & T & T & F & F & F & T \\
\hline
T & F & F & F & T & T & F \\
T & F & T & F & F & F & F \\
T & T & F & F & F & T & F \\
T & T & T & F & F & F & F \\
\hline
\end{array}
\end{displaymath}
They are not equivalent for all values of $A,B,$ and $C$ so NOR is not associative.


\problem{}
Let
\[
S_i = \biggl\{\, x \in R \mathrel{\Big|} 1<x<1+\frac{1}{i}\, \text{ for all positive integers }\,i \,\biggr\}
\]
\begin{enumerate}
\item Are the sets $S_1,S_2,S_3,\ldots$ mutually disjoint? Explain.
\item $\bigcup^{\infty}_{i = 1} S_i = ? $
\item $\bigcap^{\infty}_{i = 1} S_i = ? $
\end{enumerate}
\solution
\part
The sets are not mutually disjoint. For example $1.5\in S_1$ and $1.5\in S_2$.
\part
For every set $S_i$, $S_i\subseteq S_1$. Therefore their union is equal to $S_1$.
\part
The intersection of the sets is $=\emptyset$. As $i$ approaches infinity, in order for a number $x$ to be a member of $S_i$ it would have to satisfy the condition $1<x<1+\frac{1}{\infty}$. This is equivalent to saying that $1<x<1$ which is clearly not possible to satisfy.
\problem{}
Let $A$ and $B$ be two finite sets. Let $\mathcal P(X)$ denote the power set of the set $X$. Prove or disprove the following:
\begin{enumerate}
\item If $A \subseteq B$, then $\mathcal P(A) \subseteq \mathcal P(B)$
\item If $\mathcal P(A) \subseteq \mathcal P(B)$, then $A \subseteq B$
\end{enumerate}

\solution
\part
$A\subseteq B$ so any subset of $A$ is also a subset of $B$. So for any $S\in \mathcal P(A)$, $S\in \mathcal P(B)$ and by definition of a subset $\mathcal P(A) \subseteq \mathcal P(B)$.
\part
$A\subseteq A$ so we know that $A\in \mathcal P(A)$. $\mathcal P(A)\subseteq \mathcal P(B)$ so therefore $A\in \mathcal P(B)$. Thus by the definition of the power set $A\subseteq B$.


\problem{}
For any sets $A,B,C,$ and $D$, prove that if $A\subseteq B$ and $C\subseteq D$, then $A\times C\subseteq B\times D$, where $X\times Y$ denotes the Cartesian product of the sets $X$ and $Y$.

\solution
Let $(a,c)$ represent any given element in $A\times C$ where $a\in A$ and $c\in C$. For every element $a\in A\,\Rightarrow a\in B$ by the definition of a subset. Similarly for every element $c\in C\,\Rightarrow c\in D$. Therefore for any $(a, c)\in A\times C$ there exists an element $(b, d)\in B\times D$ such that $a=b$ and $c=d$. Thus $A\times C\subseteq B\times D$.

\problem{}
Represent the common form of the following argument using letters to stand for a component sentence, and fill in the blanks so that the argument in part (b) has the same logical form as the argument in part (a).
\begin{enumerate}
\item If all computer programs contain errors, then this program contains an error.\\
This program does not contain an error.\\
Therefore, it is not the case that all computer programs contain errors.
\item If $\ldots$, then $\ldots$\\
2 is not odd.\\
Therefore, it is not the case that all prime numbers are odd.
\end{enumerate}
\solution
(b) should be written as
\begin{itemize}
\item If all prime numbers are odd, then 2 is odd\\
2 is not odd.\\
Therefore, it is not the case that all prime numbers are odd.
\end{itemize}


\problem{}
Write the following statements in symbolic form using the symbols $\neg, \land, \lor$ and the indicated letters to represent component statements.
\begin{enumerate}
\item Juan is a math major but not a computer science major. ($m=$"Juan is a math major", $c=$"Juan is a computer science major")
\item Let $h=$"John is healthy," $w=$"John is wealthy," and $s=$"John is wise."\\
John is not wealthy but he is healthy and wise.
\item John is wealthy, but he is not both healthy and wise.
\item Either Olga will go out for tennis or she will go out for track but not both. ($n=$"Olga will go out for tennis," $k=$"Olga will go out for track")
\end{enumerate}

\solution
\part
$$(m\land \neg c)$$
\part
$$\neg w \land (h\land s)$$
\part
$$w\land \neg(h\land s)$$
\part
$$(n\lor k)\land \neg(n\land k)$$


\problem{}
Are the following two statements logically equivalent? Justify your answers using truth tables and include a few words of explanation:
\begin{enumerate}
\item $\neg(p\land q)$ and $\neg p\land \neg q$
\item $(p\lor q)\lor(p\land r)$ and $(p\lor q)\land r$
\item $\neg(p\lor q)$ and $\neg p\land \neg q)$
\end{enumerate}
\solution
\part
\begin{displaymath}
\begin{array}{|c|c|c|c|c|c|c}
   p
 & q
 & \lnot{}p
 & \lnot{}q
 & p\land{}q
 & \lnot{}(p\land{}q)
 & \lnot{}p\land{}\lnot{}q \\
\hline
F & F & T & T & F & T & T \\
F & T & T & F & F & T & F \\
T & F & F & T & F & T & F \\
T & T & F & F & T & F & F \\
\hline
\end{array}
\end{displaymath}\\
From the table we can see that $\neg(p\land q)$ does not have the same truth value as $\neg p \land \neg q$ for all truth values of $p$ and $q$. Therefore the two statements are not logically equivalent.
\part
\begin{displaymath}
\begin{array}{|c|c|c|c|c|c|c}
   p
 & q
 & r
 & p\lor{}q
 & p\lor{}r
 & (p\lor{}q)\lor{}(p\lor{}r)
 & ((p\lor{}q)\lor{}r) \\
\hline
F & F & F & F & F & F & F \\
F & F & T & F & T & T & T \\
F & T & F & T & F & T & T \\
F & T & T & T & T & T & T \\
\hline
T & F & F & T & T & T & T \\
T & F & T & T & T & T & T \\
T & T & F & T & T & T & T \\
T & T & T & T & T & T & T \\
\hline
\end{array}
\end{displaymath}\\
For all possible combinations of truth values for $p,q$, and $r$ the expressions $(p\lor q)\lor(p\land r)$ and $(p\lor q)\land r$ have the same overall truth value. Thus they are logically equivalent.
\part
\begin{displaymath}
\begin{array}{|c|c|c|c|c|c|c}
   p
 & q
 & \lnot{}p
 & \lnot{}q
 & p\lor{}q
 & \lnot{}(p\lor{}q)
 & \lnot{}p\land{}\lnot{}q \\
\hline
F & F & T & T & F & T & T \\
F & T & T & F & T & F & F \\
T & F & F & T & T & F & F \\
T & T & F & F & T & F & F \\
\hline
\end{array}
\end{displaymath}\\
For all possible combinations of truth values for $p$ and $q$, the expressions $\neg(p\lor q)$ and $\neg p\land \neg q)$ have the same resulting overall truth value. Thus they are logically equivalent.

\problem{}
Use De Morgan's laws to write negations for the following statements:
\begin{enumerate}
\item Sam is an orange belt and Dave is a red belt.
\item The train is late or my watch is fast.
\end{enumerate}
\solution
\part
Let $p=$"Sam is and orange belt" and $q=$"Dave is a red belt." Then the sentence can be written in the form $p\land q$. De Morgan's laws tell us that $\neg(p\land q)$ is equivalent to $\neg p\lor \neg q$, or in english: "Sam is not an orange belt or Dave is not a red belt."

\part
Let $p=$"The train is late" and $q=$"My watch is fast." Then the sentence can be written in the form $p\lor q$. De Morgan's laws tell us that $\neg(p\lor q)$ is equivalent to $\neg p\land \neg q$, or in english: "The train is not late and my watch is not fast."

\problem{}
Determine if the following statement forms are logically equivalent: (Use any method that you would like)
$$p\implies (q\implies r)\text{ and } (p\implies q)\implies r$$
\solution
\begin{displaymath}
\begin{array}{|c|c|c|c|c|c|c}
   p
 & q
 & r
 & p\Rightarrow{}q
 & q\Rightarrow{}r
 & p\Rightarrow{}(q\Rightarrow{}r)
 & (p\Rightarrow{}q)\Rightarrow{}r \\
\hline
F & F & F & T & T & T & F \\
F & F & T & T & T & T & T \\
F & T & F & T & F & T & F \\
F & T & T & T & T & T & T \\
\hline
T & F & F & F & T & T & T \\
T & F & T & F & T & T & T \\
T & T & F & T & F & F & F \\
T & T & T & T & T & T & T \\
\hline
\end{array}
\end{displaymath}\\
From the above table we can see that the overall truth values of the expressions $p\implies (q\implies r)\text{ and } (p\implies q)\implies r$ are not the same for all possible combinations of truth values for $p,q$, and $r$. In particular they differe for the cases when $p,q,$ and $r$ are all false and when $p$ and $r$ are false and $q$ is true. Thus the two expressions are not logically equivalent.

\end{document}
		