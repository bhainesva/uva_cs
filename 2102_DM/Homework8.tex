\documentclass[paper=a4, fontsize=11pt]{jhwhw} % A4 paper and 11pt font size
\usepackage{pgfplots}
\pgfplotsset{compat=1.9}
\usepackage{amsmath, amssymb}
\newcommand\SetSymbol[1][]{\:#1\vert\:}
\providecommand\given{} % to make it exist
\DeclarePairedDelimiterX\Set[1]\{\}{\renewcommand\given{\SetSymbol[\delimsize]}#1}
\usepackage{listings}
\usepackage{color}
\DeclareSymbolFont{extraup}{U}{zavm}{m}{n}
\DeclareMathSymbol{\varheart}{\mathalpha}{extraup}{86}
\DeclareMathSymbol{\vardiamond}{\mathalpha}{extraup}{87}
\definecolor{dkgreen}{rgb}{0,0.6,0}
\definecolor{gray}{rgb}{0.5,0.5,0.5}
\definecolor{mauve}{rgb}{0.58,0,0.82}

\lstset{frame=tb,
  language=Python,
  aboveskip=3mm,
  belowskip=3mm,
  showstringspaces=false,
  columns=flexible,
  basicstyle={\small\ttfamily},
  numbers=none,
  numberstyle=\tiny\color{gray},
  keywordstyle=\color{blue},
  commentstyle=\color{dkgreen},
  stringstyle=\color{mauve},
  breaklines=true,
  breakatwhitespace=true,
  tabsize=3
}
\begin{document}
\title{Discrete Math - Homework Set \#8}
\author{Ben Haines (bmh5wx)}
\problem{\#1}
Use the Euclidean algorithm to calculate the greatest commmon divisor for 24,024 and 7,524. Then use your work to write the gcd as a linear combination of 24,024 and 7,524.
\solution
\begin{align*}
    24024 &= (3)7524 + 1542\\
    7524 &= (5)1452 + 264\\
    1452 &= (5)264 + 132\\
    132 &= (2)132
\end{align*}
So the GCD of 24,024 and 7524 is 132. To write the GCD as a linear combination of the two numbers we first solve for the remainders.
\begin{align*}
    1452 &= 24042(1) + 7524(-3)\\
    264 &= 7524(1) + 1452(-5)\\
    132 &= 1452(1) + 264(-5)
\end{align*}
And then we plug in these values to the equations from the first step.
\begin{align}
    \begin{split}
        132 &= 1452(1) + 264(-5)\\
            &= 1452(1) + (7524 + 1452(-5))(-5)\\
            &= 1452(1) + 7524(-5) + 1452(25)\\
            &= 1452(26) + 7524(-5)\\
            &= (24024 + 7524(-3))(26) + 7524(-5)\\
            &= 24024(26) + 7524(-78) + 7524(-5)\\
            &= 24024(26) + 7524(-83)
    \end{split}
\end{align}

\problem{\#2}
Suppose $R$ is an equivalence relation on $A$, $S$ is an equivalence relation on $B$, and $A$ and $B$ are disjoint. Prove that $R\cup S$ is an equivalence relation on $A\cup B$.
\solution
\begin{itemize}
    \item Reflexive\\
        For any $x\in A\cup B$, either
        \begin{itemize}
            \item Case 1: $x\in A$\\
                $(x, x)\in R\implies (x, x)\in R\cup S$
            \item Case 2: $x\in B$\\
                $(x, x)\in S\implies (x, x)\in R\cup S$
        \end{itemize}
    \item Symmetric\\
        For any $(x, y)\in R\cup S$, either
        \begin{itemize}
            \item Case 1: $(x, y)\in R$\\
                $\implies (y, x)\in R\implies (y, x)\in R\cup S$
            \item Case 2: $(y, x)\in S$\\
                $\implies (y, x)\in S\implies (y, x)\in R\cup S$
        \end{itemize}
    \item Transitive\\
        Take arbitrary $(x, y), (y, z)\in R\cup S$. $y$ cannot be an element of both $A$ and $B$ because they are disjoint so either
        \begin{itemize}
            \item Case 1: $(x, y), (y, z)\in R$\\
                $\implies (x, z)\in R\implies (x, z)\in R\cup S$
            \item Case 2: $(x, y), (y, z)\in S$\\
                $\implies (x, z)\in S\implies (x, z)\in R\cup S$
        \end{itemize}
\end{itemize}
$R\cup S$ is reflexive, symmetric, and transitive, so $R\cup S$ is an equivalence relation.

\problem{\#3}
Suppose that $R$ is a partial order relation on a set $A$ and that $B$ is a subset of $A$. The restriction of $R$ to $B$ is defined as follows:
$$\Set{(x, y)\given x\in B, y\in B,\text{ and } (x, y)\in R}$$
In other words, two elements of $B$ are related by the restriction of $R$ to $B$ if, and only if, they are related by $R$. Prove that the restriction of $R$ to $B$ is a partial order relation on $B$. (In less formal language, this says that a subset of a partially ordered set is partially ordered.)
\solution
Let $S =$ the restriction of $R$ to $B$.
\begin{itemize}
    \item Reflexive\\
        For any $b\in B$, $(b, b)\in R\implies (b, b)\in S$.
    \item Antisymmetric\\
        For any $(x, y)\in S$ such that $x\not=y$:
        \begin{align*}
            \begin{split}
                (x, y)\in S &\implies (x, y)\in R\\
                            &\implies (y, x)\not\in R\\
                            &\implies (y, x)\not\in S
            \end{split}
        \end{align*}
    \item Transitive\\
        For any $(x, y), (y, z)\in S$ it must be the case that $x, y, z\in B$ and:
        \begin{align*}
            \begin{split}
                (x, y), (y, z)\in S  &\implies (x, y), (y, z)\in R\\
                                     &\implies (x, z)\in R\\
                                     &\implies (x, z)\in S
            \end{split}
        \end{align*}
        So $S$ is transitive.
\end{itemize}
So the restriction of $R$ to $B$ is a partial order relation.

\problem{\#4}
Suppose $R$ is a partial order on $A$. Prove that $R^{-1}$ is also a partial order on $A$.
\solution
\begin{itemize}
    \item Reflexive\\
        For all $x\in A$, $(x, x)\in R\implies (x, x)\in R^{-1}$
    \item Antisymmetric\\
        For any $x, y\in A$ such that $x\not=y$
            $$(x, y)\in R \implies (y, x)\in R^{-1}$$
        and
        \begin{align*}
            (x, y)\in R &\implies (y, x)\not\in R\\
                        &\implies (x, y)\not\in R^{-1}
        \end{align*}
    \item Transitive\\
        For all $x, y, z\in A$ such that $(x, y), (y, z)\in R^{-1}$.
        \begin{align*}
            (x, y), (y, z)\in R^{-1} &\implies (y, x), (z, y)\in R\\
                                     &\implies (z, x)\in R\\
                                     &\implies (x, z)\in R^{-1}
        \end{align*}
\end{itemize}
So $R^{-1}$ is a partial order on $A$.

\problem{\#5}
In each case, say whether or not $R$ is a partial order on $A$. If it is explain why and if not explain why not.
\begin{enumerate}
    \item $A = $ the set of all words in English,
        $$R = \Set{(x, y)\in A \times A\given \text{ the word $y$ occurs at least as late in alphabetical order as the word $x$.}}$$
    \item $A$ is the same as above and\\
        $R = \{(x, y)\in A\times A\mid$ The first letter of the word $y$ occurs at least as late in the alphabet as the first letter of the word $x\}$
\end{enumerate}
\solution
\part
The relation is a partial order on $A$. It is reflexive, antisymmetric, and transitive.
\part
The relation is not a partial order on $A$. It is reflexive and transitive but not antisymmetric. For example, ("cat", "crab")$\in R$ and ("crab", "cat)$\in R$ and "cat"$\not=$"crab".

\problem{\#6}
For any sets $A, B, C,$ and $D$, if $A\times B\subseteq C\times D$ then $A\subseteq C$ and $B\subseteq D$. Is the following proof correct? If so, what proof strategies does it use? If not, can it be fixed? Is the theorem correct?\\

\textbf{Proof.} Suppose $A\times B\subseteq C\times D$. Let $a$ be an arbitrary element of $A$ and let $b$ be an arbitrary element of $B$. Then $(a, b)\in A\times B$. Since $A\times B\subseteq C\times D$, $(a, b)\in C\times D$. Therefore $a\in C$ and $b\in D$. Since $a$ and $b$ were arbitrary elements of $A$ and $B$ respectively, this shows that $A\subseteq C$ and $B\subseteq D$.
\solution
The proof is not correct. If $B$ and $C$ are empty then $A\times B\subseteq C\times D$ and it is not necessarily the case that $A\subseteq C$. The proof can be made correct by adding the qualification that the sets $A, B, C,$ and $D$ are not empty.

\problem{\#7}
Let $p=7$ and $q=13$ and $e=5$, using RSA Encryption, encrypt the following message and decrypt it to prove you get the same message back. The message is "I$\varheart$MATH", use the Caesar cipher for the letters (A=1, B=2,\ldots, Z=26) and let $\varheart=$27. You can use a calculator but you must show where the numbers come from.
\solution
The multiplicative inverse of 5 mod (6)(13) is 29. Then with I=9, $\varheart$=27, M=13, A=1, T=20, and H=8 and using the encryption function encrypt(T) = $(T^e) \mod pq$ we have:\\
encrypt(I) = 81, encrypt($\varheart$) = 27, encrypt(M) = 13, encrypt(A) = 1, encrypt(T)=76, and encrypt(H)=8.

Then using the decrpytion function decrypt(C) = $C^D \mod PQ$ we have:\\
decrypt(81) = I, encrypt(27) = $\varheart$, encrypt(13) = M, encrypt(1) = A, encrypt(76)=T, and encrypt(8)=H, the original message.

\end{document}
