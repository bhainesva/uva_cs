\documentclass[paper=a4, fontsize=11pt]{jhwhw} % A4 paper and 11pt font size
\usepackage{amsmath,amsfonts,amsthm, amssymb} % Math packages
\setlength\parindent{0pt} % Removes all indentation from paragraphs - comment this line for an assignment with lots of text
\usepackage{graphicx}
\DeclareMathOperator{\lcm}{lcm}

\begin{document}
\title{Discrete Math - Homework Set \#3}
\author{Ben Haines (bmh5wx)}
\problem{}
Rewrite each statement without using quantifiers or variables. Indicate which are true and which are false, and justify your answer as best as you can.
\begin{enumerate}
\item $\exists x$ such that Prime($x$)$\land \neg$Odd($x$).
\item $\forall x$, Prime($x$) $\implies \neg$Square($x$).
\item $\exists x$ such that Odd$(x) \land$ Square$(x)$.
\end{enumerate}

\solution
\part
"There exists an integer that is both prime and not odd." This statement is true, 2 is an integer that is both prime and not odd.
\part "For every integer, if that integer is a prime number, then it is not a perfect square." This statement is true. If a number is a perfect square then it has factors other than one and itself and is therefore not prime. 
\part "There exists an integer that is both add and a perfect square." This statement is true. The number 9 is an odd integer and it is the square of 3.


\problem{}
Write a formal negation for each of the following statements.
\begin{enumerate}
\item $\forall$ fish $x$, $x$ has gills.
\item $\forall$ computers $c$, $c$ has a CPU.
\item $\exists$ a movie $m$ such that $m$ is over 6 hours long.
\item $\exists$ a band $b$ such that $b$ has won at least 10 Grammy awards.
\end{enumerate}

\solution
\part
$\exists$ a fish $x$ such that $x$ does not have gills.
\part
$\exists$ a computer $c$ such that $c$ does not have a CPU.
\part
$\forall$ movies $m$, $m$ is less than 6 hours long.
\part
$\forall$ bands $b$, $b$ has won fewer than 10 Grammy awards.


\problem{}
Let $S$ be the set of students at UVA, let $M$ be the set of movies that have ever been released, and let $V(s,m)$ be "student $s$ has seen movie $m$." Rewrite each of the following statements without using the symbol $\forall$, and $\exists$, or variables.
\begin{enumerate}
\item $\exists s \in S$ such that $V(s, \text{Casablanca})$.
\item $\forall s \in S$, $V(s, \text{Star Wars})$.
\item $\forall s \in S$, $\exists m \in M$ such that $V(s, m)$.
\item $\exists m \in M$ such that $\forall s \in S$, $V(s, m)$.
\item $\exists s \in S, \exists t \in S$ and $\exists m \in M$ such that $s\not= t$ and $V(s, m)\land V(t, m)$.
\item $\exists s \in S, \exists t \in S$ and $\forall m \in M$ such that $s\not= t$ and $V(s, m)\implies V(t, m)$.
\end{enumerate}

\solution
\part
There is at least one student at UVA that has seen Casablanca.
\part
Every student at UVA has seen Star Wars.
\part
Every student at UVA has seen at least one movie.
\part
There exists at least one movie that every student at UVA has seen.
\part
There exists at least one movie that at least two different students at UVA have seen.
\part
There exists two students at UVA such that every movie that has been seen by the first student has also been seen by the second.

\problem{}
Prove the following statement. $\exists x(A(x)\implies B(x)) \equiv \forall xA(x)\implies \exists xB(x)$.

\solution
\begin{align}
\begin{split}
\exists x(A(x)\implies B(x)) &\equiv \exists x (\neg A(x) \lor B(x))\\
&\equiv \exists x \neg A(x) \lor \exists x B(x)\\
&\equiv \neg \forall x A(x) \lor \exists x B(x)\\
&\equiv \forall x A(x) \implies \exists x B(x)
\end{split}
\end{align}


\problem{}
Prove that for any positive integers $a$ and $b$ that the following is true. $ab = \gcd{(a, b)}\cdot \lcm{(a, b})$. 
\solution
Let $a$ and $b$ be some positive integers with the unique prime factorizations $a=p_1^{a_1}p_2^{a_2}\cdots p_n^{a_n}$ and $b=p_1^{b_1}p_2^{b_2}\cdots p_n^{b_n}$.  

Then
\begin{align}
a\cdot b = p_1^{a_1}p_2^{a_2}\cdots p_n^{a_n}p_1^{b_1}p_2^{b_2}\cdots p_n^{b_n} = p_1^{a_1+ b_1}p_2^{a_2+ b_2}\cdots p_n^{a_n+ b_n}
\end{align}
We know that 
\begin{align}
\gcd{(a, b)} = p_1^{\min{(a_1,b_1)}}p_2^{\min{(a_2, b_2)}}\cdots p_n^{\min{(a_n, b_n)}}
\end{align}
\centerline{and}
\begin{align}
\lcm{(a, b)} = p_1^{\max{(a_1,b_1)}}p_2^{\max{(a_2, b_2)}}\cdots p_n^{\max{(a_n, b_n)}}
\end{align}
because there are only two choices and we are multipliying the min and the max the product \begin{align}
\lcm{(a, b)}\cdot \gcd{(a, b)} &= p_1^{a_1+ b_1}p_2^{a_2+ b_2}\cdots p_n^{a_n+ b_n}\\
&= a\cdot b
\end{align}



\problem{}
Suppose that the domain of the propositional function $P(x)$ consists of the integers -2, -1, 0, 1, and 2. Write out each of these propositions using disjunctions, conjunctions, and negations.
\begin{enumerate}
\item $\exists xP(X)$
\item $\forall xP(X)$
\item $\exists x\neg P(X)$
\item $\forall x\neg P(X)$
\item $\neg \exists xP(X)$
\item $\neg \forall xP(X)$
\end{enumerate}
\solution
\part $P(-2) \lor P(-1) \lor P(0) \lor P(1) \lor P(2)$
\part $P(-2) \land P(-1) \land P(0) \land P(1) \land P(2)$
\part $\neg P(-2) \lor \neg P(-1) \lor \neg P(0) \lor \neg P(1) \lor \neg P(2)$
\part $\neg P(-2) \land \neg P(-1) \land \neg P(0) \land \neg P(1) \land \neg P(2)$
\part $\neg P(-2) \land \neg P(-1) \land \neg P(0) \land \neg P(1) \land \neg P(2)$
\part $\neg P(-2) \lor \neg P(-1) \lor \neg P(0) \lor \neg P(1) \lor \neg P(2)$

\problem{}
The computer scientists Richard Conway and David Gries once wrote: "The absence of error messages during translation of a computer program is only a necessary and not a sufficient condition for reasonable [program] correctness." Rewrite this statement without using the words necessary or sufficient.

\solution
Reasonable program correctness implies the absence of error messages during translation of a computer program but the absence of error messages during translation of a computer program does not imply reasonable program correctness.

\problem{}
Let $R(m,n)$ be the predicate "If $m$ is a factor of $n^2$ then $m$ is a factor of $n$," with domain for both $m$ and $n$ being the set of integers.
\begin{enumerate}
\item Explain why $R(m,n)$ is false if $m=25$ and $n=10$.
\item Give values different from those in the previous part for which $R(m, n)$ is false.
\item Explain why $R(m,n)$ is true if $m=5$ and $n=10$.
\end{enumerate}
\solution
\part
Substituting the values 25 and 10 into the predicate we obtain the proposition that "If $25$ is a factor of $100$ then $25$ is a factor of $10$." This is clearly false because 25 is a factor of 100 but it is not a factor of 10.
\part
The statement is also false when $m=100$ and $n=20$. $m$ is a factor of 400 but it is not a factor of 20.
\part
Substituting the values 5 and 10 into the predicate we obtain the proposition that "If $5$ is a factor of $100$ then $5$ is a factor of $10$." This is clearly true, 5 is a factor of 100 and it is also a factor of 10.

\problem{}
Let $P(x), Q(X),$ and $R(x)$ be the statements "x is a clear explanation," "x is satisfactory," and "x is an excuse," respectively. Suppose that the domain for x consists of all English text. Express each of the following statments using quantifiers, logical connectives, and $P(x), Q(x)$, and $R(x)$.
\begin{enumerate}
\item All clear explanation are satisfactory
\item Some excuses are unsatisfactory
\item Some excuses are not clear explanations
\end{enumerate}
\solution
\part 
$\forall x\in X s.t. (P(x)\implies Q(x))$
\part 
$\exists x\in X s.t. (R(x)\land \neg Q(x))$
\part 
$\exists x\in X s.t. (R(x)\land \neg P(x))$

\end{document}
		