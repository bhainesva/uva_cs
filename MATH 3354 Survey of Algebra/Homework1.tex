\documentclass[paper=a4, fontsize=11pt]{jhwhw} % A4 paper and 11pt font size
\usepackage{amsmath,amsfonts,amsthm, amssymb} % Math packages
\setlength\parindent{0pt} % Removes all indentation from paragraphs - comment this line for an assignment with lots of text
\usepackage{graphicx}
\usepackage{mathtools}
\newcommand\SetSymbol[1][]{\:#1\vert\:}
\providecommand\given{} % to make it exist
\DeclarePairedDelimiterX\Set[1]\{\}{\renewcommand\given{\SetSymbol[\delimsize]}#1}

\begin{document}
\title{Survey of Algebra - Homework Set \#1}
\author{Ben Haines}
%SECTION 1.1
\section*{Section 1.1}
\problem{\#38}
Prove or disprove that $\mathcal P (A\cup B) = \mathcal P (A) \cup \mathcal P (B)$.
\solution
\begin{align}
\begin{split}
\mathcal P \left({A \cup B}\right)
&= \Set{X \given X \subseteq (A \cup B)}\\
&=\Set{X\given X \subseteq A} \cup \Set{X\given X \subseteq B}\\
&=\mathcal P \left({A}\right) \cup \mathcal P \left({B}\right)
\end{split}                   
\end{align}

\problem{\#40}
Prove or disprove that $\mathcal P (A-B) = \mathcal P (A) - \mathcal P (B)$.
\solution
\begin{align}
\begin{split}
\mathcal P \left({A - B}\right)
&= \Set{X\given X \subseteq (A - B)}\\
&= \Set{X\given X \subseteq \Set{x \in U \given x \in A \text{ and } x \not\in B}}\\
&= \Set{X\given X \subseteq (\Set{x \in U \given x \in A} \cap \Set{x \in U \given x \not\in B})}\\
&= \Set{X\given X \subseteq (A \cap B')}\\
&= \Set{X\given X \in (\mathcal P \left({A}\right) \cap \mathcal P \left({B'}\right)) }\\
&= \Set{X \given X \in \mathcal P \left({A}\right) \text{ and } X \not\in \mathcal P \left({B}\right) }\\
&= \Set{X \given X \in \Set{Y\given Y \subseteq A} \text{ and } X \not\in \Set{Z\given Z \subseteq B }}\\
&= \mathcal P(A) - \mathcal P(B)
\end{split}
\end{align}

%SECTION 1.2
\newpage
\section*{Section 1.2}
\problem{\#14}
Let $f: Z \to \left\{ {-1, 1} \right\}$ be given by
\[  
f(x) =
     \begin{cases}
       \text{1} &\quad\text{if x is even}\\
       \text{-1} &\quad\text{if x is odd}\
     \end{cases}
\]
\begin{enumerate}
\item Prove or disprove that $f$ is onto.
\item Prove or disprove that $f$ is one-to-one.
\item Prove or disprove that $f(x_{1} + x_{2}) = f(x_{1})f(x_{2})$.
\item Prove or disprove that $f(x_{1}x_{2}) = f(x_{1})f(x_{2})$.
\end{enumerate}
\solution
\part
\begin{align}
1, 2 &\in \mathbb{Z}\\
f(1) &= -1\\
f(2) &= 1
\end{align}
This demonstrates that for all $x\in \Set{-1, 1}$ there exists a $y\in \mathbb{Z}$ such that $f(y) = x$ and f is onto by definition.
\part
\begin{align}
2,4 \in \mathbb{Z}\\
f(2) = 1\\
f(4) = 1
\end{align}
$f$ is not one-to-one because for two values $x,y\in \mathbb{Z}$ where $x \not= y$, $f(x) = f(y)$.
\part
Case 1: $x_{1}$ and $x_{2}$ are even.\\
The sum of any two even numbers is also an even number. 
\begin{align}
f(x_1 + x_2) &= 1\\
f(x_1)f(x_2) &= (1)(1) = 1
\end{align}
in this case $f(x_1 + x_2) = f(x_1)f(x_2)$.\\

Case 2: $x_{1}$ and $x_{2}$ are odd.\\
The sum of any two odd numbers is also an odd number. 
\begin{align}
f(x_1 + x_2) &= 1\\
f(x_1)f(x_2) &= (-1)(-1) = 1
\end{align}
in this case $f(x_1 + x_2) = f(x_1)f(x_2)$.\\

Case 3: One of $x_{1}$ and $x_{2}$ is odd and the other is even.\\
The sum of any odd numbers and any even number is an odd number.\\
\begin{align}
f(x_1 + x_2) &= -1\\
f(x_1)f(x_2) &= (1)(-1) = -1
\end{align}
in this case $f(x_1 + x_2) = f(x_1)f(x_2)$.\\
In all cases $f(x_1 + x_2) = f(x_1)f(x_2)$

\part
For $x_1 = 1, x_2 = 1$
\begin{align}
f(x_1 + x_2) &= 1\\
f(x_1) + f(x_2) &= -1 + (-1) = -2
\end{align}
$f(x_1 + x_2) \not= f(x_1) + f(x_2)$\\

In Exercises 20-22 , suppose m and n are positive integers, A is a set with exactly m elements, and B is a set with exactly n elements.
\problem{\#20}
How many mappings are there from $A$ to $B$?
\solution
For each of the $n$ positions in $B$ there are $m$ possible values. The total number of mappings is $m^{n}$.

\problem{\#22}
If $m \le n$, how many one-to-one mappings are there from $A$ to $B$?
\solution
\begin{align}
\frac{n!}{(n-m)!}
\end{align}

%SECTION 1.3
\newpage
\section*{Section 1.3}
\problem{\#10}
Let $g:A \to B$ and $f:B\to C$. Prove that $f$ is onto if $f\circ g$ is onto.
\solution
By definition $(f\circ g)$ being onto means that $\forall c \in C$ there exists an element $a\in A$ such that
\begin{align}
\begin{split}
c &= (f\circ g)(a)\\
&= f(g(a))
\end{split}
\end{align}
Let $b$ be the result of $g(a)$. It is given that $b\in B$.\\
Therefore for any element $c\in C$ there exists an element $b\in B$ such that $f(b) = c$ and $f$ is onto by definition.

\problem{\#11}
Let $g:A\to B$ and $f:B\to C$. Prove that $f$ is one-to-one if $f\circ g$ is one-to-one.
\solution
Assume that $g$ is not one-to-one. This implies that there exist elements $a_1, a_2 \in A$ such that $a_1 \not= a_2$ and $g(a_1) = g(a_2)$. If this is the case then $f(g(a_1)) = f(g(a_2))\text{ and } (f\circ g)(a_1) = (f\circ g)(a_2)$. This contradicts the given that $(f\circ g)$ is one-to-one and thus $g$ must also be one-to-one.

\problem{\#12}
Let $f:A\to B$ and $g:B\to A$. Prove that $f$ is one-to-one and onto if $f\circ g$ is one-to-one and $g\circ f$ is onto.
\solution
This is a specific case of the situation in problems \#10 and \#11 where $A=C$. Thus we can use the same steps from those proofs to show that $g$ is both one-to-one and onto.
\part
\begin{align}
(f\circ g) \text{is one-to-one so by definition if } f(g(b_1)) = f(g(b_2)) \text{ then } b_1 = b_2
\end{align}
Let us assume that $f$ is not one-to-one. This implies that there exist elements $a_1, a_2 \in A$ such that $f(a_1) = f(a_2)$ and $a_1 \not= a_2$.\\

Define $b_1, b_2$ such that $g(b_1) = a_1$ and $g(b_2) = a_2$. We know that $b_1$ and $b_2$ exist because $g$ is onto and because $a_1 \not= a_2$ we know from the definition of a mapping that $b_1 \not= b_2$. This contradicts (19). Therefore $f$ must be one-to-one.
\part
It has been shown that bth $g$ and $f$ are one-to-one. By Theorem 1.17 (pg. 27) the composition $(g\circ f)$ is also one-to-one. We are told that $(g\circ f)$ is onto. The composition is thus bijective. By definition the following is then true
\begin{align}
\forall a_2 \in A \text{ there must exist a unique element } a_1\in A\text{ such that } a_2 = (g\circ f)(a_1)
\end{align}
Let us assume that $f$ is not onto. Therefore
\begin{align}
\text{there exists an element } x\in X \text{ such that for all }y\in A, f(y)\not= x
\end{align}
Let $a = g(x)$ for the $x$ indicated above. By (20) we know that there must exist some value $n\in A$ such that 
\begin{align}
\begin{split}
a &= g(x)\\
&= (g\circ f)(n)\\
&=g(f(n))
\end{split}
\end{align}
by (20) $f(n) = x$. This contradicts (21). The function $f$ must be onto.

%SECTION 1.4
\newpage
\section*{Section 1.4}
\problem{\#11b}
Prove or disprove that the set $B = \Set{z^3 \given z \in \mathbb{Z}}$ is closed with respect to multiplication defined on $\mathbb{Z}$.
\solution
Select arbitrary elements $x, y \in B$. By the definition of $B$, $x=z_1^3$ and $y=z_2^3$ for some $z_1, z_2 \in \mathbb{Z}$. Then
\begin{align}
\begin{split}
xy &= (z_1 \cdot z_1 \cdot z_1)\cdot (z_2\cdot z_2\cdot z_2)\\
&= (z_1\cdot z_2)\cdot (z_1\cdot z_2)\cdot (z_1\cdot z_2)\\
&= (z_1\cdot z_2)^3
\end{split}
\end{align}
Let $z_3 = (z_1\cdot z_2)$. $\mathbb{Z}$ is closed under multiplication and therefore $z_3 \in \mathbb{Z}$. It has been shown that for any elements $x, y\in B$ there exists an element $z_3\in \mathbb{Z}$ such that $x*y = z_3^3$. Therefore for all $x,y\in B$ the product $x*y$ is also in $B$.$B$ is closed under multiplication defined on $\mathbb{Z}$.

\problem{\#12b}
Prove or disprove that the set $\mathbb{Q} - \Set{0}$ of nonzero rational numbers is closed with respect to division defined on the set $\mathbb{R} - \Set{0}$ of nonzero real numbers.
\solution
$\mathbb{Q}$ is defined as $\Set{\frac{m}{n} \given m,n\in \mathbb{Z}\text{ and } n\not=0}$
Choose two arbitrary elements $x,y \in \mathbb{Q}$. Then $x = \frac{m_1}{n_1}, y = \frac{m_2}{n_2}$ for some elements $m_1, m_2, n_1, n_2 \in \mathbb{Z} \not= 0$.
\begin{align}
\frac{x}{y} &=  \frac{m_1\cdot n_2}{n_1\cdot m_2}
\end{align}
$\mathbb{Z}$ is closed under multiplication so the products $m_1\cdot n_2$ and $n_1\cdot m_2$ are both elements of $\mathbb{Z}$ and $n_1\cdot m_2 \not= 0$ because neither $n_1$ nor $n_2$ is equal to 0. Thus $\frac{x}{y}$ is by definition an element of $\mathbb{Q}$. This means that $\mathbb{Q}$ is closed under division defined on $\mathbb{Z}$.
\problem{\#13}
Assume that * is an associative binary operation on the nonempty set $A$. Prove that
\begin{align}
a * [b * (c * d)] = [a * (b * c)] * d
\end{align}
for all $a,b,c,\text{ and }d$ in $A$.
\solution
Let $e=(b*c)$. Using the fact that * is associative on $A$
\begin{align}
\begin{split}
a * [b * (c * d)] &= a * [(b * c) * d]\\
&= a * [e * d]\\
&=[a * e] * d\\
&=[a * (b * c)] * d
\end{split}
\end{align}

%SECTION 1.5
\newpage
\section*{Section 1.5}
\problem{\#4}
Let $f:A\to A$, where $A$ is nonempty. Prove that $f$ has a left inverse if and only if $f^{-1}(f(S)) = S$ for every subset $S$ of $A$.
\solution
\part
Assume that f has a left inverse. By definition every element of $S$ is an element of $A$ and $1_A(a) = a$ for every element of $A$. Then 
\begin{align}
\begin{split}
f^{-1}(f(S)) &= (f^{-1}\circ f )(S)\\
&= 1_A(S)\\
&= S
\end{split}
\end{align}
\part
Assume $f^{-1}(f(S)) = S$ for every subset $S$ of $A$. $A$ is a possible subset of $A$. Therefore for all elements $a\in A$
\begin{align}
\begin{split}
a &= f^{-1}(f(a))\\
&= (f^{-1}\circ f)(a)
\end{split}
\end{align}
and $f^{-1}$ must be the left inverse of $f$.

\problem{\#6}
Prove that if $f$ is a permutation on $A$, then $f^{-1}$ is a permutation on $A$.
\solution
By Theorem 1.26 (pg. 42) we know that, because $f$ is a permutation, it must be invertible. So there exists a single function $g:A\to A$ such that
\begin{align}
(g\circ f) = 1_A = (f\circ g)
\end{align}
The function $g$ has both a left and right inverse and is thus invertible. By Theorem 1.26 we know that $g$ must be a permutation.

\problem{\#7}
Prove that if $f$ is a permutation on $A$, then $(f^{-1})^{^{-1}} = f$.
\solution
For a permutation $f, f^{-1}$ is defined such that
\begin{align}
(f^{-1}\circ f) = 1_A = (f\circ f^{-1})
\end{align}
It can be seen from the above equation that $f$ is both the left and right inverse of $f^{-1}$ and thus $(f^{-1})^{^{-1}} = f$.



\end{document}