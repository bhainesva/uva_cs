\documentclass[paper=a4, fontsize=11pt]{jhwhw} % A4 paper and 11pt font size
\usepackage{amsmath,amsfonts,amsthm, amssymb} % Math packages
\setlength\parindent{0pt} % Removes all indentation from paragraphs - comment this line for an assignment with lots of text
\usepackage{graphicx}
\usepackage{verbatim}
\usepackage{mathtools}
\usepackage{color}
\newcommand\SetSymbol[1][]{\:#1\vert\:}
\providecommand\given{} % to make it exist
\DeclarePairedDelimiterX\Set[1]\{\}{\renewcommand\given{\SetSymbol[\delimsize]}#1}

\begin{document}
\title{Survey of Algebra - Assignment \#6}
\author{Ben Haines}

%SECTION 3.5
\section*{Section 3.5}

\problem{\# 7}
Find an isomorphism $\phi$ from the additive group $\mathbb Z$ to the multiplicative group
$$H = \left\{\left[\begin{array}{cc}
1 & n  \\
0 & 1   \end{array}\right] \middle| n\in \mathbb{Z} \right\}$$
\solution
Let $\phi: \mathbb Z \to H$ be defined as 
$$\phi(x) = 
\left[\begin{array}{cc}
1 & x  \\
0 & 1   \end{array}\right]
$$
\part
It is clear that $\phi$ is both one to one and onto.
\part
\begin{align}
\begin{split}
\phi(n)\phi(m) &= 
\left[\begin{array}{cc}
1 & n  \\
0 & 1   \end{array}\right]
\left[\begin{array}{cc}
1 & m  \\
0 & 1   \end{array}\right]\\
&=
\left[\begin{array}{cc}
1 & n+m  \\
0 & 1   \end{array}\right]\\
&= 
\left[\begin{array}{cc}
1 & m+n  \\
0 & 1   \end{array}\right]\\
&= \phi(n+m)
\end{split}
\end{align}
So $\phi$ preserves the operation. Therefore $\phi$ is an isomorphism from $\mathbb Z$ to $H$.

\problem{\# 30}
For an arbitrary positive integer $n$, prove that any two cyclic groups of order $n$ are isomorphic.
\solution
Let $A = \langle a \rangle$ be some cyclic group of order $n$ and $\phi: \mathbb Z_n \to A$ defined by $\phi([x]) = a^x$. We know that $a^x = a^{[x]}$ because although there are many members of the equivalence class, they are all congruent to $x \mod n$. This is clearly a bijection because $A$ is cyclic. Then for any $[x], [y] \in \mathbb Z_n$
\begin{align}
\begin{split}
\phi([x])\phi([y]) &= a^{[x]}a^{[y]}\\
&= a^{[x]+[y]}\\
&= \phi([x] + [y])
\end{split}
\end{align}
So $\phi$ preserves the operation and $A$ and $\mathbb Z_n$ are isomorphic. So any cyclic group of order $n$ is isomorphic to $\mathbb Z_n$ and thus by the transitive property, any two cyclic groups of the same order are isomorphic to each other.

\problem{\# 31}
Prove that any infinite cyclic group is isomorphic to $\mathbb Z$ under addition.
\solution
Let $G$ be the an infinite cyclic group defined by $\langle a \rangle$ and let $\phi: \mathbb Z \to G$ be defined by $\phi(x) = a^x$. $\phi$ is clearly both one to one and onto. Then
\begin{align}
\begin{split}
\phi(x)\phi(y) &= a^xa^y\\
&= a^{x+y}\\
&= \phi(x+y)
\end{split}
\end{align}
So $\phi$ preserves the operation and $G$ is isomorphic to $\mathbb Z$ under addition. 

\problem{\# 32}
Let $H$ be the group $\mathbb Z_6$ under addition. Find all isomorphisms from the multiplicative group $\mathbb U_7$ of units in $\mathbb Z_7$ to $H$.
\solution
For a a mapping to be an isomorphism it must map the identity element of the first group to the identity element of the second. Therefore we know that in all isomorphisms $\phi: \mathbb U_7 \to H$,  $\phi([1]) = [0]$. We also know that inverses must map to inverses. The element [6] is the only element in $\mathbb U_7$ that is its own inverse so it must map to $[3]$, the only element in $H$ that is its own inverse. We also know that the generators of $\mathbb U_7$ must map to the generators of $H$. Therefore $\phi([5])$ and $\phi([3])$ must map to either [1] or [5]. The remaining elements have no constraints. Therefore all isomorphisms can be listed as:
\begin{flalign*}
&\phi([1])=[0]&\phi([1])=[0]&&\phi([1])=[0]&&\phi([1])=[0]\\
&\phi([2])=[2]&\phi([2])=[4]&&\phi([2])=[2]&&\phi([2])=[4]\\
&\phi([3])=[1]&\phi([3])=[1]&&\phi([3])=[5]&&\phi([3])=[5]\\
&\phi([4])=[4]&\phi([4])=[2]&&\phi([4])=[4]&&\phi([4])=[2]\\
&\phi([5])=[5]&\phi([5])=[5]&&\phi([5])=[1]&&\phi([5])=[1]\\
&\phi([6])=[3]&\phi([6])=[3]&&\phi([6])=[3]&&\phi([6])=[3]\\
\end{flalign*}

%SECTION 3.6
\newpage
\section*{Section 3.6}

\problem{\# 16}
Suppose that $G$, $G'$, and $G''$ are groups. If $G'$ is a homomorphic image of $G$, and $G''$ is a homomorphic image of $G'$, prove that $G''$ is a homomorphic image of $G$. (Thus the relation in Exercise 15 has the transitive property.
\solution
According to the given information there exist epimorphisms $\phi_1: G\to G'$ and $\phi_2: G'\to G''$. Let $\phi_3 = \phi_2\circ\phi_1$.
\part
By Theorem 1.16 regarding the composition of onto mappings we know that $\phi_2 \circ \phi_1: G\to G''$ is onto. 
\part
For $x, y \in G$,
\begin{align}
\begin{split}
\phi_3(x)\phi_3(y) &= \phi_2(\phi_1(x))\phi_2(\phi_1(y))\\
&=\phi_2(\phi_1(x)\phi_1(y))\\
&=\phi_2(\phi_1(xy))\\
&= \phi_3(xy)
\end{split}
\end{align}
Therefore the mapping $\phi_3: G\to G''$ is an epimorphism because it is both onto and preserves operation. Thus $G''$ is a homomorphic image of $G$.

\problem{\# 18}
Suppose that $\phi$ is an epimorphism from the group $G$ to a group $G'$. Prove that $\phi$ is an isomorphism if and only if $\ker \phi = \Set{e}$, where $e$ denotes the identity in $G$.
\solution
\part
Let $\phi$ be an isomorphism. We know that $e\in \ker \phi$. Assume that there is another element $x$ of $G$ that is also in $\ker \phi$. This means that for $x\not= e, \phi(x)=\phi(e)=e$. This means that $\phi$ is not one-to-one and is therefore not an isomorphism. Therefore, if $\phi$ is an isomorphism, $\ker \phi = \Set{e}$.
\part
Let $\ker \phi = \Set{e}$. Assume there are two elements in $G$ such that $\phi(x) = \phi(y)$. We know that either $x=y=e$ or that $\phi(x)$ and $\phi(y)$ are not equal to $e$ or $x$ and $y$ would be members of $\ker \phi$.
\begin{align}
\begin{split}
\phi(x^{-1})\phi(y) &= \phi(x^{-1}y)\\
\phi(x)^{-1}\phi(y) &= \phi(x^{-1}y)\\
\phi(y)^{-1}\phi(y) &= \phi(x^{-1}y)\\
e &= \phi(x^{-1}y)
\end{split}
\end{align}
We know that $e$ is the only member of $\ker \phi$ so $x^{-1}y = e$. Then $x^{-1}$ is the inverse of both $x$ and $y$. Inverses are unique so $x=y$. Therefore $\phi$ is an isomorphism.


\problem{\# 19}
Let $\phi$ be a homomorphism from a group $G$ to a group $G'$. Prove that $\ker \phi$ is a subgroup of $G$. 
\solution
\part
$e \in \ker \phi$ so $\ker \phi$ is not empty and contains the identity element.
\part
For two elements $x, y \in \ker \phi$
\begin{align}
\begin{split}
\phi(x)\phi(y) &= \phi(xy)\\
e\cdot e &= \phi(xy)\\
e &= \phi(xy)
\end{split}
\end{align}
Therefore $xy\in \ker\phi$ and $\ker \phi$ is closed.
\part
For any $x \in \ker \phi$, $\phi(x^{-1}) = \phi(x)^{-1}$.
\begin{align}
\begin{split}
e &= \phi(x)^{-1}\\
 &= \phi(x^{-1})
 \end{split}
 \end{align}
 Therefore $x^{-1} \in \ker\phi$ and $\ker \phi$ contains inverses. Having satisfied the necessary conditions, $\ker\phi$ is a subgroup of $G$.
\problem{\# 20}
If $G$ is an abelian group and the group $G'$ is a homomorphic image of $G$, prove that $G'$ is abelian.
\solution
$G'$ is a homomorphic image of $G$ so any elements $a, b$ in $G'$ can be represented as $\phi(x), \phi(y)$ for some $x, y\in G$. Therefore
\begin{align}
\begin{split}
ab &= \phi(x)\phi(y)\\
&= \phi(xy)\\
&= \phi(yx)\\
&= \phi(y)\phi(x)\\
&= ba
\end{split}
\end{align}
Therefore $G'$ is abelian.

\problem{\# 23}
Assume that $\phi$ is a homomorphism from the group $G$ to the group $G'$. 
\begin{itemize}
\item Prove that if $H$ is any subgroup of $G$, then $\phi(H)$ is a subgroup of $G'$.
\item Prove that if $K$ is any subgroup of $G'$, then $\phi^{-1}(K)$ is a subgroup of $G$.
\end{itemize}

\solution
\begin{itemize}
\item
\begin{enumerate}
\item
$e \in H$ and $\phi(e) = e$ so $\phi(H)$ is nonempty and contains the identity element.
\item
Any $a,b\in \phi(H)$ can be written as $\phi(x),\phi(y)$ for $x,y \in H$. So
\begin{align}
\begin{split}
ab &= \phi(x)\phi(y)\\
&= \phi(xy)
\end{split}
\end{align}
$H$ is closed so $xy$ is in $H$. Therefore $ab$ is in $\phi(H)$ and $\phi(H)$ is closed.
\item
Any $a\in \phi(H)$ can be written as $\phi(x)$ for some $x\in H$. Then
\begin{align}
\begin{split}
a^{-1} &= \phi(x)^{-1}\\
&= \phi(x^{-1})
\end{split}
\end{align}
$H$ is a group so $x^{-1}$ must be an element of $H$ and thus $a^{-1}$ is an element of $\phi(H)$. So $\phi(H)$ contains inverses. 
Having satisfied all the necessary conditions, $\phi(H)$ is a subgroup of $G'$.
\end{enumerate}
\end{itemize}
\begin{itemize}
\item
\begin{enumerate}
\item
$e \in K$ and $\phi^{-1}(e) = e$ so $\phi^{-1}(K)$ is nonempty and contains the identity element.
\item
Any $a,b\in \phi^{-1}(K)$ can be written as $\phi(x),\phi(y)$ for $x,y \in K$. So
\begin{align}
\begin{split}
ab &= \phi^{-1}(x)\phi^{-1}(y)\\
&= \phi^{-1}(xy)
\end{split}
\end{align}
$K$ is closed so $xy$ is in $K$. Therefore $ab$ is in $\phi^{-1}(K)$ and $\phi^{-1}(K)$ is closed.
\item
Any $a\in \phi^{-1}(K)$ can be written as $\phi^{-1}(x)$ for some $x\in K$. Then
\begin{align}
\begin{split}
a^{-1} &= \phi^{-1}(x)^{-1}\\
&= \phi^{-1}(x^{-1})
\end{split} 
\end{align}
$K$ is a group so $x^{-1}$ must be an element of $K$ and thus $a^{-1}$ is an element of $\phi^{-1}(K)$. So $\phi(K)$ contains inverses. 
Having satisfied all the necessary conditions, $\phi^{-1}(K)$ is a subgroup of $G$.
\end{enumerate}
\end{itemize}
\end{document}