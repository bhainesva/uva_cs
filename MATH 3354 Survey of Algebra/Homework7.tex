\documentclass[paper=a4, fontsize=11pt]{jhwhw} % A4 paper and 11pt font size
\usepackage{amsmath,amsfonts,amsthm, amssymb} % Math packages
\setlength\parindent{0pt} % Removes all indentation from paragraphs - comment this line for an assignment with lots of text
\usepackage{graphicx}
\newcommand\SetSymbol[1][]{\:#1\vert\:}
\providecommand\given{} % to make it exist
\DeclarePairedDelimiterX\Set[1]\{\}{\renewcommand\given{\SetSymbol[\delimsize]}#1}
\DeclareMathOperator{\lcm}{lcm}

\begin{document}
\title{Survey of Algebra - Assignment \#7}
\author{Ben Haines (bmh5wx)}
%SECTION 4.1
\section*{Section 4.1}
\problem{\#10}
Let $f=(1, 2, 3)(4, 5)$. Compute each of the following powers of $f$.
\begin{enumerate}
        \item $f^{-1}$
        \item $f^{18}$
        \item $f^{23}$
        \item $f^{201}$
\end{enumerate}
\solution
\part
$f^{-1} = (5, 4)(3, 2, 1)$
\part
$f^{18} =(1, 2, 3)(4, 5)$
\part
$f^{23} = (5, 4)(3, 2, 1)$
\part
$f^{201} = (1, 2, 3)(5, 4)$


\problem{\#12}
Compute $gfg^{-1}$, the conjugate of $f$ by $g$, for each pair $f, g$.
\begin{enumerate}
        \item $f=(1, 2, 4, 3)$; \hspace{1em} $g=(1, 3, 2)$
        \item $f=(1, 3, 5, 6)$; \hspace{1em} $g=(2, 5, 4, 6)$
        \item $f=(2, 3, 5, 4)$; \hspace{1em} $g=(1, 3, 2)(4, 5)$
        \item $f=(1, 4)(2, 3)$; \hspace{1em} $g=(1, 2, 3)$
        \item $f=(1, 3, 5)(2, 4)$; \hspace{1em} $g=(2, 5)(3, 4)$
        \item $f=(1, 3, 5, 2)(4, 6)$; \hspace{1em} $g=(1, 3, 6)(2, 4, 5)$
\end{enumerate}
\solution
\part (3, 1, 4, 2)
\part (1, 3, 4, 2)
\part (1, 2, 4, 5)
\part (2, 4)(3, 1)
\part (1, 4, 2)(5, 3)
\part (3, 6, 2, 4)(5, 1)

\problem{\#14}
Write the permutation $f=(1, 2, 3, 4, 5, 6)$ as a product of a permutation $g$ of order 2 and a permutation $h$ of order 3.
\solution
\begin{align}
	&g = (1, 5, 3)(2, 6, 4) && h= (6, 3)(5, 2)(4, 1)\\
	&f = g \circ h
\end{align}

\problem{\#15}
Write the permutation $f=(1, 2, 3, 4, 5, 6, 7, 8, 9, 10, 11, 12)$ as a product of a permutation $g$ of order 3 and $h$ of order 4.
\solution
\begin{align}
	&g = (1, 5, 9)(2, 6, 10)(3, 7, 11)(4, 8, 12) && h= (1, 10, 7, 4)(2, 11, 8, 5)(3, 12, 9, 6)\\
	&f = g \circ h
\end{align}

\problem{\#16}
List all the elements of the alternating group $A_3$, written in cyclic notation.
\solution
$A_3 = \Set{(1), (2), (3), (1, 2, 3)}$

\problem{\#18}
Find all the distinct cyclic subgroups of $A_4$.
\solution
$(1, 2)(3, 4), (1, 3)(2, 4), (1, 4)(2, 3), (1, 2, 3), (1, 3, 2), (1, 2, 4), (1, 4, 2), (1, 3, 4), (1, 4, 3), (2, 3, 4), (2, 4, 3)$

\problem{\#19}
Find cyclic subgroups of $S_4$ that have three different orders.
\solution
\begin{itemize}
        \item $(1, 2)$ has order 2.
        \item $(1, 2, 3)$ has order 3.
        \item $(1, 2, 3, 4)$ has order 4.
\end{itemize}

\problem{\#20}
Construct a multiplication table for the octic group $D_4$ in Example 12 of this section.
\solution
\begin{table}[h]
\begin{tabular}{l|l|l|l|l|l|l|l|l|}
$\cdot$    & e   & $\alpha^{2}$  & r180 & $\alpha^{3}$ & $\delta$    & $\beta$    & $\lambda$   & $\theta$   \\ \hline
e   & e   & $\alpha^{2}$  & r180 & $\alpha^{3}$ & $\delta$    & $\beta$    & $\lambda$   & $\theta$   \\ \hline
$\alpha^{2}$  & $\alpha^{2}$  & r180 & ab   & b    & $\theta$   & $\lambda$   & $\delta$    &      \\ \hline
r180 & r180 & $\alpha^{3}$ & e   & b    & $\beta$    & $\delta$    & $\theta$   & $\lambda$   \\ \hline
$\alpha^{3}$ & $\alpha^{3}$ & e   & a    & r180 & $\lambda$   & $\theta$   & $\beta$    & $\delta$    \\ \hline
$\delta$    & $\delta$    & $\lambda$   & $\beta$    & $\theta$   & e   & r180 & $\alpha^{2}$  & $\alpha^{3}$ \\ \hline
$\beta$    & $\beta$    & $\theta$   & $\delta$    & $\lambda$   & r180 & e   & $\alpha^{3}$ & $\alpha^{2}$  \\ \hline
$\lambda$   & $\lambda$   & $\beta$    & $\theta$   & $\delta$    & $\alpha^{3}$ & $\alpha^{2}$  & e   & r180 \\ \hline
$\theta$   & $\theta$   & $\delta$    & $\lambda$   & $\beta$    & $\alpha^{2}$  & $\alpha^{3}$ & r180 & e   \\ \hline
\end{tabular}
\end{table}


\problem{\#30}
Let $\phi$ be the mapping from $S_n$ to the additive group $\mathbb Z_2$ defined by
$$\phi(f) =
  \begin{cases} 
          \hfill [0]    \hfill & \text{ if $f$ is an even permutation} \\
          \hfill [1] \hfill & \text{ if $f$ is an odd permutation} \\
  \end{cases}
$$
\begin{enumerate}
        \item Prove that $\phi$ is an isomorphism.
        \item Find the kernel of $\phi$.
        \item Prove or disprove that $\phi$ is an epimorphism.
        \item Prove or disprove that $\phi$ is an isomorphism.
\end{enumerate}
\solution
\part
All $a, b\in S_n$ are either even or odd. 
\begin{enumerate}
    \item Both $a$ and $b$ are even:\\
        $\phi(a) + \phi(b) = [0]+[0] = [0] = \phi(ab)$
    \item Both $a$ and $b$ are odd:\\
        $\phi(a) + \phi(b) = [1]+[1] = [0] = \phi(ab)$
    \item $a$ is even and $b$ is odd:\\
        $\phi(a) + \phi(b) = [0]+[1] = [1] = \phi(ab)$
\end{enumerate}
So $\phi$ is a homomorphism.
\part
The identity element in $\mathbb Z_2$ is $[0]$. The kernel of $\phi$ is $A_n$.
\part
For $S_1$ the only permutation is $e$. Therefore there is no even permutation and no element of $S_1$ maps to $[0]$. Thus $\phi$ is not an epimorphism.
\part
$\phi$ is not an epimorphism and therefore it cannot be an isomorphism.

\problem{\#31}
Let $f$ and $g$ be disjoint cycles in $S_n$. Prove that $fg = gf$.
\solution
If $f$ and $g$ are disjoint then any element permuted under one is invariant under the other. It is clear that the order in which the operations is applied has no impact on the result.

\problem{\#32}
Prove that the order of $A_n$ is $\frac{n!}{2}$.
\solution
Let us claim that in $S_n$ there are the same number of even and odd permutations. Let $O_n$ be the set of all odd permutations in $S_n$. Then select an arbitrary element of $O_n$ denoted as x. Define the function $f: A_n \to O_n$ by $f(y) = xy$.
\part
Suppose that $f(y) = f(y')$ for some $y, y'$ in $A_n$. Then $xy = xy'$ and thus $y=y'$. Therefore $f$ is one-to-one.
\part
The composition of and two odd cycles is an even cycle. For any $b\in O_n$ $x^{-1}b\in A_n$ and 
\begin{align}
        \begin{split}
                f(x^{-1}b) &= xx^{-1}b\\
                &= eb\\
                &= b. 
        \end{split}
\end{align}
The claim is proved. Knowing that the total number of permutations is equal to the magnitued of $S_n = n!$ and that there are an equal number of even and odd permutations we can say that $\mid A_n\mid = \frac{n!}{2}$.

%SECTION 4.2
\newpage
\section*{Section 4.2}
In Exercises 1-4, let $G$ be the given group. Write out the elements of a group of permutations that is isomorphic to $G$, and exhibit an isomorphism from $G$ to this group.

\problem{\#1}
Let $G$ be the additive group $\mathbb Z_3$.
\solution
\part
For all $a\in G$ let the permutation $f_a: G\to G$ be defined by $f_a(x) = ax$. So:
\begin{align}
        f_{[0]} = e & f_{[1]} = ([0], [1], [2]) & f_{[2]} = ([0], [2], [1])
\end{align}
Then the set $G' = \Set{f_{[0]}, f_{[1]}, f_{[2]}}$ is a group of permutations isomorphic to $G$ according to the proof of Cayley's Theorem.
\part
Then the mapping $\phi: G\to G'$ defined by $\phi(x) = f_x$ is an isomorphism between $G$ and $G'$ according to the proof of Cayley's Theorem.

\problem{\#2}
Let $G$ be the cyclic group $\langle a \rangle$ of order 5.
\solution
\part
For all $a\in G$ let the permutation $f_a: G\to G$ be defined by $f_a(x) = ax$. So:
\begin{align}
        f_{a} = (a, a^2, a^3, a^4, e) & f_{a^2} = (a^2, a^4, a, a^3, e) & f_{a^3} = (a^3, a, a^4, a^2, e) & f_{a^4} = (a^4, a^3, a^2, a, e) & f_{e} = e
\end{align}
Then the set $G' = \Set{f_{a}, f_{a^2}, f_{a^3}, f_{a^4}, f_{e}}$ is a group of permutations isomorphic to $G$ according to the proof of Cayley's Theorem.
\part
Then the mapping $\phi: G\to G'$ defined by $\phi(x) = f_x$ is an isomorphism between $G$ and $G'$ according to the proof of Cayley's Theorem.

\problem{\#3}
Let $G$ be the klein four group $\Set{e, a, b, ab}$ with its multiplication table as given:
\begin{table}[h]
\begin{tabular}{l|l|l|l|l|}
\cline{2-5}
$\cdot$                        & e  & a  & b  & ab \\ \hline
\multicolumn{1}{|l|}{e}  & e  & a  & b  & ab \\ \hline
\multicolumn{1}{|l|}{a}  & a  & e  & ab & b  \\ \hline
\multicolumn{1}{|l|}{b}  & b  & ab & e  & b  \\ \hline
\multicolumn{1}{|l|}{ab} & ab & a  & a  & e  \\ \hline
\end{tabular}
\end{table}
\solution
\part
For all $a\in G$ let the permutation $f_a: G\to G$ be defined by $f_a(x) = ax$. So:
\begin{align}
        f_{e} = e & f_{a} = (a, e)(b, ab) & f_{b} = (b, e)(a, ab) & f_{ab} = (e, ab)(b, a)
\end{align}
Then the set $G' = \Set{f_{e}, f_{a}, f_{b}, f_{ab}}$ is a group of permutations isomorphic to $G$ according to the proof of Cayley's Theorem.
\part
Then the mapping $\phi: G\to G'$ defined by $\phi(x) = f_x$ is an isomorphism between $G$ and $G'$ according to the proof of Cayley's Theorem.


\problem{\#4}
Let $G$ be the multiplicative group of units $\mathbb U_5 = \Set{[1], [2], [3], [4]} \subseteq \mathbb Z_{10}$
\solution

\problem{\#11}
For each element $a$ in the group $G$, define a mapping $k_a:G\to G$ by $k_a(x) = xa^{-1}$ for all $x$ in $G$.
\begin{enumerate}
        \item Prove that each $k_a$ is a permutation on the set of elements of $G$.
        \item Prove that $K=\Set{k_a\given a\in G}$ is a group with respect to mapping composition.
        \item Define $\phi: G\to K$ by $\phi(a) = k_a$ for each $a$ in $G$. Determine whether $\phi$ is always an isomorphism. This mapping $\phi$ is known as the \textbf{right regular representation} of $G$.
\end{enumerate}
\solution
\part
Assume that for some $x, y\in G$ $k_a(x) = k_a(y)$. This implies
\begin{align}
        xa &= ya\\
        x &= y
\end{align}
Therefore $k_a$ is one-to-one. Let $b$ be an arbitrary element in $G$. Let $x=ba^{-1}$. We know that $x$ is in $G$ because $G$ is a group. Then 
\begin{align}
        \begin{split}
                k_a(x) &= xa\\
                       &= (ba^{-1})a\\
                       &= b
        \end{split}
\end{align}
and $k_a$ is onto. $k_a$ is a bijection from $G$ to $G$ so $k_a$ is a permutation on the elements of $G$.
\part
\begin{enumerate}
        \item
                For any $k_a, k_b \in K$.
                \begin{align}
                        \begin{split}
                                k_ak_b(x) &= k_a(k_b(x))\\
                                          &= k_a(xb)\\
                                          &= (xb)a\\
                                          &= k_{ba}(x)
                        \end{split}
                \end{align}
                So $K$ is closed under composition of mappings.
        \item
                $k_e(x) = xe = x$ for all $x$ in $G$ so $k_e$ is the identity element in $K$.
        \item
                \begin{align}
                        \begin{split}
                                k_ak_{a^{-1}} &= k_{a^{-1}a}\\
                                             &= k_e
                        \end{split}
	       \end{align}
	       \centerline{and}
	       \begin{align}
                        \begin{split}
                                k_{a^{-1}}k_{a} &= k_{aa^{-1}}\\
                                                &= k_e
                        \end{split}
                \end{align}
\end{enumerate}
So $K$ is a group with respect to composition of mappings.
\part
$\phi: \phi(x) = k_a$ is clearly onto. 
\begin{alignat*}{2}
                \phi(a) = \phi(b) &\implies k_a = k_b\\
                                  &\implies k_a(x) = k_b(x) &&\forall x \in G\\
                                  &\implies xa = xb &&\forall x\in G\\
                                  &\implies a = b
\end{alignat*}
so $\phi$ is one-to-one.
\begin{align}
        \phi(a)\phi(b) = k_ak_b = k_ba = \phi(ba)
\end{align}
Thus $\phi$ is isomorphic when $\phi(ab) = \phi(ba)$.
\end{document}
