\documentclass[paper=a4, fontsize=11pt]{jhwhw} % A4 paper and 11pt font size
\usepackage{amsmath,amsfonts,amsthm, amssymb} % Math packages
\setlength\parindent{0pt} % Removes all indentation from paragraphs - comment this line for an assignment with lots of text
\usepackage{graphicx}
\usepackage{verbatim}
\usepackage{mathtools}
\usepackage{color}
\newcommand\SetSymbol[1][]{\:#1\vert\:}
\providecommand\given{} % to make it exist
\DeclarePairedDelimiterX\Set[1]\{\}{\renewcommand\given{\SetSymbol[\delimsize]}#1}

\begin{document}
\title{Survey of Algebra - Assignment \#5}
\author{Ben Haines}

%SECTION 3.3
\section*{Section 3.3}
\problem{\# 23}
Let $R$ be the equivalence relation on $G$ defined by $xRy$ if and only if there exists an element $a$ in $G$ such that $y=a^{-1}xa$. If $x\in Z(G)$, find $[x]$, the equivalence class containing $x$.
\solution
It is given that $x\in Z(G)$. Therefore $\forall y\in G$, $xy=yx$. By the definition of an equivalence relation if $xRy$ then $yRx$. Therefore
\begin{align}
xRy\implies y=a^{-1}xa\\
yRx \implies x=a^{-1}ya
\end{align}
\centerline{and}
\begin{align}
\begin{split}
x &= a^{-1}ya\\
&= a^{-1}(a^{-1}xa)a\\
&=a^{-1}(xa^{-1}a)a\\
&=a^{-1}xea\\
&=a^{-1}xa
\end{split}
\end{align}
From the above we can say that, given $xRy$
\begin{flalign*}
&&x = a^{-1}ya &= a^{-1}xa&&\\
&&aa^{-1}ya &= aa^{-1}xa&&\\
&&ya &= xa && \text{so by Theorem 3.4e   (4)}\\
&&y &= x&&
\end{flalign*}
Therefore for the equivalence relation $R$, $[x] = \Set{x}$.

\problem{\# 25}
Let $G$ be a group and $Z(G)$ its center. Prove or disprove that if $ab$ is in $Z(G)$, then $ab=ba$.
\solution
It is given that $ab\in Z(G)$ and therefore commutes with every element in $G$.
\begin{align}
\begin{split}
a^{-1}(ab) &= (ab)a^{-1}\\
b &= aba^{-1}\\
ba &= (aba^{-1})a\\
&= ab
\end{split}
\end{align}

\problem{\# 26}
Let $A$ be a given nonempty set. As noted in Example 2 of Section 3.1, $S(A)$ is a group with respect to mapping composition. for a fixed element $a$ in $A$, let $H_a$ denote the set of all $f\in S(A)$ such that $F9a) = a$. Prove that $H_a$ is a subgroup of $S(A)$.
\solution
\part
For any element $a\in A$ the identity mapping $e$ must have $e(a)=a$. $H_a$ contains all such mapping so it must contain the identity mapping.
\part
The inverse $f^{-1}$ of any mapping $f$ such that $f(a)=a$ must also have the property $f^{-1}(a)$. Therefore $H_a$ contains inverses.
\part
Consider two mappings $f$ and $g$ in $H_a$. Then
\begin{align}
\begin{split}
(f\circ g)a &= f(g(a))\\
&= f(a)\\
&=a
\end{split}
\end{align}
Therefore the composition of any two arbitrary functions in $H_a$ is also in $H_a$ and $H_a$ is closed under composition of mappings. $H_a$ has satisfied all of the necessary conditions and is thus a group with respect to mapping composition.

\problem{\# 29}
Let $G$ be an abelian group. For a fixed positive integer $n$, let $$G_n = \Set{a\in G\given a = x^n \text{ for some }x \in G}.$$ Prove that $G_n$ is a subgroup of $G$.
\solution
\part
$G_n$ contains the identity element $e$ for any integer $n$ because $e=e^n$ for any $n$. 
\part
For $a,b\in G$, $a=z^n$ and $b=y^n$ for some $z,y\in G$. Then
\begin{align}
\begin{split}
ab &= z^ny^n\\
&= (zy)^n
\end{split}
\end{align}
We know that the product $zy$ is in $G$ because $G$ is a group and thus closed. Therefore $ab=x^n$ for some $x\in G$, namely for $x=zy$. So $ab$ is in $G_n$ and $G_n$ is closed.
\part
Let $a=x^n$ for some $x\in G$. $G$ is a group so $a$ has an inverse $a^{-1}$ in $G$. So
\begin{align}
\begin{split}
e&=aa^{-1}\\
&=x^na^{-1}\\
&= x^n(x^n)^{-1}\\
&= x^n(x^{-1})^n\\
&= a(x^{-1})^n
\end{split}
\end{align}
So $a^{-1} = (x^{-1})^{n}$. $(x^{-1})^n$ is in $G_n$ because we know that $x^{-1}$ is in $G$. Therefore $G_n$ contains inverses. Having satisfied all the necessary conditions, $G_n$ is a subgroup of $G$.

\problem{\# 41}
Let $G$ be a cyclic group, $G=\langle a\rangle$. Prove that $G$ is abelian.
\solution
For any two $x,y\in G$ there exist some $m,n\in \mathbb Z$ such that $x=a^m$ and $y=a^n$. Then
\begin{align}
\begin{split}
xy &= a^ma^n\\
&= a^{m+n}\\
&= a^{n+m}\\
&= a^na^m\\
&= yx
\end{split}
\end{align}
Therefore $G = \langle a \rangle$ is abelian.

\problem{\# 45}
Assume that $G$ is a finite group, and let $G$ be a nonempty subset of $G$. Prove that $H$ is closed if and only if $H$ is a subgroup of $G$.
\solution
\part
First assume that $H$ is a subgroup of $G$. Then by definition of a group $H$ is closed. 
\part
Assume that $H$ is closed and that for some $x\in H$, $x^{-1}\not\in H$. $H$ is a subset of a finite group so therefore the order of $H$ is some integer $n$. Then for the product $xy$ where $y$ is any arbitrary element in $H$ there are $n$ possible values of $y$. $y$ is not $x^{-1}$ so we know that $xy \not= e$ and there are thus $n-1$ possible values of $xy$. This implies that for some $y, z\in H$ $xy=xz$ but $y\not=z$. However, this contradicts Theorem 3.4e which tells us that for $x,y,z\in G$, $xy = xz$ means that $y=z$. So there cannot exist an element in $H$ such that its inverse is not in $H$. Therefore $H$ satisfies the necessary conditions and is a subgroup of $G$. 

%SECTION 3.4
\newpage
\section*{Section 3.4}

\problem{\# 11f}
According to Exercise 33 of Section 3.1, if $n$ is prime, the nonzero elements of $Z_n$ form a group $U_n$ with respect to numtiplication. For $n=19$, show that this group $U_n$ is cyclic.
\solution
\begin{flalign*}
[2] &= [2]^1 &[3] &= [2]^{13} & [4] &= [2]^2 &[5] &= [2]^{16}\\
[6] &= [2]^{14}&[7] &= [2]^6 & [8] &= [2]^3&[9] &= [2]^8\\
[10] &= [2]^{17}&[11] &= [2]^{12} & [12] &= [2]^{15}&[13] &= [2]^5\\
[14] &= [2]^7&[15] &= [2]^{11} & [16] &= [2]^4&[17] &= [2]^{10}\\
[18] &= [2]^9& &
\end{flalign*}
It has been shown that $[2]$ is a generator for $\mathbb U_{19}$ and therefore $\mathbb U_{19}$ is cyclic.

\problem{\# 12f}
Find all distinct generators of the group $U_{19}$ described in Exercise 11. 
\solution
By Theorem 3.28 we know that $a^m$ is a generator for a cyclic group of order $n$ if and only if $(m, n) = 1$. We know from 11f that $a=2$ is a generator of $\mathbb U_{19}$. $\mathbb U_{19}$ has order 18. Therefore the distinct generators of $\mathbb U_{19}$ are given by
$$[2]^1, [2]^5, [2]^7, [2]^{11}, [2]^{13}, [2]^{17}$$
\centerline{=}
$$[2], [13], [14], [15], [3], [10]$$

\problem{\# 33}
If $G$ is a cyclic group, prove that the equation $x^2 = e$ has at most two distinct  solutions in $G$.
\solution
Let $G = \langle a \rangle$ be a cyclic group of order $n$. Let $x$ be an element of $G$ such that $x^{2} = e$. $x$ can be written as $a^{2k}$ for some $k \in \mathbb Z$. So
\begin{align}
\begin{split}
e &= x^2 \\
&= (a^{k})^{2}\\
&= a^{2k}
\end{split}
\end{align}
We know that $a^{0} = e$ so by Theorem 3.21 $2k \equiv 0 \mod{n}$. If $2 \nmid n$ then the only solution is $k=0$. Therefore $x$ must equal $a^{0}=e$. In the case that $2 \mid n$ The solutions are $n=0, \frac{n}{2}, -\frac{n}{2}$. We know by theorem 3.28 that $a^{n} = a^{-n}$ so the last two solutions are the same. Therefore there are at most two solutions. 

\problem{\# 35}
If $G$ is a cyclic group of order $p$ and $p$ is a prime, how many elements in $G$ are generators of $G$?
\solution
$G$ is cyclic so we know that $G=\langle a \rangle$ for some $a\in G$. By the statment made on page 178 we know that $G$ has $\phi(p)$ generators. When $p$ is prime, $\phi(p)=p-1$ so $G$ has $p-1$ generators.

\problem{\# 41}
Let $G$ be an abelian group. Prove that the set of all elements of finite order in $G$ forms a subgroup of $G$. This subgroup is called the torsion subgroup of $G$.
\solution
Let $H$ be the set of all elements of finite order in $G$. 
\part
The order of $e$ is 1 so $e\in H$ and $H$ is not empty.
\part
For $x,y \in H$ let $o(x)=n$ and $o(y)=m$. Then
\begin{align}
\begin{split}
(xy)^{nm} &= x^{nm}y^{nm}\\
&= (x^n)^m(y^m)^n\\
&= e^me^n\\
&= e
\end{split}
\end{align}
There exists an integer $nm$ such that $(xy)^{nm}=e$ so $\langle xy \rangle$ is finite and $H$ is closed.
\part
By Theorem 3.28, because $(1,-1)=1$, $\langle x \rangle = \langle x^{-1} \rangle$. So if $x\in H$ then $x^{-1}\in H$. So $H$ contains inverses.\\
$H$ has satisfied all of the necessary conditions and is thus a subgroup of $G$.


\problem{\# 42}
Let $d$ be a positive integer and $\phi(d)$ the Euler-phi function. Use corollary 3.27 and the additive groups $\mathbb Z_d$ to show that
$$n = \sum\limits_{d|n}\phi(d)$$
where the sum has one term for each positive divisor $d$ of $n$.
\solution
Suppose that $n$ is the order of some cyclic group $G$. Corollary 3.27 says that the distinct subgroups of finite cyclic group $G=\langle a \rangle$ are given by $\langle a^d \rangle$ where $d$ is a positive divisor of $n$ and that $\langle a^d \rangle$ has order $k$ where $n=dk$. The function $\phi(d)$ tells us the number of distinct generators of a group of order $d$.  Because the function is applied to every divisor of $n$ the original expression is equivalent to writing
$$n = \sum\limits_{kd=n}\phi(k)$$
This summation takes the total number of generators of each distinct subgroup of $G$ and adds them together. We know that every element of $G$ is the generator of a subgroup of $G$. So taking the sum of the number of generators for each distinct subgroup of $G$ is equivalent to the number of elements in $G$ which is equivalent to $n$.\\

%SECTION 3.5
\begin{comment}
\newpage
\section*{Section 3.5}

\problem{\# 7}
Find an isomorphism $\phi$ from the additive group $\mathbb Z$ to the multiplicative group
$$H = \left\{\left[\begin{array}{cc}
1 & n  \\
0 & 1   \end{array}\right] \middle| n\in \mathbb{Z} \right\}$$
\solution
Let $\phi: \mathbb Z \to H$ be defined as 
$$\phi(x) = 
\left[\begin{array}{cc}
1 & x  \\
0 & 1   \end{array}\right]
$$
\part
It is clear that $\phi$ is both one to one and onto.
\part
\begin{align}
\begin{split}
\phi(n)\phi(m) &= 
\left[\begin{array}{cc}
1 & n  \\
0 & 1   \end{array}\right]
\left[\begin{array}{cc}
1 & m  \\
0 & 1   \end{array}\right]\\
&=
\left[\begin{array}{cc}
1 & n+m  \\
0 & 1   \end{array}\right]\\
&= 
\left[\begin{array}{cc}
1 & m+n  \\
0 & 1   \end{array}\right]\\
&= \phi(n+m)
\end{split}
\end{align}
So $\phi$ preserves the operation. Therefore $\phi$ is an isomorphism from $\mathbb Z$ to $H$.

\problem{\# 30}
For an arbitrary positive integer $n$, prove that any two cyclic groups of order $n$ are isomorphic.
\solution
\textcolor{red}{THIS IS FUCKED}
Let $A = \langle a \rangle$ be some cyclic group of order $n$ and $\phi: \mathbb Z_n \to A$ defined by $\phi(x) = a^x$. This is clearly a bijection because $A$ is cyclic. Then for any $[x], [y] \in \mathbb Z_n$
$$x + y = kn + ([x] + [y])$$
\begin{align}
\begin{split}
\phi([x])\phi([y]) &= a^{[x]}a^{[y]}\\
&= a^{[x]+[y]}\\
&= a^{(x + y)-kn}\\\
&= a^{-kn}a^{(x + y)}\\
&= (a^n)^{-k} + a^{([x] + [y])}\\
&= a^{([x] + [y])}\\
&= \phi([x] + [y])
\end{split}
\end{align}
So $\phi$ preserves the operation and $A$ and $\mathbb Z_n$ are isomorphic. So any cyclic group of order $n$ is isomorphic to $\mathbb Z_n$ and thus by the transitive property, any two cyclic groups of the same order are isomorphic to each other.

\problem{\# 31}
Prove that any infinite cyclic group is isomorphic to $\mathbb Z$ under addition.
\solution
Let $G$ be the an infinite cyclic group defined by $\langle a \rangle$ and let $\phi: \mathbb Z \to G$ be defined by $\phi(x) = a^x$. $\phi$ is clearly both one to one and onto. Then
\begin{align}
\begin{split}
\phi(x)\phi(y) &= a^xa^y\\
&= a^{x+y}\\
&= \phi(x+y)
\end{split}
\end{align}
So $\phi$ preserves the operation and $G$ is isomorphic to $\mathbb Z$ under addition. 

\problem{\# 32}
Let $H$ be the group $\mathbb Z_6$ under addition. Find all isomorphisms from the multiplicative group $\mathbb U_7$ of units in $\mathbb Z_7$ to $H$.
\solution
For a a mapping to be an isomorphism it must map the identity element of the first group to the identity element of the second. Therefore we know that in all isomorphisms $\phi: \mathbb U_7 \to H$,  $\phi([1]) = [0]$. We also know that inverses must map to inverses. The element [6] is the only element in $\mathbb U_7$ that is its own inverse so it must map to $[3]$, the only element in $H$ that is its own inverse. We also know that the generators of $\mathbb U_7$ must map to the generators of $H$. Therefore $\phi([5])$ and $\phi([3])$ must map to either [1] or [5]. The remaining elements have no constraints. Therefore all isomorphisms can be listed as:
\begin{flalign*}
&\phi([1])=[0]&\phi([1])=[0]&&\phi([1])=[0]&&\phi([1])=[0]\\
&\phi([2])=[2]&\phi([2])=[4]&&\phi([2])=[2]&&\phi([2])=[4]\\
&\phi([3])=[1]&\phi([3])=[1]&&\phi([3])=[5]&&\phi([3])=[5]\\
&\phi([4])=[4]&\phi([4])=[2]&&\phi([4])=[4]&&\phi([4])=[2]\\
&\phi([5])=[5]&\phi([5])=[5]&&\phi([5])=[1]&&\phi([5])=[1]\\
&\phi([6])=[3]&\phi([6])=[3]&&\phi([6])=[3]&&\phi([6])=[3]\\
\end{flalign*}
\end{comment}
\end{document}