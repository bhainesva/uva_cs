\documentclass[paper=a4, fontsize=11pt]{jhwhw} % A4 paper and 11pt font size
\usepackage{amsmath,amsfonts,amsthm, amssymb} % Math packages
\setlength\parindent{0pt} % Removes all indentation from paragraphs - comment this line for an assignment with lots of text
\usepackage{graphicx}
\DeclareMathOperator{\lcm}{lcm}

\begin{document}
\title{Discrete Math - Homework Set \#4}
\author{Ben Haines (bmh5wx)}
\problem{}
Prove that for all positive integers $a$ and $b$, $a\mid b$ if, and only if, gcd$(a, b) = a$.
\solution

\problem{}
Prove the following statement. For all integers $a, b$ and $c$, if $a\mid b$ and $a\mid c$ then $a\mid (b-c)$ and $a\mid(b+c)$.
\solution

\problem{}
Prove that for ANY 4 digit nonnegative integer $n$, if the sum of the digits of $n$ is divisble by 3, then $n$ is divisible by 3.
\solution

\problem{}
When an integer $b$ is divided by 12, the remainder is 7. What is the remainder when $8b$ is divided by 12?
\solution

\problem{}
Prove that 3, 5, and 7 are the only prime numbers of the form $p, p+2$ and $p+4$. 
\solution

\problem{}
Prove that $\sqrt(5)$ is irrational.
\solution

\problem{}
Suppose $((A\cap C) \subseteq (B\cap C))$ and $((A\cup C) \subseteq (B\cup C))$. Prove using a proof by cases that $A\subseteq B$. Be sure to state your assumptions and what you are proving, and to give a step-by-step proof with rules of inference, laws of equivalence, algebraic truths and/or and othher justifications required to make the logic of your proof perfectly clear.

\problem{}
Prove by contradiction:
$$\text{If } (A\cap C) \subseteq B \lannd a\in C \text{ then } a\not\in (A - B)$$
Be sure you justify ever step of your proof with a rule of inference or law of equivalence or set definition. Rules of inference should be used when possible.
\solution

\problem{}
What conclusion(s) can you draw from the following set of premises? Explain the rules of inference you apply in obtaining the conclusion(s).
\begin{enumerate}
	\item $\neg p \implies r \land s$
	\item $t \implies s$
	\item $u \implies \neg p$
	\item $\neg w$
	\item $u \lor w$
\end{enumerate}
\solution



\problem{}
Use truth tables to determine which one of the following is a tautolog and which one is a contradiction. 
\solution

\problem{}
Using existential instantiation, universal instantiation and existential generalization (and any other inference rules necessary), prove
$$\forall x P(x) \land \exists x Q(x) \implies \exists x (P(x) \land Q(x))$$
\solution

\problem{}
\begin{quote}
	"Do you mean that you think you can find out the answer to it" said the March Hare. 
	"Exactly so," said Alice.
	"Then you should say what you mean," the March Hare went on.
	"I do," Alice hastily replied; "at least-at least I mean what I say-that's the same thing, you know."
	"Not the same thing a bit!" said the Hatter. "Why you might just as well say that 'I see what I eat' is the same thing as 'I eat what I see'!"
\end{quote}
\centerline{-from "A Mad Tea-Party" in \textit{Alice in Wonderland}, by Lewis Carroll}
The Hatter is correct. "I say what I mean" is not the same thing as "I mean what I say." Rewrite each of these two sentences in if-tehn form and explain the logical relation between them. Also, write the negation of each if-then statement.
\solution
\end{document}

