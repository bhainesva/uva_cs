\documentclass[paper=a4, fontsize=11pt]{jhwhw} % A4 paper and 11pt font size
\usepackage{amsmath,amsfonts,amsthm, amssymb} % Math packages
\setlength\parindent{0pt} % Removes all indentation from paragraphs - comment this line for an assignment with lots of text
\usepackage{graphicx}
\usepackage{verbatim}
\usepackage{enumerate}
\usepackage{mathtools}
\usepackage{color}
\newcommand\SetSymbol[1][]{\:#1\vert\:}
\providecommand\given{} % to make it exist
\DeclarePairedDelimiterX\Set[1]\{\}{\renewcommand\given{\SetSymbol[\delimsize]}#1}

\begin{document}
\title{Information Retrieval - Written Homework \#2}
\author{Ben Haines}

\section*{1. Joint, Marginal, and Conditional Probabilities} 
\begin{enumerate}
    \item [1.1] Compute $P(X \le 2, Y > 1)$.
        $$P(X \le 2, Y > 1) = P(X = 1, Y = 2) + P(X = 2, Y = 2) = \frac{1}{12}$$
    \item [1.2] Compute marginal probability mass function for $X$ and $Y$.
        $$f_X(1)  = \frac{1}{3} + \frac{1}{12} = \frac{5}{12}, f_X(2) = \frac{1}{6}, f_X(4) = \frac{1}{12} + \frac{1}{3} = \frac{5}{12}$$
        $$f_Y(1)  = \frac{1}{3} + \frac{1}{6} + \frac{1}{12} = \frac{7}{12}, f_Y(2) = \frac{1}{12} + \frac{1}{3} = \frac{5}{12}$$
    \item [1.3] Compute $P(Y = 2|X = 1)$.
        $$P(Y = 2|X = 1) = \frac{P(Y = 2, X = 1)}{P(X = 1)} = \frac{1/12}{1/3 + 1/12} = \frac{1}{5}$$
    \item [1.4] Are $X$ and $Y$ independent?\\
        $$P(X = 1) = \frac{5}{12} \not= \frac{4}{7} = P(X = 1|Y = 1)$$
        Thus they are not independent.
    \item [1.5] Define $Z = X - 2Y$, compute $P(X = 2|Z = 0)$. 
        $$P(X = 2|Z = 0) = \frac{P(X = 2, Z = 0)}{P(Z = 0)}$$
        $P(Z = 0) = P(X = 2, Y = 1) + P(X = 4, Y = 2) = \frac{1}{6} + \frac{1}{3} = \frac{1}{2}$ and
        $P(X = 2, Z = 0) = P(X = 2, Y = 1) = \frac{1}{6}$ so 
        $$P(X = 2|Z = 0) = \frac{1/6}{1/2} = \frac{1}{3}$$
    \item [1.6] Compute $\mathrm{E}[X|Y = 1]$.
        First compute $P(Y = 1) = \frac{1}{3} + \frac{1}{6} + \frac{1}{12} = \frac{7}{12}$.
        \begin{align*}
            E[X|Y = 1] &= 1(P(X = 1|Y = 1) + 2(P(X = 2| Y = 1)) + 4(P(X = 4|Y = 1))\\
                       &= 1(\frac{P(X = 1, Y = 1)}{P(Y = 1)}) + 2(\frac{P(X = 2, Y = 1)}{P(Y = 1)}) + 4(\frac{P(X = 4, Y = 1)}{P(Y = 1)})\\
                       &= \frac{1/3}{7/12} + 2(\frac{1/6}{7/12}) + 4(\frac{1/12}{7/12})\\
                       &= \frac{12}{7}
        \end{align*}
    \item [1.7] Compute $\mathrm{Var}[X|Y = 2]$.
        $\mathrm{Var}[X|Y = 2] = \mathrm{E}[X^2|Y = 2] - \mathrm{E}[X|Y = 2]^2$
        First compute $P(Y = 2) = \frac{1}{12} + \frac{1}{3} = \frac{5}{12}$.
        \begin{align*}
            E[X|Y = 2] &= 1(P(X = 1|Y = 2) + 2(P(X = 2| Y = 2)) + 4(P(X = 4|Y = 2))\\
                       &= 1(\frac{P(X = 1, Y = 2)}{P(Y = 2)}) + 2(\frac{P(X = 2, Y = 2)}{P(Y = 2)}) + 4(\frac{P(X = 4, Y = 2)}{P(Y = 2)})\\
                       &= \frac{1/12}{5/12} + 4(\frac{1/3}{5/12})\\
                       &= \frac{17}{5}
        \end{align*}
        and
        \begin{align*}
            E[X^2|Y = 2] &= 1(P(X = 1|Y = 2) + 4(P(X = 2| Y = 2)) + 16(P(X = 4|Y = 2))\\
                       &= 1(\frac{P(X = 1, Y = 2)}{P(Y = 2)}) + 4(\frac{P(X = 2, Y = 2)}{P(Y = 2)}) + 16(\frac{P(X = 4, Y = 2)}{P(Y = 2)})\\
                       &= \frac{1/12}{5/12} + 16(\frac{1/3}{5/12})\\
                       &= 13
        \end{align*}
        so $\mathrm{Var}[X|Y = 2] = 13 - \frac{17}{5}^2 = 1.44$
\end{enumerate}
\solution
\section*{2. Proof of Probabilistic Ranking Principle}
\solution
Assume that the system returns $n$ results. If the user reads all of the results or none of the results then it is clear the ranking doesn't matter. So we assume that the user reads some number $r\in \mathbb N$ of the results and then stops. We will show that changing the ordering of the results (by swapping two results) can only make the ordering worse from a risk minimization perspective. First consider the risk of presenting an irrelevant document. This risk is calculated as:
$$\prod_{i=0}^{r} 1 - p(R = 1|Q, D_i)$$
where $D_1$ is the first result presented, $D_2$ the second, and so on. Consider what happens when we swap the ordering of two documents $D_j$ and $D_k$. If $j, k \le r$ or if $j, k > r$ then the risk doesn't change. So we only care what happens when one is less than or equal to $r$ and the other is greater. Without loss of generality assume $j \le r$ and $k > r$. In calculating the risk we've replaced the term $1 - p(R = 1|Q, D_j)$ with $1 - p(R = 1| Q, D_k)$. By our ordering we know that $1 - p(R = 1|Q, D_j) \le 1 - p(R = 1|Q, D_k)$. Thus the risk either increases or stays the same as a result of the swap.

Now we look at the risk of missing a relevant document. This risk is computed as:
$$\prod_{i=r+1}^{n} p(R = 1|Q, D_i)$$
Consider what happens when we swap the ordering of two documents $D_j$ and $D_k$. If $j, k \le r$ or if $j, k > r$ then the risk doesn't change. So we only care what happens when one is less than or equal to $r$ and the other is greater. Without loss of generality assume $j \le r$ and $k > r$. In calculating the risk we've replaced the term $p(R = 1|Q, D_j)$ with $p(R = 1| Q, D_k)$. By our ordering we know that $p(R = 1|Q, D_j) \ge p(R = 1|Q, D_k)$. Thus the risk either increases or stays the same as a result of the swap.
\section*{3. Maximum Likelihood Estimation}
\solution
The likeliness function is:
\begin{align*}
    L(\mu, \sigma) &= f(x_1|\mu, \sigma) \times \cdots \times f(x_n|\mu, \sigma)\\
                   &= \frac{1}{\sqrt{2\pi\sigma^2}}e^{\frac{-(x_1 - \mu)^2}{2\sigma^2}} \times \cdots \times \frac{1}{\sqrt{2\pi\sigma^2}}e^{\frac{-(x_n - \mu)^2}{2\sigma^2}}\\
                   &= \frac{1}{\sqrt{(2\pi\sigma^2)^n}}e^{\frac{-1}{2\sigma^2}\sum_{i=1}^{n}(x_i - \mu)^2}
\end{align*}
Take the log to get:
$$L(\mu, \sigma) = -\frac{n}{2}(\ln{2\pi}) - \frac{1}{2\sigma^2}\sum_{i=1}^{n}(x_i - \mu)^2$$
To maximize this function we take the derivative and then set it equal to zero. First we take the derivative with respect to $\mu$. 
$$\frac{\partial L}{\partial \mu} = \frac{1}{\sigma^2}\sum_{i=1}^{n}(x_i - \mu) = \frac{1}{\sigma^2}\sum_{i=1}^{n}x_i - n\mu$$.

In order for this to be zero we must have $\hat\mu = \bar{x}$.

We repeat the process for $\sigma$. We take the derivative by $\sigma^2$ to simplify computations. 
$$\frac{\partial L}{\partial \sigma^2} = -\frac{n}{2\sigma^2} + \frac{1}{(2\sigma^2)^2}[\sum_{i=1}^{n}(x_i - \mu)^2] = \frac{1}{2\sigma^2}[\frac{1}{\sigma^2}\sum_{i=1}^{n}(x_i - \mu)^2 - n]$$.

In order for this to be zero we must have
\begin{align*}
    \frac{1}{2\sigma^2}[\frac{1}{\sigma^2}\sum_{i=1}^{n}(x_i - \mu)^2] &= n\\
    \implies \sigma^2 &= \frac{1}{n}\sum_{i=1}^{n}(x_i - \mu)^2\\
    \implies \hat\sigma &= \sqrt{\frac{1}{n}\sum_{i=1}^{n}(x_i - \hat\mu)^2}
\end{align*}

\section*{4. Cosine vs. Euclidean Distance}
\solution
Let $x = \Set{x_1, \ldots, x_n}$ and $y = \Set{y_1, \ldots, y_n}$. Then
$$C(x, y) = \frac{x\cdot y}{\lvert x\rvert \lvert y \rvert}$$
We are dealing with unit vectors so the denominator is zero and
$$C(x, y) = x \cdot y = \sum_{i = 1}^{n} x_iy_i$$
Euclidean distance is given by $E(x, y) = \sqrt{\sum_{i=1}^{n} (x_i - y_i)^2}$. Then to discover the relationship between $C$ and $E$ consider $E(x, y)^2$.
\begin{align*}
    E(x, y)^2 &= \sum_{i=1}^{n} (x_i - y_i)^2\\
              &= \sum_{i=1}^{n} x_i^2 + y_i^2 - 2x_iy_i\\
              &= \sum_{i=1}^{n} x_i^2 + \sum_{i=1}^{n}y_i^2 - 2\sum_{i=1}^{n}x_iy_i\\
              &= 2 - 2\sum_{i=1}^{n}x_iy_i\\
              &= 2(1 - C(x_iy_i))
\end{align*}
\section*{5. $\alpha_d$ in Language Model Ranking Modules}
\solution
By definition if $w$ is not seen in $d$ then $p(w|d) = \alpha_d p(w|REF)$. For Dirichlet prior smoothing 
$$p(w|d) = \frac{\lvert d\rvert}{\lvert d \rvert + \mu}\frac{c(w, d)}{\lvert d\rvert} + \frac{\mu}{\lvert d\rvert + \mu}p(w|REF)$$
In the case of an unseen word $c(w, d) = 0$. This implies that
$$p(w|d) = \alpha_d p(w|REF) = \frac{\mu}{\lvert d\rvert + \mu}p(w|REF)\\
\implies \alpha_d = \frac{\mu}{\lvert d\rvert + \mu}$$

\section*{Bonus Question: $\alpha_d$ in Language Model Ranking Modules}
\solution
\begin{itemize}
    \item Linear Interpolation Smoothing\\
        We are given $p(w\mid d) = (1-\lambda)\frac{c(w,d)}{\lvert d\rvert} + \lambda p(w\mid REF)$.
        \begin{align*}
            \alpha_d &= \frac{1 - \sum_{\text{w seen}} p(w\mid d)}{\sum_{\text{w unseen}} p(w\mid REF)}\\
                     &= \frac{\sum_{\text{w unseen}} p(w\mid d)}{\sum_{\text{w unseen}} p(w\mid REF)}\\
                     &= \frac{\sum_{\text{w unseen}} (1-\lambda)\frac{c(w,d)}{\lvert d\rvert} + \lambda p(w\mid REF)}{\sum_{\text{w unseen}} p(w\mid REF)}\\
                     &= \frac{\sum_{\text{w unseen}} \lambda p(w\mid REF)}{\sum_{\text{w unseen}} p(w\mid REF)}\\
                     &= \lambda
        \end{align*}
\end{itemize}
\end{document}
