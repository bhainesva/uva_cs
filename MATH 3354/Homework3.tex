\documentclass[paper=a4, fontsize=11pt]{jhwhw} % A4 paper and 11pt font size
\usepackage{amsmath,amsfonts,amsthm, amssymb} % Math packages
\setlength\parindent{0pt} % Removes all indentation from paragraphs - comment this line for an assignment with lots of text
\usepackage{graphicx}
\usepackage{mathtools}
\newcommand\SetSymbol[1][]{\:#1\vert\:}
\providecommand\given{} % to make it exist
\DeclarePairedDelimiterX\Set[1]\{\}{\renewcommand\given{\SetSymbol[\delimsize]}#1}

\begin{document}
\title{Survey of Algebra - Assignment \#3}
\author{Ben Haines}

%SECTION 2.5
\section*{Section 2.5}
In this section, all variables are integers.
Find a solution $x\in \mathbb Z, 0\le x < n$, for each of the congruences $ax=b \pmod n$ in Exercises 4, 6, 22, and 24.
\problem{\# 4}
$2x\equiv 3 \pmod 5$
\solution
\begin{equation}
\begin{alignedat}{2}
2x&\equiv 3 & &\pmod 5\\
2x &\equiv 8 & &\pmod 5\\
x &\equiv 4 & &\pmod 5
\end{alignedat}
\end{equation}


\problem{\# 6}
$3x\equiv 4 \pmod {13}$
\solution
\begin{equation}
\begin{alignedat}{2}
3x&\equiv 4 & &\pmod {13}\\
3x &\equiv 30 & &\pmod {13}\\
x &\equiv 10 & &\pmod {13}
\end{alignedat}
\end{equation}

\problem{\# 22}
$57x + 7\equiv 78 \pmod {58}$
\solution
\begin{equation}
\begin{alignedat}{2}
57x + 7&\equiv 78 & &\pmod {58}\\
58x + 7&\equiv 78 + x & &\pmod {58}\\
7&\equiv 78 + x & &\pmod {58}\\
7&\equiv 20 + x & &\pmod {58}\\
x&\equiv -13 & &\pmod {58}\\
x&\equiv 45 & &\pmod {58}
\end{alignedat}
\end{equation}

\problem{\# 24}
$82x + 23\equiv 2 \pmod {47}$
\solution
\begin{equation}
\begin{alignedat}{2}
35x + 23&\equiv 2 & &\pmod {47}\\
35x + 23&\equiv 49 & &\pmod {47}\\
5x + 23&\equiv 7 & &\pmod {47}\\
82x&\equiv -16 & &\pmod {47}\\
5x&\equiv 125 & &\pmod {47}\\
x&\equiv 25 & &\pmod {47}
\end{alignedat}
\end{equation}\\


Use the results in Exercises 38 and 39 to determine whether there are solutions. If there are, find $d$ incongruent solutions modulo $n$.
\problem{\# 40}
$4x\equiv 18 \pmod{28}$
\solution
Let $d=(4,28)=4$. $b=18$. From Exercise 38 we know that if there is a solution to $ax=b$ then $d\mid b$. $4 \nmid 18$. Therefore there are no solutions.

\problem{\#42}
$18x\equiv 33 \pmod{15}$
\solution
Let $d=(18, 15) = 3$. $3\mid 33$ so by Exercise we know there are solutions. 
\begin{equation}
\begin{alignedat}{2}
18x&\equiv 33 & &\pmod{15}\\
18x&\equiv 18 & &\pmod{15}\\
x&\equiv 1 & &\pmod{15}
\end{alignedat}
\end{equation}
Let $x_1 = 1$, then by Exercise 39 we know the set of discongruent solutions is given by
$$x_1, x_1+n_0, x_1+2n_0, \ldots x_1 + (d-1)n_0$$
For $n_0$ such that $dn_0=15$.
So the solutions are $x=1, 6, 11$.


\problem{\#44}
$35x\equiv 10\pmod{20}$
\solution
Let $d=(35, 20) = 5$. $5\mid 20$ so by Exercise 38 we know there are solutions. 
\begin{align}
\begin{split}
35x\equiv 10 \pmod{20}\\
15x\equiv 10 \pmod{20}\\
15x\equiv 30 \pmod{20}\\
x\equiv 2 \pmod{20}
\end{split}
\end{align}
Let $x_1 = 2$, then by Exercise 39 we know the set of discongruent solutions is given by
$$x_1, x_1+n_0, x_1+2n_0, \ldots x_1 + (d-1)n_0$$
For $n_0$ such that $dn_0=20$.
So the solutions are $x=2, 6, 10, 14, 18$.

%SECTION 2.6
\newpage
\section*{Section 2.6}
\problem{\#4b}
Make a multiplication table for $\mathbb{Z}_3$.
\solution
$$\vbox{\tabskip0.5em\offinterlineskip
    \halign{\strut$#$\hfil\ \tabskip1em\vrule&&$#$\hfil\cr
    ~   & [0]   & [1]   & [2]       \cr
    \noalign{\hrule}\vrule height 12pt width 0pt
    [0]   & [0]   & [0]   & [0]       \cr
    [1]   & [0]   & [1] & [2]         \cr
    [2] & [0] & [2]   & [0]      \cr
}}$$


\problem{\#5}
Find the multiplicative inverse of each given element.
\begin{itemize}
\item $[7]$ in $\mathbb{Z}_{11}$
\item $[16]$ in $\mathbb{Z}_{27}$
\end{itemize}
\solution
\soln{b)}
\begin{equation}
\begin{alignedat}{2}
[7][x] &= [1]\\
7x&\equiv 1 & &\pmod{11}\\
7x&\equiv 56 & &\pmod{11}\\
x&\equiv 8 & &\pmod{11}
\end{alignedat}
\end{equation}

So $[7]^{-1}$ in $\mathbb{Z}_{11}$ is $[8]$.
\soln{d)}
\begin{equation}
\begin{alignedat}{2}
[16][x] &= [1]\\
16x&\equiv 1 & &\pmod{27}\\
16x&\equiv 28 & &\pmod{27}\\
4x&\equiv 7 & &\pmod{27}\\
4x&\equiv 88 & &\pmod{27}\\
x&\equiv 22 & &\pmod{27}
\end{alignedat}
\end{equation}
So $[16]^{-1}$ in $\mathbb{Z}_{27}$ is $[22]$.

\problem{\#6}
For each of the following $\mathbb{Z}_n$, list all the elements in $\mathbb{Z}_n$ that have multiplicative inverses in $\mathbb{Z}_n$.
\begin{itemize}
\item $\mathbb{Z}_8$
\item $\mathbb{Z}_{12}$
\end{itemize}
\soln{b)}
$$[1], [3], [5], [7]$$
\soln{d)}
$$[1], [5], [7], [11]$$

\problem{\#9}
Let $[a]$ be an element of $\mathbb{Z}_n$ that has a multiplicative inverse $[a]^{-1}$ in $\mathbb{Z}_n$. Prove that $[x]=[a]^{-1}[b]$ is the unique solution in $\mathbb{Z}_n$ to the equation $[a][x] = [b]$.
\solution
1. For $[x]=[a]^{-1}[b]$
\begin{align}
\begin{split}
[a][x]&=[a][a]^{-1}[b]\\
&=[1][b]\\
&=[b]
\end{split}
\end{align}
So $[x]=[a]^{-1}[b]$ is a solution.\\
2. Assume there is another solution $[y]$. Then
\begin{align}
\begin{split}
[a][y] &= [b]\\
[a]^{-1}[a][y] &= [a]^{-1}[b]\\
[1][y]&=[a]^{-1}[b]\\
[y]&=[a]^{-1}[b]\\
[y]&=[x]
\end{split}
\end{align}
Therefore [x] is a unique solution.


\problem{\#10}
Solve each of the following equations by finding $[a]^{-1}$ and using the result in Exercise 9.
\begin{itemize}
\item $[8][x] = [7] \text{ in }\mathbb{Z}_{11}$
\item $[8][x] = [11]\text{ in }\mathbb{Z}_{15}$
\end{itemize}

\problem{\#22}
Let $p$ be a prime integer. Prove that $[1]$ and $[p-1]$ are the only elements in $\mathbb{Z}_p$ that are their own multiplicative inverses.
\solution
Assume that $[a]$ is an element of $\mathbb{Z}_p$ such that $[a][a]=[1]$. Then 
\begin{align}
\begin{split}
a^2 &\equiv 1 \pmod{p}\\
a^2-1&\equiv 0 \pmod{p}\\
(a+1)(a-1)&\equiv 0 \pmod{p}
\end{split}
\end{align}
If there were zero divisors $[x],[y]$ in $\mathbb{Z}_p$ that would imply that $xy\equiv p \mod{p}$. We know that this is not the case because $p$ is prime. Therefore there are no zero divisors and either $(x-1)\equiv 0$ or $(x+1) \equiv 0$. In the first case $x \equiv 1$ and in the second $x \equiv p-1$. There are no other cases so these are the only elements in $\mathbb{Z}_p$ that are their own multiplicative inverses.

%SECTION 3.1
\newpage
\problem{\#38}
Let $G$ be the set of all matrices in $M_3(R)$ that have the form
\[ \left[ \begin{array}{ccc}
a & 0 & 0 \\
0 & b & 0 \\
0 & 0 & c \end{array} \right]\]
with all three numbers $a,b,$ and $c$ nonzero. Prove or disprove that $G$ is a group with respect to multiplication.
\solution
\part
\[ \left[ \begin{array}{ccc}
a_1 & 0 & 0 \\
0 & b_1 & 0 \\
0 & 0 & c_1 \end{array} \right]
\left[ \begin{array}{ccc}
a_2 & 0 & 0 \\
0 & b_2 & 0 \\
0 & 0 & c_2 \end{array} \right] 
= 
\left[ \begin{array}{ccc}
a_1a_2 & 0 & 0 \\
0 & b_1b_2 & 0 \\
0 & 0 & c_1c_2 \end{array} \right]\]

The products $a_1a_2, b_1b_2,$ and $c_1c_2$ are not zero because none of their components are zero and there are no zero divisors in $\mathbb Z$. So the set $G$ is closed under multiplication.

\part
\[ \left(\left[ \begin{array}{ccc}
a & 0 & 0 \\
0 & b & 0 \\
0 & 0 & c \end{array} \right]
\left[ \begin{array}{ccc}
d & 0 & 0 \\
0 & e & 0 \\
0 & 0 & f \end{array} \right]\right)  
\left[ \begin{array}{ccc}
x & 0 & 0 \\
0 & y & 0 \\
0 & 0 & z \end{array} \right]
=
\left[ \begin{array}{ccc}
adx & 0 & 0 \\
0 & bey & 0 \\
0 & 0 & cfz \end{array} \right]
\]
\centerline{and}
\[ \left[ \begin{array}{ccc}
a & 0 & 0 \\
0 & b & 0 \\
0 & 0 & c \end{array} \right]
\left(\left[ \begin{array}{ccc}
d & 0 & 0 \\
0 & e & 0 \\
0 & 0 & f \end{array} \right]  
\left[ \begin{array}{ccc}
x & 0 & 0 \\
0 & y & 0 \\
0 & 0 & z \end{array} \right]\right)
=
\left[ \begin{array}{ccc}
adx & 0 & 0 \\
0 & bey & 0 \\
0 & 0 & cfz \end{array} \right]
\]
Therefore multiplication is associative in $G$.

\part
\[ \left[ \begin{array}{ccc}
a & 0 & 0 \\
0 & b & 0 \\
0 & 0 & c \end{array} \right]
\left[ \begin{array}{ccc}
1 & 0 & 0 \\
0 & 1 & 0 \\
0 & 0 & 1 \end{array} \right]  
=
\left[ \begin{array}{ccc}
a & 0 & 0 \\
0 & b & 0 \\
0 & 0 & c \end{array} \right]
\]
\centerline{and}
\[ \left[ \begin{array}{ccc}
1 & 0 & 0 \\
0 & 1 & 0 \\
0 & 0 & 1 \end{array} \right]
\left[ \begin{array}{ccc}
a & 0 & 0 \\
0 & b & 0 \\
0 & 0 & c \end{array} \right]  
=
\left[ \begin{array}{ccc}
a & 0 & 0 \\
0 & b & 0 \\
0 & 0 & c \end{array} \right]
\]
Therefore 
$
\left[ \begin{array}{ccc}
1 & 0 & 0 \\
0 & 1 & 0 \\
0 & 0 & 1 \end{array} \right]
$
is the identity element in $G$.

\part
For all matrices $A$ in $G$ there exists an inverse $A^{-1}$ so that
\[
AA^{-1}
=
\left[\begin{array}{ccc}
a & 0 & 0 \\
0 & b & 0 \\
0 & 0 & c \end{array}\right]
\left[\begin{array}{ccc}
\frac{1}{a} & 0 & 0 \\
0 & \frac{1}{b} & 0 \\
0 & 0 & \frac{1}{c} \end{array}\right]
=
\left[\begin{array}{ccc}
1 & 0 & 0 \\
0 & 1 & 0 \\
0 & 0 & 1 \end{array}\right]
=
e
\]

Having satisfied the four conditions above, it can be concluded that $G$ is a group with respect to multiplication.

\problem{\#39}
Let $G$ be the set of all matrices in $M_3(R)$ that have the form
\[
\left[\begin{array}{ccc}
1 & a & b \\
0 & 1 & c \\
0 & 0 & 1 \end{array}\right]
\]
for arbitrary real numbers $a,b,$ and $c$. Prove or disprove that $G$ is a group with respect to multiplication.
\solution
\part
\[
\left[\begin{array}{ccc}
1 & a & b \\
0 & 1 & c \\
0 & 0 & 1 \end{array}\right]
\left[\begin{array}{ccc}
1 & x & y \\
0 & 1 & z \\
0 & 0 & 1 \end{array}\right]
=
\left[\begin{array}{ccc}
1 & a+x & b+y \\
0 & 1 & c+z \\
0 & 0 & 1 \end{array}\right]
\]\\

The real numbers are closed under addition so the resulting matrix is also a member of set $G$.

\part
Matrix multiplication is associative. That means it is also associative for all subsets of matrices. $G$ is such a matrix.

\part
For all matrices $A$ in $G$
\[
\left[\begin{array}{ccc}
1 & a & b \\
0 & 1 & c \\
0 & 0 & 1 \end{array}\right]
\left[\begin{array}{ccc}
1 & 0 & 0 \\
0 & 1 & 0 \\
0 & 0 & 1 \end{array}\right]
=
\left[\begin{array}{ccc}
1 & a & b \\
0 & 1 & c \\
0 & 0 & 1 \end{array}\right]
\]
\centerline{and}
\[
\left[\begin{array}{ccc}
1 & 0 & 0 \\
0 & 1 & 0 \\
0 & 0 & 1 \end{array}\right]
\left[\begin{array}{ccc}
1 & a & b \\
0 & 1 & c \\
0 & 0 & 1 \end{array}\right]
=
\left[\begin{array}{ccc}
1 & a & b \\
0 & 1 & c \\
0 & 0 & 1 \end{array}\right]
\]\\

So 
$
\left[\begin{array}{ccc}
1 & 0 & 0 \\
0 & 1 & 0 \\
0 & 0 & 1 \end{array}\right]
$
is the identity element in $G$ under multiplication.

\problem{\#40}
Prove or disprove that the set $G$ in Exercise 38 is a group with respect to addition.
\solution
\[
\left[\begin{array}{ccc}
1 & 0 & 0 \\
0 & 1 & 0 \\
0 & 0 & 1 \end{array}\right]
+
\left[\begin{array}{ccc}
-1 & 0 & 0 \\
0 & -1 & 0 \\
0 & 0 & -1 \end{array}\right]
=
\left[\begin{array}{ccc}
0 & 0 & 0 \\
0 & 0 & 0 \\
0 & 0 & 0 \end{array}\right]
\]\\

The matrix
$
\left[\begin{array}{ccc}
0 & 0 & 0 \\
0 & 0 & 0 \\
0 & 0 & 0 \end{array}\right]
$
is not a member of $G$. So $G$ is not closed under addition. $G$ is not a group with respect to addition.

\problem{\#41}
Prove or disprove that the set $G$ in Exercise 39 is a group with respect to addition.
\solution
\part
\[
\left[\begin{array}{ccc}
1 & 1 & 1 \\
0 & 1 & 1 \\
0 & 0 & 1 \end{array}\right]
+
\left[\begin{array}{ccc}
1 & 1 & 1 \\
0 & 1 & 1 \\
0 & 0 & 1 \end{array}\right]
=
\left[\begin{array}{ccc}
2 & 2 & 2 \\
0 & 2 & 2 \\
0 & 0 & 2 \end{array}\right]
\]\\

The resulting matrix is not a member of $G$. Therefore $G$ is not closed under addition. $G$ is not a group with respect to addition.

\problem{\#42a}
For an arbitrary set $A$, the power set $\mathcal P(A)$ was defined in Section 1.1 by $\mathcal P(A) = \Set{X\given X\subseteq A}$, and addition in $\mathcal P(A)$ was defined by
\begin{align}
\begin{split}
X+Y &= (X\cup Y) - (X\cap Y)\\
&=(X-Y)\cup (Y-X)
\end{split}
\end{align}
Prove that $\mathcal P(A)$ is a group with respect to this operation of addition.
\solution
\part
Addition as defined here can be summarized as taking every element that is a member of either $X$ or $Y$ but not both. Being members of the set $\mathcal P(A)$, every element of both $X$ and $Y$ must also be an element of $A$. This means that the result of their addition is composed entirely of elements from $X$ and $Y$. The result is a subset of $A$, and is contained in $\mathcal P(A)$. So $\mathcal P(A)$ is closed with respect to this operation of addition.

\part
\begin{align}
(Y+Z) &= (Y-Z)\cup (Z-Y)\\
&= (Y\cap Z')\cup (Y'\cap Z)
\end{align}
We can then use this substitution
\begin{align}
\begin{split}
X+(Y+Z) &= (X-(Y+Z))\cup ((Y+Z)-X)\\
&=(X\cap (Y+Z)')\cup ((Y+Z)\cap X')\\
&=(X\cap ((Y\cap Z')\cup (Y'\cap Z))'\cup(((Y\cap Z') \cup (Y'\cap Z))\cap X')\\
&=(X\cap ((Y\cup Z)\cap (Y'\cup Z'))'\cup(((Y\cap Z') \cup (Y'\cap Z))\cap X')\\
&=(X\cap ((Y\cup Z)'\cup (Y'\cup Z)')\cup(((Y\cap Z') \cup (Y'\cap Z))\cap X')\\
&=(X\cap ((Y'\cap Z')\cup (Y\cap Z))\cup(((Y\cap Z') \cup (Y'\cap Z))\cap X')\\
&=((X\cap Y'\cap Z')\cup (X\cap Y\cap Z))\cup((X'\cap Y\cap Z') \cup (X'\cap Y'\cap Z))\\
&=(X\cap Y'\cap Z')\cup (X\cap Y\cap Z)\cup(X'\cap Y\cap Z') \cup (X'\cap Y'\cap Z)\\
&=(X\cap Y'\cap Z')\cup (X'\cap Y\cap Z')\cup(X'\cap Y'\cap Z) \cup (X\cap Y\cap Z)\\
&=(X\cap Y'\cap Z')\cup (X'\cap Y\cap Z')\cup(((X'\cap Y') \cup (X\cap Y))\cap Z)\\
&=(((X\cap Y')\cup (X'\cap Y))\cap Z') \cup(((X'\cap Y') \cup (X\cap Y))\cap Z)\\
&=(((X\cap Y')\cup (X'\cap Y))\cap Z') \cup(((X\cup Y)' \cup (X'\cup Y')')\cap Z)\\
&=(((X\cap Y')\cup (X'\cap Y))\cap Z') \cup(((X\cup Y) \cap (X'\cup Y'))'\cap Z)\\
&=(((X\cap Y')\cup (X'\cap Y))\cap Z') \cup(((X\cup Y) \cap (X'\cup Y'))'\cap Z)\\
&=(((X\cap Y')\cup (X'\cap Y))\cap Z') \cup(((X\cap Y') \cup (X'\cap Y))'\cap Z)\\
&=(((X\cap Y')\cup (X'\cap Y))\cap Z') \cup((X+Y)'\cap Z)\\
&=((X+Y)\cap Z') \cup((X+Y)'\cap Z)\\
&=(X+Y)+Z
\end{split}
\end{align}
So this operation of addition is associative in $\mathcal P(A)$.

\part
For any set $X$ in $\mathcal P(A)$ it can be seen that $X+\emptyset = X$. Thus $\emptyset$ is the identity element for addition in $\mathcal P(A)$.

\part
For any set $X$ in $\mathcal P(A)$  it can be seen that
\begin{align}
\begin{split}
X+X &= (X\cap X') \cup (X'\cap X)\\
&= \emptyset
\end{split}
\end{align}
So every set in $\mathcal P(A)$ is its own inverse.\\

Having satisfied the four necessary conditions, we can conclude that $\mathcal P(A)$ is a group with respect to the addition operation defined above.

\end{document}