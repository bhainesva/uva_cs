%%%%%%%%%%%%%%%%%%%%%%%%%%%%%%%%%%%%%%%%%
% Short Sectioned Assignment
% LaTeX Template
% Version 1.0 (5/5/12)
%
% This template has been downloaded from:
% http://www.LaTeXTemplates.com
%
% Original author:
% Frits Wenneker (http://www.howtotex.com)
%
% License:
% CC BY-NC-SA 3.0 (http://creativecommons.org/licenses/by-nc-sa/3.0/)
%
%%%%%%%%%%%%%%%%%%%%%%%%%%%%%%%%%%%%%%%%%

%----------------------------------------------------------------------------------------
%    PACKAGES AND OTHER DOCUMENT CONFIGURATIONS
%----------------------------------------------------------------------------------------

\documentclass[paper=a4, fontsize=11pt]{scrartcl} % A4 paper and 11pt font size

\usepackage[T1]{fontenc} % Use 8-bit encoding that has 256 glyphs
\usepackage[english]{babel} % English language/hyphenation
\usepackage{amsmath,amsfonts,amsthm} % Math packages
\usepackage{enumerate}

\usepackage{sectsty} % Allows customizing section commands
\allsectionsfont{\centering \normalfont\scshape} % Make all sections centered, the default font and small caps

\usepackage{fancyhdr} % Custom headers and footers
\pagestyle{fancyplain} % Makes all pages in the document conform to the custom headers and footers
\fancyhead{} % No page header - if you want one, create it in the same way as the footers below
\fancyfoot[L]{} % Empty left footer
\fancyfoot[C]{} % Empty center footer
\fancyfoot[R]{\thepage} % Page numbering for right footer
\renewcommand{\headrulewidth}{0pt} % Remove header underlines
\renewcommand{\footrulewidth}{0pt} % Remove footer underlines
\setlength{\headheight}{13.6pt} % Customize the height of the header

\numberwithin{equation}{section} % Number equations within sections (i.e. 1.1, 1.2, 2.1, 2.2 instead of 1, 2, 3, 4)
\numberwithin{figure}{section} % Number figures within sections (i.e. 1.1, 1.2, 2.1, 2.2 instead of 1, 2, 3, 4)
\numberwithin{table}{section} % Number tables within sections (i.e. 1.1, 1.2, 2.1, 2.2 instead of 1, 2, 3, 4)

\setlength\parindent{0pt} % Removes all indentation from paragraphs - comment this line for an assignment with lots of text

%----------------------------------------------------------------------------------------
%    TITLE SECTION
%----------------------------------------------------------------------------------------

\newcommand{\horrule}[1]{\rule{\linewidth}{#1}} % Create horizontal rule command with 1 argument of height

\title{   
\normalfont \normalsize
\textsc{MATH 3354} \\ [25pt] % Your university, school and/or department name(s)
\horrule{0.5pt} \\[0.4cm] % Thin top horizontal rule
\huge Homework \#1 \\ % The assignment title
\horrule{2pt} \\[0.5cm] % Thick bottom horizontal rule
}

\author{Ben Haines} % Your name

\date{\normalsize\today} % Today's date or a custom date

\begin{document}

\maketitle % Print the title

%----------------------------------------------------------------------------------------
%    PROBLEM 1
%----------------------------------------------------------------------------------------

\section{Section 1.1}
\subsection{Problem \#38}
Prove or disprove that  $\mathcal P \left({A \cup B}\right) = \mathcal P \left({A}\right) \cup \mathcal P \left({B}\right)$

\begin{align}
\begin{split}
\mathcal P \left({A \cup B}\right)
&= \{X|X \subseteq (A \cup B)\}\\
&=\{X|X \subseteq A\} \cup \{X|X \subseteq B\}\\
&=\mathcal P \left({A}\right) \cup \mathcal P \left({B}\right)
\end{split}                   
\end{align}

Phasellus viverra nulla ut metus varius laoreet. Quisque rutrum. Aenean imperdiet. Etiam ultricies nisi vel augue. Curabitur ullamcorper ultricies

%------------------------------------------------

\subsection{Problem \#40}

Prove or disprove that $\mathcal P \left({A - B}\right) = \mathcal P \left({A}\right) - \mathcal P \left({B}\right)$
\begin{align}
\mathcal P \left({A - B}\right)
&= \{X|X \subseteq (A - B)\}\\
&= \{X|X \subseteq \{x \in U | x \in A \text{ and } x \not\in B\}\}\\
&= \{X|X \subseteq (\{x \in U | x \in A\} \cap \{x \in U | x \not\in B\})\}\\
&= \{X|X \subseteq (A \cap B')\}\\
&= \{x \subseteq U | x \in (\mathcal P \left({A}\right) \cap \mathcal P \left({B'}\right)) \}\\
&= \{x \subseteq U | x \in \mathcal P \left({A}\right) \text{ and } x \not\in \mathcal P \left({B}\right) \}\\
&= \{x \subseteq  U | x \in \{X|X \subseteq A\} \text{ and } x \not\in \{X|X \subseteq B \}\}\\
&= \mathcal P(A) - \mathcal P(B)
\end{align}

%------------------------------------------------

\section{Section 1.2}

\paragraph{Problem \#14}

Let $f: Z \to \left\{ {-1, 1} \right\}$ be given by
\[  
f(x) =
     \begin{cases}
       \text{1} &\quad\text{if x is even}\\
       \text{-1} &\quad\text{if x is odd}\
     \end{cases}
\]
\begin{enumerate}[a)]
\item Prove or disprove that $f$ is onto
\begin{align}
1, 2 \in \mathbb{Z}\\
f(1) = -1, f(2) = 1\\
\text{Therefore, } \forall x \in \left\{ {-1, 1} \right\} \exists y \in \mathbb{Z} \text{s.t.} f(y) = x                   
\end{align}
\item Prove or disprove that $f$ is one-to-one
\begin{align}
f(2) = 1\\
f(4) = 1
\end{align}
$f$ is not one-to-one because for two values $x,y$ where $x \not= y$, $f(x) = f(y)$.
\item Prove or disprove that $f(x_{1} + x_{2}) = f(x_{1})f(x_{2})$\\
Case 1: $x_{1}$ and $x_{2}$ are even.\\
\begin{align}
\begin{split}
\mathcal P \left({A \cup B}\right)
&= \{X|X \subseteq (A \cup B)\}\\
&=\{X|X \subseteq A\} \cup \{X|X \subseteq B\}\\
&=\mathcal P \left({A}\right) \cup \mathcal P \left({B}\right)
\end{split}                   
\end{align}
\item Prove or disprove that $f(x_{1}x_{2}) = f(x_{1})f(x_{2})$

\end{enumerate}

%----------------------------------------------------------------------------------------
%    PROBLEM 2
%----------------------------------------------------------------------------------------

\section{Lists}

%------------------------------------------------

\subsection{Example of list (3*itemize)}
\begin{itemize}
    \item First item in a list
        \begin{itemize}
        \item First item in a list
            \begin{itemize}
            \item First item in a list
            \item Second item in a list
            \end{itemize}
        \item Second item in a list
        \end{itemize}
    \item Second item in a list
\end{itemize}

%------------------------------------------------

\subsection{Example of list (enumerate)}
\begin{enumerate}
\item First item in a list
\item Second item in a list
\item Third item in a list
\end{enumerate}

%----------------------------------------------------------------------------------------

\end{document}