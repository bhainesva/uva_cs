\documentclass[paper=a4, fontsize=11pt]{jhwhw} % A4 paper and 11pt font size
\usepackage{amsmath,amsfonts,amsthm, amssymb} % Math packages
\setlength\parindent{0pt} % Removes all indentation from paragraphs - comment this line for an assignment with lots of text
\usepackage{graphicx}
\usepackage{mathtools}
\newcommand\SetSymbol[1][]{\:#1\vert\:}
\providecommand\given{} % to make it exist
\DeclarePairedDelimiterX\Set[1]\{\}{\renewcommand\given{\SetSymbol[\delimsize]}#1}

\begin{document}
\title{Survey of Algebra - Assignment \#4}
\author{Ben Haines}

%SECTION 3.2
\section*{Section 3.2}
\problem{\# 15}
Prove that if $x=x^{-1}$ for all $x$ in the group $G$, then $G$ is abelian.
\solution
For any $x,y \in G$ $xy$ is in $G$ by the definition of a group and $(xy)(xy) = e$. Therefore
\begin{align}
\begin{split}
xy &= xey\\
&= x((xy)(xy))y\\
&=(xx)yx(yy)\\
&= eyxe\\
&=yx
\end{split}
\end{align}
Therefore the group is abelian.
\problem{\# 16}
Suppose $ab = ca$ implies $b = c$ for all elements $a, b$, and $c$ in a group $G$. Prove that $G$ is abelian.
\solution
\begin{align}
\begin{split}
ab &= abe\\
&=ab(a^{-1}a)\\
&= (aba^{-1})a
\end{split}
\end{align}
this implies that $b = aba^{-1}$. Therefore 
\begin{align}
\begin{split}
ba &= (aba^{-1})a\\
&= ab(a^{-1}a)\\
&= abe\\
&= ab
\end{split}
\end{align}
and thus $G$ is abelian.

\problem{\# 17}
Let $a$ and $b$ be elements of a group $G$. Prove that $G$ is abelian if and only if $(ab)^{-1} = a^{-1}b^{-1}$.
\solution
Let $G$ be an abelian group. Then ab = ba and
\begin{align}
\begin{split}
(ab)^{-1} &= (ba)^{-1}\\
&= a^{-1}b^{-1}
\end{split}
\end{align}
Now let $a$ and $b$ be elements of a group $G$ such that $(ab)^{-1} = a^{-1}b^{-1}$. Then $(ab)^{-1} = (ba)^{-1}$. Assume that $ab \not = ba$. Thus
\begin{align}
\begin{split}
(ab)^{-1}ab &\not= (ab)^{-1}ba\\
e &\not= (ba)^{-1}ba\\
&\not = e
\end{split}
\end{align}
This is a contradiction. Therefore $ab$ must equal $ba$ and $G$ is therefore abelian.

\problem{\# 18}
Let $a$ and $b$ be elements of a group $G$. Prove that $G$ is abelian if and only if $(ab)^2 = a^2b^2$.
\solution
Let $G$ be an abelian group. Therefore
\begin{align}
\begin{split}
(ab)^2 &= (ab)(ab)\\
&= abab\\
&= a(ba)b\\
&=a(ab)b\\
&=aabb\\
&=a^2b^2
\end{split}
\end{align}
Now let $a$ and $b$ be two elements in a group $G$ such that $(ab)^2 = a^2b^2$.  Then
\begin{align}
\begin{split}
abab&= aabb\\
a^{-1}abab &= a^{-1}aabb\\
ebab &= eabb\\
bab &= abb\\
babb^{-1} &= abbb^{-1}\\
bae &= abe\\
ba &= ab
\end{split}
\end{align}
Therefore $G$ is abelian. 

\problem{\# 19}
Use mathematical induction to prove that if $a$ is an element of a group $G$, then $(a^{-1})^n = (a^n)^{-1}$ for every positive integer $n$.
\solution
\begin{enumerate}
\item For $n=1$
\begin{align}
\begin{split}
(a^{-1})^n &= (a^n)^{-1}\\
(a^{-1})^1 &= (a^1)^{-1}\\
a^{-1} &= a^{-1}
\end{split}
\end{align}
This is true.
\item
Assume the statement is true for $n=k$: $(a^k)^{-1} = (a^{-1})^k$.
\item
For $n=k+1$
\begin{align}
\begin{split}
(a^{k+1})^{-1} &= (aa^{k})^{-1}\\
&= (a^k)^{-1}a^{-1}\\
&= (a^{-1})^ka^{-1}\\
&= (a^{-1})^{k+1}
\end{split}
\end{align}
It follows from the inductive hypothesis that $(a^{-1})^n = (a^n)^{-1}$ for every positive integer $n$.
\end{enumerate}

\problem{\# 20}
Let $a$ and $b$ be elements of a group $G$. Use mathematical induction to prove each of the following statements for all positive integers $n$.
\begin{itemize}
\item If the operation is multiplication, then $(a^{-1}ba)^n = a^{-1}b^na$.
\item If the operation is addition and $G$ is abelian, then $n(a+b) = na + nb$.
\end{itemize}
\solution
\part
\begin{enumerate}
\item
For $n=1$, $(a^{-1}ba)^{1} = a^{-1}b^1a$. This is clearly true.
\item
Assume it is true for $n=k$: $(a^{-1}ba)^{k} = a^{-1}b^{k}a$.
\item
For $n=k+1$
\begin{align}
\begin{split}
(a^{-1}ba)^{k+1} &= (a^{-1}ba)(a^{-1}ba)^{k}\\
&= (a^{-1}ba)(a^{-1}b^{k}a)\\
&=a^{-1}b(aa^{-1})b^{k}a\\
&=a^{-1}beb^{k}a\\
&= a^{-1}bb^{k}a\\
&= a^{-1}b^{k+1}a
\end{split}
\end{align}
\end{enumerate}
It follows from the inductive hypothesis that $(a^{-1}ba)^n = a^{-1}b^na$.

\part
\begin{enumerate}
\item
For $n=1$, $1(a+b) = 1(a) + 1(b)$. This is clearly true.
\item
Assume it is true for $n=k$: $k(a+b) = ka + kb$
\item 
For $n=k+1$
\begin{align}
\begin{split}
(k+1)(a+b) &= k(a+b) + 1(a+b)\\
&= (ka + kb) + (a + b)\\
&= (ka + a) + (kb + b)\\
&= (k+1)a + (k+1)b
\end{split}
\end{align}
It follows from the inductive hypothesis that $n(a+b) = na + nb$.
\end{enumerate}

\problem{\# 22}
Use mathematical induction to prove that if $a_1, a_2, \ldots, a_n$ are elements of a group $G$, then $(a_1a_2\cdots a_n)^{-1} = a_{n}^{-1}a_{n-1}^{-1}\cdots a_2^{-1}a_1^{-1}$. (This is the general form of the reverse order law for inverses.)
\solution
\part
For $n=1$, $(a_1)^{-1} = a_{1}^{-1}$.
\part
Assume the statement is true for $n=k$.
\part
For $n=k+1$
\begin{align}
\begin{split}
(a_1\cdots a_{k+1})^{-1} &= ((a_1\cdots a_{k})a_{k+1})^{-1}\\
&= a_{k+1}^{-1}(a_1\cdots a_{k})^{-1}\\
&= a_{k+1}^{-1}(a_{k}^{-1}\cdots a_1^{-1})\\
&= a_{k+1}^{-1}a_k^{-1}\cdots a_2^{-1}a_1^{-1}
\end{split}
\end{align}
It follows from the inductive hypothesis that the statement is true for all $n\ge 1$.

\problem{\# 23}
Let $G$ be a group that has even order. Prove that there exists at least one element $a\in G$ such that $a \not= e$ and $a = a^{-1}$.
\solution
Assume that for some group $G$ with even order there does not exist an element $a\in G$ such that $a \not= e$ and $a = a^{-1}$. The order of $G$ is even so there exists some integer $n$ such that $o(G) = 2n$. Therefore, selecting some arbitrary $x\in G$, $n$ pairs of inverses can be removed from $G$ until the identity $e$ and $x$ are the only remaining elements of $G$. We know from the definition of a group that $G$ initially contained $x^{-1}$ and, because we removed two elements at a time and we know inverses are unique, we know that we have not removed $x^{-1}$. Therefore either $xe=e$ or $xx=e$. By the definition of $e$ we know that $xe=x$. Therefore $xx=e$ and $x=x^{-1}$. This contradicts our assumption. Therefore, there does exist an element in $G$ that does not equal $e$ and is its own inverse.

\problem{\# 24}
Prove or disprove that every group of order 3 is abelian.
\solution
Let $X=\Set{e, a, b}$ be an arbitrary group of order 3. By the definition of a group, $ab \in X$. Therefore $ab = a$, $ab=b$, or $ab=e$. The first case implies that $b=e$. This is impossible because we know $e$ is unique. The second case is impossible because it implies that $a=e$. Therefore $ab=e$. This means that $a$ and $b$ are each others inverses and $e=ab=ba$. So the group is abelian.

\problem{\# 25}
Prove or disprove that every group of order 4 is abelian.
\solution
Let $X = \Set{e, a, b, c}$ be an arbitrary group of order 4. By definition of a group, $ab \in X$. Therefore $ab = a$, $ab=b$, $ab=c$, or $ab=e$. The first two cases are impossible for the same reason demonstrated in the previous problem. Therefore either $ab=c$ or $ab=e$. If $ab=e$ then $a$ and $b$ are each other's inverses and $e=ab=ba$. If $ab=c$ then $ba$ cannot equal $e$ and must equal $c$. In all cases $ab=ba$. Therefore all groups of order 4 are abelian.

%SECTION 3.3
\newpage
\section*{Section 3.2}
\problem{\# 8}
Find a subset of $\mathbb Z$ that is closed under addition but is not a subgroup of the additive group $\mathbb Z$.
\solution
$\mathbb Z^{+}$

\problem{\# 10}
Let $n > 1$ be an integer, and let $a$ be a fixed integer. Prove or disprove that the set
$$H = \Set{x\in \mathbb Z \given ax \equiv 0 \pmod{n}}$$
is a subgroup of $G$.
\solution
$H$ is not empty because $0\in H$ for all $n$ and $a$. $0$ is the identity element under addition so $H$ always contains $e$.\\
Select arbitrary elements $x, y\in H$. $n\mid xa$ and $n\mid ya$. $(y+z)a = ya+za$ and $n\mid (ya+za)$. Therefore $y+z$ is in $H$ and $H$ is closed under addition.\\
The inverse of $z$ under addition is $-z$. If $n\mid z$ then $n\mid -z$. We know that $n\mid z$ so therefore $n\mid -z$ and $-z$ is also in $H$.\\
$H$ has met all of the necessary requirements and is thus a subgroup of $G$ under addition.

\problem{\# 12}
Prove or disprove that $H = \Set{h\in G \given h^{-1} = h}$ is a subgroup of the group $G$ if $G$ is abelian.
\solution
$e$ is an element of $H$ so therefore $H$ is nonempty and contains the identity element. \\
Each element of $H$ is its own inverse, so every element of $H$ has an inverse.\\
$G$ is abelian so for any $a,b \in H$, $ab=ba$. Therefore
\begin{align}
\begin{split}
ab(ab) &= ab(ba)\\
abab &= aea\\
abab &= aa\\
(ab)(ab) &= e
\end{split}
\end{align}
Therefore $(ab)^{-1} = (ab)$. We know $ab$ is an element of $G$ because $G$ is closed and we have shown that $ab$ is its own inverse. So $(ab) \in H$. Thus $H$ is closed under multiplication.\\
$H$ has satisfied all of the necessary conditions and is thus a subgroup of $G$.

\problem{\# 13}
Let $G$ be an abelian group with respect to multiplication. Prove that each of the following subsets $H$ of $G$ is a subgroup of $G$.
\begin{enumerate}
\item $H = \Set{x\in G \given x^2 = e}$.
\item $H = \Set{x\in G \given x^n = e}$ for a fixed positive integer $n$.
\end{enumerate}
\solution
\part
$x^2=e$ implies that $x^{-1} = x$ and $H$ is a subgroup of $G$ by the result of problem \#12.
\part
$H$ is nonempty because $e^n = e$ so $e\in H$.\\
For two elements $a$ and $b$ in $H$. 
\begin{align}
\begin{split}
(ab^{-1})^{n} &= a^{n}(b^{n})^{-1}\\
&= e(e^{-1})\\
&= e
\end{split}
\end{align}
Thus for any $a,b\in H$, $ab^{-1}\in H$. $H$ has satisfied the necessary conditions and thus is a subgroup of $G$.
\problem{\# 19}
Prove that each of the following subsets $H$ of $SL(2, \mathbb R)$ is a subgroup of $SL(2, \mathbb R)$.
\begin{enumerate}
\item 
$H = \left\{\left[\begin{array}{cc}
1 & a  \\
0 & 1   \end{array}\right] \middle| a\in \mathbb{R} \right\}$
\item
$H = \left\{\left[\begin{array}{cc}
a & -b  \\
b & a   \end{array}\right] \middle| a^2 + b^2 = 1\right\}$
\end{enumerate}
\solution
\part
The inverse matrix 
$\left[\begin{array}{cc}
1 & 0  \\
0 & 1   \end{array}\right] $ is an element of $H$. \\
$$\left[\begin{array}{cc}
1 & a_1  \\
0 & 1   \end{array}\right]
\left[\begin{array}{cc}
1 & a_2  \\
0 & 1   \end{array}\right] 
=
\left[\begin{array}{cc}
1 & a_1+a_2  \\
0 & 1   \end{array}\right]$$
$a_1$ and $a_2$ are both in $\mathbb R$ so $a_1 + a_2$ is also in $\mathbb R$.
$1(1)-0(a_1+a_2$, so $H$ is closed under multiplication.\\
For any $a_1$
$$\left[\begin{array}{cc}
1 & a_1  \\
0 & 1   \end{array}\right]
\left[\begin{array}{cc}
1 & -a_1  \\
0 & 1   \end{array}\right] 
=
\left[\begin{array}{cc}
1 & 0  \\
0 & 1   \end{array}\right] = e$$
If $a_1$ is in $\mathbb R$ then so is $-a_1$ so every member of $H$ has an inverse in $H$.\\
$H$ has satisfied the necessary requirements and therefore is a subgroup of $SL(2, \mathbb R)$.

\part
The inverse matrix 
$\left[\begin{array}{cc}
1 & 0  \\
0 & 1   \end{array}\right] $ is an element of $H$. \\
$$\left[\begin{array}{cc}
a_1 & -b_1  \\
b_1 & a_1   \end{array}\right]
\left[\begin{array}{cc}
a_2 & -b_2  \\
b_2 & a_2   \end{array}\right] 
=
\left[\begin{array}{cc}
a_1a_2-b_1b_2 & -a_1b_2-b_1a_2  \\
b_1a_2 + b_2a_1 & -b_1b_2 + a_1a_2   \end{array}\right]$$\\
\begin{align}
\begin{split}
(a_1a_2 - b_1b_2)^2 + (b_1a_2 + b_2a_2)^2 &= (a_1a_2)^2 - 2a_1a_2b_1b_2 + (b_1b_2)^2 + (b_1a_2)^2 + 2b_1a_2b_2a_1 + (b_2a_1)^2\\
&= a_1^2a_2^2 + b_1^2b_2^2 -2a_1a_2b_1b_2 + b_1^2a_2^2 + b_2^2a_1^2 + 2b_1a_2b_2a_1\\
&= a_1^2a_2^2 + b_1^2a_2^2 + b_1^2b_2^2 + b_2^2a_1^2\\
&= a_2^2(a_1^2 + b_1^2) + b_2^2(b_1^2 + a_2^2)\\
&= a_2^2(1) + b_2^2(1)\\
&= a_2^2 + b_2^2\\
&= 1
\end{split}
\end{align}
Therefore $H$ is closed under multiplication.\\

For any values of $a$ and $b$
$$\left[\begin{array}{cc}
a & -b  \\
b & a   \end{array}\right]
\left[\begin{array}{cc}
a & b  \\
(a^2-1)b^{-1} & (1-b^2)a^{-1}   \end{array}\right] 
=
\left[\begin{array}{cc}
1 & 0  \\
0 & 1   \end{array}\right]$$\\

\begin{align}
\begin{split}
a(1-b^2)a^{-1} - b(a^2-1)b^{-1} &= 1-b^2 - a^2 + 1\\
&= 1-(b^2 + a^2) + 1\\
&= 1 - 1 + 1\\
&= 1
\end{split}
\end{align}
So the inverse is an element of $H$.\\
Having satisfied all of the necessary conditions, $H$ is a subgroup of $SL(2, \mathbb R)$.


\end{document}