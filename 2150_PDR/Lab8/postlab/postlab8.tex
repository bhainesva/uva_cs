\documentclass{article}
\usepackage{fancyhdr}
\usepackage{hyperref}
\usepackage{amsmath}
\usepackage{color}
\usepackage{listings}

\definecolor{dkgreen}{rgb}{0,0.6,0}
\definecolor{gray}{rgb}{0.5,0.5,0.5}
\definecolor{mauve}{rgb}{0.58,0,0.82}

\lstset{frame=tb,
  language=C++,
  aboveskip=3mm,
  belowskip=3mm,
  showstringspaces=false,
  columns=flexible,
  basicstyle={\small\ttfamily},
  numbers=none,
  numberstyle=\tiny\color{gray},
  keywordstyle=\color{blue},
  commentstyle=\color{dkgreen},
  stringstyle=\color{mauve},
  breaklines=true,
  breakatwhitespace=true,
  numbers=left,
  tabsize=3
}

\pagestyle{fancy}
\fancyhf{}
\chead{inlab8.pdf}
\lhead{Ben Haines, bmh5wx}
\rhead{11/4/14}
\begin{document}
\section*{Parameter Passing}
\subsection*{1. Passing an int, char, etc.}
\begin{lstlisting}
int foo(int x) {
    return x+3;
}

int main() {
    foo(4);
    return 0;
}
\end{lstlisting}
\lstset{language={[x86masm]Assembler}}
When compiled into assembly, the above code produces:
\begin{lstlisting}
_Z3fooi:
	push	ebp
	mov	ebp, esp
	mov	eax, DWORD PTR [ebp+8]
	add	eax, 3
	pop	ebp
	ret
main:
	push	ebp
	mov	ebp, esp
	push	4
	call	_Z3fooi
	add	esp, 4
	mov	eax, 0
	leave
	ret
\end{lstlisting}
The caller, main(), pushes the argument onto the stack and then calls the foo subroutine. After the standard prologue, the callee then copies the argument into the eax register. The callee accesses the argument by a constant difference from the ebp pointer.

The same function modified to take a char rather that an int functions in an almost identical way. The only difference is that, because a char is a smaller tpye than an int, precautions are taken to make sure that when the argument is copied into the eax register no extra data is accidentally used. 

\subsubsection*{Passing int by reference}
When modified slightly to take a reference to an int as an argument, the code compiles to:
\begin{lstlisting}
_Z3fooRi:
	push	ebp
	mov	ebp, esp
	mov	eax, DWORD PTR [ebp+8]
	mov	eax, DWORD PTR [eax]
	add	eax, 3
	pop	ebp
	ret
main:
	push	ebp
	mov	ebp, esp
	sub	esp, 20
	mov	DWORD PTR [ebp-4], 4
	lea	eax, [ebp-4]
	mov	DWORD PTR [esp], eax
	call	_Z3fooRi
	mov	eax, 0
	leave
	ret
\end{lstlisting}
Rather that pushing the number four to the stack to be used directly as an argument like before, it now puts four on the stack and then pushes the address of 4 which is taken as the argument to foo. Then foo dereferences the argument in order to use to value 4. This approach of passing and then dereferencing an address is how passing any data-type by reference works.
\subsubsection*{Passing a pointer}
\begin{lstlisting}
int foo(int * x) {
    return *x+3;
}

int main() {
    int * c = new int;
    *c = 3;
    foo(c);
    return 0;
}
\end{lstlisting}
Compiles to:
\begin{lstlisting}
_Z3fooPi:
	push	ebp
	mov	ebp, esp
	mov	eax, DWORD PTR [ebp+8]
	mov	eax, DWORD PTR [eax]
	add	eax, 3
	pop	ebp
	ret
main:
	lea	ecx, [esp+4]
	and	esp, -16
	push	DWORD PTR [ecx-4]
	push	ebp
	mov	ebp, esp
	push	ecx
	sub	esp, 20
	sub	esp, 12
	push	4
	call	_Znwj
	add	esp, 16
	mov	DWORD PTR [ebp-12], eax
	mov	eax, DWORD PTR [ebp-12]
	mov	DWORD PTR [eax], 3
	sub	esp, 12
	push	DWORD PTR [ebp-12]
	call	_Z3fooPi
	add	esp, 16
	mov	eax, 0
	mov	ecx, DWORD PTR [ebp-4]
	leave
	lea	esp, [ecx-4]
	ret
\end{lstlisting}
Here, the call to \_Znwj creates a pointer to the int 4, and returns it in the eax register. In line 21 it takes the value of eax, which is an address that contians the value 4, and stores it in [ebp-12] which is space allocated by main for local variables. Then, [ebp-12] is pushed on the stack so it can be used as an argument when the call to foo() is made in the next line. Within foo, the variable [ebp-12] (an address pointing to the integer 4) is moved to the eax register. Then eax is dereferenced to get the value being pointed to and the function procedes from there on the same as when passing a normal int.

\subsubsection*{Float}
When the function is modified to take a floating point number, the argument is passed and accessed in the same way as an int. There is additional complexity involved with creating the floating point number but the general procedure of pushing an argument to the stack then accessing it by an offset from [ebp] is unchanged. 

\subsubsection*{Object}
The following code passes a simple user defined object to a function.
\begin{lstlisting}
struct simple {
  int x;
  int y;
} test;

int foo(simple z) {
    return z.x;
}

int main() {
    test.x = 3;
    test.y = 4;
    foo(test);
    return 0;
}
\end{lstlisting}
It compiles to the following assembly.
\begin{lstlisting}
test:
_Z3foo6simple:
	push	ebp
	mov	ebp, esp
	mov	eax, DWORD PTR [ebp+8]
	pop	ebp
	ret
main:
	push	ebp
	mov	ebp, esp
	mov	DWORD PTR test, 3
	mov	DWORD PTR test+4, 4
	push	DWORD PTR test+4
	push	DWORD PTR test
	call	_Z3foo6simple
	add	esp, 8
	mov	eax, 0
	leave
	ret
\end{lstlisting}
As can be seen, the members of the object are pushed to the stack in reverse order and then accessed as normal arguments by an offset from [ebp].

\subsection*{2. Arrays}
The same procedure is used for passing arrays as arguments. For example, the following c++ code:
\begin{lstlisting}
int foo(int x[]) {
    return x[0];
}

int foo2(int x[]) {
    return x[1];
}

int foo3(int x[]) {
    return x[2];
}

int main() {
    int y[] = {1, 2, 3};
    foo(y);
    foo2(y);
    foo3(y);

    return 0;
}
\end{lstlisting}
Compiles to:
\begin{lstlisting}
_Z3fooPi:
	push	ebp
	mov	ebp, esp
	mov	eax, DWORD PTR [ebp+8]
	mov	eax, DWORD PTR [eax]
	pop	ebp
	ret
_Z4foo2Pi:
	push	ebp
	mov	ebp, esp
	mov	eax, DWORD PTR [ebp+8]
	mov	eax, DWORD PTR [eax+4]
	pop	ebp
	ret
_Z4foo3Pi:
	push	ebp
	mov	ebp, esp
	mov	eax, DWORD PTR [ebp+8]
	mov	eax, DWORD PTR [eax+8]
	pop	ebp
	ret
main:
	push	ebp
	mov	ebp, esp
	sub	esp, 20
	mov	DWORD PTR [ebp-12], 1
	mov	DWORD PTR [ebp-8], 2
	mov	DWORD PTR [ebp-4], 3
	lea	eax, [ebp-12]
	mov	DWORD PTR [esp], eax
	call	_Z3fooPi
	lea	eax, [ebp-12]
	mov	DWORD PTR [esp], eax
	call	_Z4foo2Pi
	lea	eax, [ebp-12]
	mov	DWORD PTR [esp], eax
	call	_Z4foo3Pi
	mov	eax, 0
	leave
	ret
\end{lstlisting}
The base of the array is accessed as [ebp+8]. In order to access the second element of the array (i.e. index 1), the base is found and then offset by the size of a single element. So the second element is at [ebp+8], the third is the base offset by two elements and is locate at [ebp+12], and so on.

\subsection*{3. Pointers vs. References}
Passing values by reference works by pushing an argument to the stack that contains the address of the actual object. The argument is then dereferenced inside the callee to access the object itself. The implementation of pointers and references are identical in assembly. 

\subsubsection*{Summary}
In general there are two ways that arguments are passed to functions. The first, by value, involves simply pushing the argument to the stack. It is then accessed from within the callee by a memory offset from [ebp] calculated based on the size of the argument. The second, by reference, involves storing the argument somewhere in memory ad then pushing the address of its location to the stack before making the subroutine call. Then inside the callee the address of the argument is located by an offset from [ebp]. Then, the address is dereferenced using "[ ]" in order to obtain the actual value of the argument.

\section*{Objects}
\subsection*{1. Overview}
The memeber variables of an object are pushed to the stack in consecutive memory locations. The compiler may insert buffer space between the members in order to make access more efficient but must preserve order within access blocks. Data fields are then accessed like elements of an array, by offsetting from the address of the base of the object.

The following c++ code defines a simple class that contains two integers, a char, a bool, and a user defined struct as member variable. The class has a single public function that changes the value of the first int. The main method then creates an instance of the class and calls the function. It will be used as an example to demonstrate data layout, access of data members, and access of member functions.

\begin{lstlisting}
struct simple {
  int x;
  int y;
};

class Test {
    int x;
    bool b;
    char c;
    simple s;

  public:
    void setX(int a);
    int y;
};

void Test::setX(int a) {
    this->x = a;
}

int main() {
    Test myTest = Test();
    myTest.setX(3);
    myTest.y = 2;
    return 0;
}



\end{lstlisting}
It compiles to the following.
\begin{lstlisting}
_ZN4Test4setXEi:
	push	ebp
	mov	ebp, esp
	mov	eax, DWORD PTR [ebp+8]
	mov	edx, DWORD PTR [ebp+12]
	mov	DWORD PTR [eax], edx
	pop	ebp
	ret
main:
	push	ebp
	mov	ebp, esp
	sub	esp, 32
	mov	DWORD PTR [ebp-20], 0
	mov	BYTE PTR [ebp-16], 0
	mov	BYTE PTR [ebp-15], 0
	mov	DWORD PTR [ebp-12], 0
	mov	DWORD PTR [ebp-8], 0
	mov	DWORD PTR [ebp-4], 0
	push	3
	lea	eax, [ebp-20]
	push	eax
	call	_ZN4Test4setXEi
	add	esp, 8
	mov	DWORD PTR [ebp-4], 2
	mov	eax, 0
	leave
	ret
\end{lstlisting}
\subsection*{2. Data Layout Sample}
As can be seen above, data is stored in a very familiar way. The data members are moved to the stack in memory locations reserved for local variables so that they can be popped in the same order as they are listed in the class definition. The struct is placed on the stack, just as it would be as an argument, by pushing each of its components in reverse order. In this example [ebp-20] is the address of test.$x$, [ebp-16] of $b$, [ebp-15] of $c$, [ebp-12] of simple.$x$, [ebp-8] of simple.$y$, and [ebp-4] of test.$y$. Notice that although $b$ only requires one byte of memory, the compiler is allowed to leave space between it and test.$x$.
There is no fundamental difference between how the private and public variables are stored in memory. 

\subsection*{4. Public Member Function Sample}
The member method 'setX(int)' is stored in memory just like a global function. The method needs to know what instance of the class it is being called on. In order to make this possible, a 'this' pointer is passed as an argument. The pointer is the memory address of the memory location where the particular instance is stored. The pointer itself doesn't need to be stored seperately because it is simply the address of the first member of the object. Every time a method is called on a different instance the this pointer is different.

\subsection*{3. Data Access Sample}
In the example above, data is accessed in two ways. First, the private variable test.$x$ is accessed through the method 'setX(int).' Second, the public variable test.$y$ is set directly. The second case is the simpler one, in order to set test.$y$ the assembly simply moves the desired value into the memory location allocated for $y$ ([ebp-4] as mentioned previously). This happend in line 24. The first case involves calling a method of the class which is discussed in the previous paragraph. It uses the 'this' pointer to access the location of the class instance. In line 4 it loads this address into the register eax. In line 5 it loads the explicit argument 3 into the register edx. It then moves the value of edx (3) into the location pointed to by eax (test.$x$.). This requires three steps in order to avoid accessing memory twice in once cycle.

\section*{References}
The following links were used when learning about their respective topics.
\begin{itemize}
    \item http://stackoverflow.com/questions/405112/how-are-objects-stored-in-memory-in-c
    \item http://stackoverflow.com/questions/12378271/what-does-an-object-look-like-in-memory
    \item http://stackoverflow.com/questions/1632600/memory-layout-c-objects
    \item http://docs.oracle.com/cd/E19455-01/806-3773/6jct9o0af/index.html
    \item http://cs.lmu.edu/~ray/notes/nasmexamples/
\end{itemize}

\end{document}
