\documentclass{article}
\usepackage{fancyhdr}
\usepackage{hyperref}
\usepackage{amsmath}
\usepackage{color}

\pagestyle{fancy}
\fancyhf{}
\chead{postlab7.pdf}
\lhead{Ben Haines, bmh5wx}
\rhead{10/26/14}
\begin{document}
\section{Thoughts on IBCM}
Using IBCM was interesting. It was my first exposure to a language that allows programs to modify themselves as they run, and where this is an important feature for doing many sorts of useful tasks. The things we had to do with it were fairly simple and made it feel more like solving a puzzle than writing a program. That part of it was fun, but trying to do more complicated things I can imagine would be very unpleasant.

It was not very easy to use. Formatting the programs with multiple columns of text was tedious. Probably the least pleasant part was when I completed a program and then realized I needed to add another variable at the beginning. I then had to readjust all of the memory references in the rest of the program to fit it in. The simulator wasn't very helpful either. Things that would have made it easier would be a breakpoint feature, not crashing the browser when it got stuck in loops, and a way to step backwards in execution. Something like automatically generated comments that give an overview of what the program was doing would be nice as well. When testing things I had to keep referring back to the commented version I had written to make sure I knew what was going on at any given time.

I'm pretty confident that given a simple task I could eventually produce IBCM code to complete it. I'm not very confident that I could handwrite code that would function correctly on the first try.

\end{document}
